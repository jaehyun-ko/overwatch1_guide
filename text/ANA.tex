Part 1. 아나는 어떤 영웅인가?

 

아나를 한 단어로 요약하면 ‘만능힐러’입니다. 아나는 힐링, 딜링, 생존기까지 힐러에게 필요한 모든 능력을 갖추고 있습니다. 그러나 아나의 플레이 목적을 이해하지 못한다면 자기도 모르는 사이 트롤을 하고 있을 수도 있습니다. 따라서 시작에 앞서 아나가 해야 할 일을 간단하게 정리해보겠습니다.

 

1. 탱커를 담당하는 힐러

 

아나의 가장 주된 임무는 탱커들의 체력관리입니다. 루시우와 젠야타는 탱커를 채우기에 힐량이 부족하고, 메르시는 전방의 탱커에게 접근 할 때 리스크를 감수해야 합니다. 반면 아나는 높은 단일 힐량, 원거리 힐, 생체수류탄을 통해 탱커들의 체력을 빠르고 안전하게 채울 수 있습니다. 아군의 탱커가 앞에서 버티고 있어야 팀의 전체적인 대형이 유지되기 때문에 아나는 항상 아군 탱커라인의 체력상황을 신경써야합니다. 혹시라도 아나가 저격하다가 아군 탱커가 죽는 불상사는 절대 일어나서는 안 됩니다.



2. 본대와 떨어져 있는 저격수

 

아나는 ① 아군을 한 눈에 볼 수 있고 ② 본인의 생존을 도모할 수 있는 위치에 있어야 합니다. 맵과 공&수에 따라 다르지만 보통 본대와 떨어진 위치에서 자리 잡고 안전하게 힐을 줄 때 위 두 가지 조건을 모두 충족할 수 있습니다. 아나의 포지셔닝에 대한 자세한 내용은 이어지는 2장에서 설명하겠습니다.

 

3. 1:1에 강한 영웅. 하지만 힐러의 본분을 망각해서는 안된다.

 

아나는 1:1에 정말 강한 스킬구성(생체수류탄, 수면총)을 가지고 있습니다. 따라서 아나를 물러온 겐지나 트레이서 등과 1:1을 할 경우 높은 확률로 이길 수 있습니다. 그러나 아나가 1:1을 하는 과정에서 아군에게 힐을 주지 못하게 되면 아군 전체의 유지력에 문제가 생깁니다. 예를 들어 한 쪽이 겐지가 없고 다른 쪽이 아나가 없이 남은 영웅들끼리 5:5를 하게 되면 아나가 없는 쪽이 더 불리합니다. 따라서 적 영웅이 아나를 물러오면 혼자 제압하려고 시도하기보다는 팀원에게 커버를 요청해서 적을 도망가게 만들거나, 팀원에게 해당 적 영웅의 견제를 맡긴 뒤 아나는 다시 안전한 포지션을 잡는 것이 좋습니다.

 

4. 저격을 통한 적 주요 영웅 견제

 

아나의 저격은 확실한 킬을 만들어내기 보다는 상대에게 압박감을 심어주고 행동반경을 제한하는 효과가 있습니다. 예를 들어서 아군 아나가 상대방의 겐지나 트레이서를 필요할 때마다 저격으로 맞춰주게 되면 그들은 ‘상대 아나가 잘 쏜다’라는 생각을 할 것입니다. 한 번 이런 생각을 하게 되면 이후부터는 항상 아나한테 저격당할 가능성을 생각하고 움직이기 때문에 적 딜러의 활동반경을 제한해주는 역할을 할 수 있습니다.

 

두 번째로 아나가 저격으로 잡아줘야 하는 대표적인 영웅으로는 파르시가 있습니다. 현재 파르시의 위상이 많이 올라갔고 숙련된 파라는 고도조절과 은폐 & 엄폐의 효율적인 활용을 하기 때문에 솔져와 맥크리를 둘 다 뽑아도 견제하기 힘든 경우도 있습니다. 반면 아나는 전장에 대한 시야가 넓기 때문에 파르시가 높이 뜨거나 뒤로 돌아도 비교적 쉽게 위치를 파악하고 견제할 수 있기 때문에 아군 솔&맥과 더불어 함께 파르시를 견제해줘야 합니다.

 

 

요약하자면 아나는 ‘하이리스크, 하이리턴’ 영웅입니다. 죽어가는 아군을 살리고, 적의 궁극기를 무력화시키고, 끊임없이 변수를 창출해 낼 수 있습니다. 1:1 전투에서도 강하고 생존능력 또한 탁월합니다. 이렇게 보면 사기영웅이 따로 없습니다.

 

그러나 뒤집어서 생각해 보면 아나가 가진 건 권리가 아니라 책임입니다. 죽어가는 아군을 살려야하고, 적의 궁극기를 무력화시켜야 하고, 끊임없이 변수를 만들어내야 합니다. 또한 아나는 상대입장에서 최우선 제거대상이기 때문에 게임 내내 상대의 견제를 받을 것입니다. 그 과정에서 아나가 죽게 되면 팀 전체의 유지력과 전투력에 엄청난 문제가 생기게 됩니다. 아나는 OP가 맞습니다. 그러나 그만큼 어깨가 무겁다는 걸 항상 인지하고 있어야합니다.

 

 

 

Part 2. 아나의 포지셔닝

 

서론에서 언급한 것처럼 아나의 포지셔닝은 ① 아군을 한 눈에 볼 수 있고 ② 본인의 생존을 도모할 수 있는 위치이면서 ③ 적군의 진입로를 확인할 수 있는 곳이 가장 이상적입니다. 이런 위치는 전장의 종류 (화물, 거점, 쟁탈)과 공격, 수비시에 조금씩 달라집니다. 따라서 전장의 종류와 공, 수를 나눠서 아나의 포지셔닝에 대해 설명해 드리겠습니다. 글의 마지막에서 부록으로 스크린샷과 함께 각 맵별로 어디가 아나가 포지션을 잡기 좋은 위치인지 추가적으로 설명해 드리겠습니다.

 

1. 화물맵과 거점맵에서의 포지셔닝

 

① (수비팀) 적의 이동기 한 번으로 접근할 수 없는 위치

아나는 1:1에 능하지만 1:多 상황에서는 무력합니다. 그렇기 때문에 아나가 적의 이동기(ex. 점프팩, 질풍참, 부스터)로 접근할 수 있는 위치에 자리 잡게 되면 둘 이상의 적이 한 번에 아나를 물러 진입했을 때(ex. 겐지와 원숭이가 같이 들어오는 경우) 아무런 대처도 하지 못하고 죽을 가능성이 높습니다. 따라서 아나는 적의 이동기로 접근이 가능한 것보다는 더 멀리 위치해야 하며 이 경우 아나를 물러오는 적은 더 큰 리스크를 안고 진입해야 합니다. 따라서 얼마나 뒤에 있어야 할까에 대한 기준이 필요하다면 ‘이동기로 한 번에 접근할 수 없는 거리’ 정도를 기준으로 세우시면 좋습니다.

 

② (수비팀) 은폐 & 엄폐가 가능한 2층 지형

공격팀 입장에서 수비팀 아나가 2층 후방을 잡고 있으면 상당히 까다롭다는 걸 다들 경험으로 아실 겁니다. 아나가 프리힐을 넣고 있는데 견제할 방법이 마땅치 않기 때문입니다. 그런데 역설적으로 2층을 먼저 잡고 있는 수비팀 아나 입장에서 가장 까다로운 건 공격팀 아나입니다. 원거리 저격이 가능한 공격팀 아나는 탱커들의 보호를 받으면서 후방 2층에 위치한 수비팀 아나를 견제해줄 수 있기 때문입니다. 이 때 수비팀 아나 입장에서 주변에 엄폐물이 없다면 기껏 잡은 2층 포지션을 포기해야합니다. 반면 주변에 엄폐물이 있다면 공격팀 아나가 자신을 볼 때는 잠깐 숨었다가 포커싱이 없어지면 다시 자리를 잡고 아군을 지원할 수 있습니다.

 

③ (수비팀) 주변에 힐팩이 있는 곳

아나의 체력회복 수단은 ① 다른 힐러 ② 생체수류탄 ③ 힐팩이 있습니다. 언뜻 보면 힐팩의 중요성이 떨어져 보이지만 그렇지 않습니다. 먼저 아나가 후방에 위치한 경우라면 다른 힐러로부터 힐을 받기 힘듭니다. 두 번째로 생체수류탄은 항상 들고 있을 거라고 보장 할 수 없습니다. 아나가 자기가 물릴 수도 있다는 생각이 들면 생체수류탄을 아끼겠지만 항상 알고 물리는 건 아니기 때문에 이미 생체수류탄이 빠졌을 수도 있습니다. 이 시점에서 주변에 힐팩이 있다면 먹으면서 한 턴을 더 버티고 수면총과 생체수류탄이 돌아오는 시점에 한 번 더 기회를 잡아볼 수 있습니다.

 

④ (수비팀) 도주로에서 수면총을 맞추기 용이한 곳

아나는 게임 내내 적군 누군가가 물러올 겁니다. 아나가 1:1에 강하다고 말씀드렸지만 실제 게임에서는 아나가 살기위해 도망쳐야 하는 상황이 더 많습니다. 이 경우 자신을 물러온 적에게 수면총을 적중시키고 도망치는 게 가장 좋습니다. 그런데 본인이 아무리 끌어치기에 능하다고 해도 개활지에서 움직이는 적을 맞추는 건 쉽지 않습니다. 따라서 아나가 도망치는 루트에 좁은 통로, 꺾이는 길, 혹은 엄폐물이 있는 게 좋습니다. 이 때 아나는 생체수류탄을 활용해 한 턴 버틴 뒤 각이 좁아지는 곳에서 수면총을 적중률을 높일 수 있습니다.

 

⑤ (수비팀) 적의 주요 우회로 및 아나를 물러오는 적의 위치가 보이는 곳

전면에서 싸우는 아군들은 뒤치기에 무력해 지는 경우가 종종 있습니다. 반면 아나의 경우 뒤에서 포지션을 잡고 있기 때문에 뒤치기를 하는 영웅들을 볼 수 있습니다. 이 때 해당 적군이 아나를 잡으러 온다면 수면총 & 생체수류탄을 활용해서 생존함과 동시에 아군에게 지원을 요청하면 되고 반대로 아군에게 간다면 브리핑을 통해 알려주면 됩니다. 다시 말해 아군전체를 조망하면서도 샛길을 모두 볼 수 있는 곳이 수비팀 아나에게 가장 좋은 위치라고 할 수 있습니다.

 

⑥ 공격팀 아나의 포지셔닝

수비팀 아나와는 달리 공격팀 아나의 포지션은 유동적입니다. 아군을 한 눈에 다 볼 수 있는 곳이 좋다는 전제는 같습니다. 그러나 너무 뒤에 있으면 오히려 고립되서 죽을 가능성이 높고 그 경우 팀원들의 보호도 받기 힘듭니다. 반면 아군 탱커나 딜러와 너무 근접하면 중력자탄과 망치 등에 같이 휘말릴 수 있습니다. 따라서 적정거리를 유지하면서 아군을 따라가는 게 좋습니다.

 

동시에 공격팀 아나의 주 임무중 하나는 생체수류탄과 수면총으로 끊임없이 변수를 만드는 것입니다. 공격팀 아나는 수비때와는 다르게 안전이 어느정도 보장되어 있기 때문에 기회가 있을 경우 적극적으로 나가서 스킬을 활용해 줄 필요가 있습니다. 공격팀 아나가 최후방 포지션으로 이동하는 하는 경우는 전투가 펼쳐지기 직전입니다. 아나는 난전에 말려들어서는 안 되며 전투가 열리기 전에 뒤로 빠져있는 게 좋습니다. 필요한 경우 난전상황에서는 쟁탈전과 마찬가지로 아군 한 명을 집중력으로 관리해야 할 때도 있습니다.

 

아나가 전방으로 나가야 할 때는 화물맵에서 팀원들이 상대 리스폰을 꼬기 위해 추격하러 갈 때입니다. 다른 힐러 한 명은 화물에 붙어있고 아나가 팀원들이 한 눈에 보이는 고지대로 올라가서 전방에 있는 아군들을 백업해 주는 게 좋습니다. 요약하자면 수비팀 아나에게 제일 중요한건 본인의 생존이 용이한 자리를 잡고 팀원들을 지원해 주는 것이라면 공격팀 아나는 지원과 함께 지속적으로 변수창출을 해야 하는 과제를 가지고 있다고 생각하시면 될 것 같습니다.

 

⑦ (공격팀) B거점 공략시 본대와 합류해서는 안 된다.

공격팀 입장에서 B거점 공략의 핵심은 아군 아나를 지키는 것입니다. 단순히 공략을 하는 상황이나 수비팀이 비비면서 버티는 상황 둘 다 마찬가지입니다. 만약 공격팀 아나가 죽으면 팀의 유지력이 급격하게 떨어지게 되고 거점 공략은 실패로 돌아갈 가능성이 높습니다. 특히 수비팀이 비비는 중이라면 공격팀 아나를 끊어내지 못하면 결국 거점을 뚫릴 가능성이 높습니다.

 

그렇기 때문에 공격팀 아나는 B거점 공략시 수면총과 수류탄을 활용한다는 명분으로 본대와 합류하기 보다는 뒤에 떨어져서 아군 체력을 보충해 주는 것이 좋습니다. 수면총과 생체수류탄은 후방포지션에서도 충분히 활용이 가능합니다. 보통 수면총은 적의 궁극기를 끊거나 원숭이나 디바등이 아나를 물러 올 때를 대비해서 아껴두고 생체수류탄은 멀리서 힐밴위주로 던져주시면 됩니다.

 

아래는 제가 글로 쓴 내용을 정리해 둔 영상이니 잠깐만 시간을 내서 봐주세요 :)




2. 쟁탈전에서의 포지셔닝
 

쟁탈전 아나는 극한직업입니다. 일단 안전한 위치가 거의 없고 상대팀은 어디서나 아나를 물러 올 수 있습니다. 난전이 많고 지형이 협소해서 아군을 한 눈에 조망할 수 있는 위치도 잡기 힘든 편입니다. 따라서 제가 알려드리는 기준은 항상 옳은 것은 아니며 그 때 그 때 주어진 상황에서 판단해야 한다는 점을 먼저 말씀드리겠습니다.

 

① 적의 진입로와 반대방향

쟁탈전에서는 적이 진입할 수 있는 루트가 두 개 이상 존재합니다. 따라서 적이 몰려온다면 아나는 아군을 중심에 놓고 그 반대방향으로 순환하는 게 좋습니다. <적군> - <아군> - <아나>의 구도가 형성되면 아나가 비교적 안전하게 힐을 넣을 수 있습니다.

 

② 아군 탱커를 방패삼아 움직이자

난전 상황에서 아군을 다 지원해 줄 수 없을 것 같다며 아군 탱커를 방패삼아 움직이세요. 이 포지션에서는 탱커를 집중적으로 케어해주면서 수면총과 생체수류탄으로 지속적으로 변수를 만들어 낼 수 있도록 노력해야 합니다. 또한 시야가 많이 좁아진 상태이기 때문에 항상 주위를 둘러보고 힐이 필요한 아군이 있다면 빠르게 지원해야합니다.

 

③ 아군 한 명을 집중적으로 케어(관리)해주는 포지셔닝

쟁탈전에서 아나의 주된 전투방식중 하나는 마치 메르시처럼 아군 한 명을 집중적으로 관리해주는 것입니다. 한 명의 아군을 따라다니면서 모든 스킬을 활용해서 아군을 살리고 적을 제압하는 걸 돕습니다. 상대에게 포커싱 당하는 아군을 케어하면서 역으로 상대를 제압할 수 도 있고, 잘 쏘는 아군 딜러를 케어해줄 수도 있습니다. 특히 자리야나 로드호그, 디바같은 딜탱 영웅들을 시기적절하게 케어해주면 적의 어그로를 다 받으면서도 죽지 않고 딜을 넣는 모습을 보입니다.

 

그러나 처음에 말씀드린 것처럼 어디까지나 이론일 뿐입니다. 쟁탈전은 예측불가능한 변수가 너무 많고 아나가 언제 물릴지 모릅니다. 화물맵에서는 아나가 물리면 도움이라도 요청할 수 있지만 쟁탈전에서는 아군도 다 바쁜 경우가 많습니다. 그러니까 아나가 살려면 상대편 눈치를 정말 잘 봐야 합니다. 야생의 감으로 닥쳐올 위험을 최대한 사전에 피하지 않으면 아나는 팀의 승패와 관련 없이 고통받을겁니다.

 

 

 

Part 3. 아나의 패줌 활용

 

아나 꿈나무 여러분들. 혹시 패줌에 환상을 품고 계신가요? 그런데 패줌은 그렇게 대단한 기술이 아닙니다. 다들 저격모드(줌)을 사용하는 이유가 히트스캔(즉발판정) 때문인 걸 알고 계실 겁니다. 패(스트)줌은 이름 그대로 단순히 줌 - 저격 - 줌해제를 빨리 하는 것 뿐입니다.

 

아나가 패줌을 쓰는 가장 큰 이유는 시야 때문입니다. 아나는 항상 전장을 넓게 봐야 하기 때문에 줌을 키고 있는 시간을 줄일수록 리스크가 줄어듭니다. 아나가 해야 할 일들에 대해서 기본적인 것만 생각해볼까요? 아군 체력관리, 생체수류탄 & 수면총 활용, 아나를 물러오는 적 대처, 전투위치에 따른 포지션 조정, 상대의 궁상태 체크 및 활용위치 파악, 오더, 브리핑 등이 있습니다. 하나 같이 전체적인 시야가 필요한 일들입니다. 따라서 줌을 키고 저격을 하는 행위(힐,딜 모두)는 아나의 전반적인 플레이를 제한하게 됩니다. 따라서 전장전체에 대한 시야를 확보하면서 힐 or 딜을 히트스캔으로 넣어주기 위해서 패줌을 사용하는 것입니다.

 

패줌을 쓰는 방식은 일반 줌을 쓰는 방식과 마찬가지로 ①줌 ②조준 ③발사 ④줌 끄기입니다. 이후 조준과정에서 끌어치기가 익숙해지면 조준시간을 줄이면서 빠른 끌어치기를 할 수 있게 됩니다. 여기가 두 번째 단계입니다. 이후 감도와 줌 시간에 적응이 되면 별도의 끌어치기 없이 줌모션 과정에서 에임을 조정하고 줌이 켜지는 순간 바로 저격을 하게 되는데 이게 흔히 알고 있는 패줌입니다. 아나가 줌을 키는 모션에서 잠깐 조준점이 사라지는데 그 상황에서 조준을 하고 줌이 켜지면 바로 쏜다고 생각하시면 됩니다.

 

여기서 한 단계 발전하게 되면 점프 패줌을 사용하게 됩니다. 점프 → 줌 → 발사로 이루어지는 점프 패줌은 앉아서 줌을 사용할 때 생기는 이동속도 패널티를 없애준다는 점에서 매력적입니다. 혹시 힐을 줘야 하는 아군이 고지대에 있다면 점프하는 만큼의 높이 힐을 줄 수 있기도 하죠. 다만 패줌은 숙련도가 생길 때까지 시간이 꽤 필요합니다. 특히 노줌으로 쏘기 애매한 거리에 있는 아군이 급하게 힐이 필요한 상황이라면 패줌을 써야하는데 난전상황에서는 이게 참 어려워요. 따라서 패줌을 자유롭게 쓰고 싶다면 천천히 그리고 꾸준하게 연습을 하셔야 합니다.

 

여러분 패줌은 특별한 간지스킬이 아닙니다. 혹시라도 패줌이 멋있어 보인다거나 자기가 잘 하는 느낌을 받기 위해서 사용하시는 일은 없길 바랍니다. 꾸준한 연습만이 패줌을 잘 쓰기위한 지름길입니다.

 

 

 

Part 4. 수면총과 생체수류탄, 궁극기의 활용

 

1. 생체수류탄의 활용

 

① 1:1 상황에서의 활용

겐지, 트레이서, 원숭이, 디바 등이 아나를 물러온 경우 아나는 우선 평타로 견제를 해야합니다. 이후 아나의 체력이 떨어진 상황에서 자신과 상대방이 둘 다 맞을 수 있도록 발 밑보다 조금 앞에 생체수류탄을 던집니다. (점프하면서 던지지 마세요) 이렇게 본인은 100의 체력회복을 하고 상대에게는 60의 데미지 + 힐밴을 넣어주는게 1:1 상황에서의 기본 생체수류탄 활용입니다. 다만 물리고 있는 상황에서는 내 체력을 회복하는 게 더 중요합니다. 욕심 부리다가 상대 힐밴만 넣고 정작 자기피를 못 채워서 죽는 경우가 없도록 신경쓰셔야 합니다.

 

② 아군 탱커의 체력회복

아나의 단일힐량은 높은 편이지만 체력이 낮은 탱커를 소총으로만 채워주는 건 너무 오래걸립니다. 따라서 아군 탱커가 빈사상태인 경우에는 평타(75) + 생체수류탄(100) + 평타(75x2=150)를 활용하면 300이 넘는 체력을 한 번에 채울 수 있습니다.

 

③ 범위 체력회복

간혹 적 파라나 자리야 등의 포킹에 아군이 단체로 맞아서 전반적으로 체력상태가 안 좋은 경우가 생길 수 있습니다. 이 경우 주저하지 마시고 아군들 사이에 생체수류탄을 던지면 됩니다. 체력의 증가효과도 있지만 체력회복증폭효과도 있기 때문에 아군이 빠르게 전열을 정비하는 데 도움이 됩니다. 카테고리와는 맞지 않지만 아나가 재장전 중일 때 급하게 회복해야 할 아군이 있다면 던져주는 것도 필요합니다.

 

③ 초월 카운터 및 아군 중력자탄 지원

만약 한타에서 적 젠야타가 초월을 들고 있다면 생체수류탄을 아껴야 합니다. 특히 아군 자리야가 중력자탄을 쓸 준비를 하고 있다면 더 그렇습니다. 적군에 만약 젠야타가 없다고 해도 아군 자리야가 중력자탄을 쓸 각을 재고 있다면 생체수류탄을 아껴야 합니다. 중력자탄 속에 들어가는 생체수류탄의 위력은 어마어마합니다. 그리고 아나는 당연히 그 플레이를 해 줘야 하구요.

 

다만 던지실 때 몇 가지 유의할 점이 있습니다. 먼저 디바의 매트릭스를 확인해야 하고, 라인의 위치와 방벽상황도 고려해야 합니다. 앞에 아군이 있으면 생체수류탄을 몸으로 막을 수도 있습니다. 위와 같은 변수들을 고려한 뒤 생체수류탄을 확실하게 맞출 수 있는 각도와 타이밍에 던져주시면 됩니다. 만약 아군 자리야 궁이 잘 들어간 경우 아나가 죽더라도 라인방벽이나 디바 매트릭스 뒤로 파고 들어서 다수에게 힐밴을 적중시키는 판단이 필요할 때도 있습니다.

 

④ 힐밴을 통한 적군의 회복능력 상실★★★

직설적으로 말하면 생체수류탄은 적군에게 던지는 게 가장 좋습니다. 불가피하게 아군에게 사용해야 하는 상황이 온다면 이미 전세가 좋지 않은 상태일겁니다. 오버워치에서 힐밴의 위력은 말로 설명할 수 없습니다. 몇 백의 체력을 순식간에 채울 수 있는 오버워치에서 ‘힐을 차단한다’라는 건 엄청난 패널티입니다. 따라서 수류탄을 대충 던져야지라는 생각을 하지 말고 적 어느 영웅에게 힐밴을 먹이는 게 좋을지 빠르게 판단하고 해당 영웅을 향해 수류탄을 던져주세요. 혹은 적이 뭉쳐있다면 뭉쳐있는 곳에 던져주시면 됩니다.

 

힐밴의 위력에 대해 부연설명을 해드리자면 대치과정에서 한 쪽이 일방적으로 힐밴을 다수에게 적중당했다면 높은 확률로 그 팀은 한타를 대패합니다. 특히 쟁탈전 첫 전투에서 한 쪽이 수류탄에 다수 적중 했다면 그 팀은 첫 전투도 지고 라운드도 질 가능성이 높습니다. 적 힐러진들에게 힐밴이 들어가면 힐러진들은 압박감을 느끼고 숨거나 포지를 변경해야 하고 로드호그나 솔져처럼 자가회복이 있는 영웅들을 손발이 하나 없어진 느낌을 받게 됩니다. 그리고 힐밴은 아군에게 적 체력을 공유해 주는 효과가 있기 때문에 아군의 포커싱에도 도움이 됩니다.

 

그럼 어떻게 하면 생체수류탄을 잘 던질 수 있나요? 라고 물어본다면 많이 던져서 곡사의 각도에 익숙해져야한다는 말씀밖에 드릴 수 없습니다. 또한 라인, 디바, 자리야 등 다양한 영웅들이 당신의 수류탄을 방어하려고 할 것입니다. 그러나 이들은 충분히 극복가능한 변수입니다. 라인이 방벽을 들고 있다면 방벽 바로 뒤쪽으로 날아갈 수 있게 조준을 하거나 라인이 방벽을 내린 타이밍을 노리세요, 디바의 경우는 매트릭스를 유심히 보고 던지면 됩니다. 혹시 적이 벽에 붙어있는 상황이라면 방벽이나 매트릭스를 피해서 벽에 맞도록 던져주시면 됩니다. 네팔(성소)맵에서 다리대치를 하는 경우 아나가 잠깐 옆으로 빠져서 대각선으로 수류탄을 던져주시면 됩니다. 결국 생체수류탄은 많이 던져보면서 센스를 키워야합니다. 얼마나 상대에게 힐밴을 많이 해주느냐가 그렇기 때문에 아나의 실력을 나타내는 주요 지표 중 하나가 되는 것입니다.

 

그렇다면 생체 수류탄을 우선적으로 맞춰야 하는 대상은 누구일까요? 일단 수면총을 맞춘 적에게는 기본적으로 수면총 + 평타 + 생체수류탄 콤보를 넣습니다. 그 외에는 뭉쳐있는 적, 체력이 낮은 영웅들, 자가 회복기가 있는 영웅들이 주 타깃입니다. 체력이 낮은 영웅들은 팀 힐러의 케어를 받으면서 전선을 유지하는데 힐밴이 들어오는 순간 전선에서 이탈해야 합니다. 위에서 언급한 것처럼 솔져나 로드호그, 방벽빠진 자리야 등 자가회복기가 있는 영웅이 힐밴을 당하게 되면 스킬을 추가로 손해보는 게 됩니다. 따라서 아군이 급하게 치료가 필요한 상황이 아니라면 항상 적 주요 영웅이나 적군 다수에게 수류탄을 맞추는 연습을 해 보세요. 물론 난전 상황일때는 아군과 적군이 모두 맞도록 생체수류탄을 던지는 게 제일 좋겠죠?

 

아래는 부케로 할리우드 수비에서 플레이하는 영상입니다. 사실 여러모로 잘한 플레이영상은 아닌데 초보자분들에게 이렇게 못해도 점수를 올릴 수 있다는 희망을 드리기 위해.. 생체수류탄 활용을 위주로 봐 주시면 좋을 것 같습니다. 




 

2. 수면총의 활용

 

수면총의 기본 활용법

 

수면총의 기본 메커니즘은 메이의 우클릭과 같습니다. 발사키를 누르면 약간의 사전딜레이가 있고 이후 에임의 위치로 수면총이 발사됩니다. 여기서 수면총을 맞추는 방법은 크게 두 가지가 있는데 ① 끌어치기를 활용해서 적에게 에임을 맞추거나 ② 적이 이동할 경로를 예측해서 예측샷을 넣는 겁니다. 저 같은 경우는 적 영웅이 저를 물러오는 경우(근접거리)에서는 대부분 끌어치기를 사용해서 수면총을 쓰고, 그 외에 아나가 다소 안정적으로 수면총을 쓸 수 있는 상황이라면 예측샷을 하는 걸 선호합니다.

 

수면총은 재사용대기시간이 긴 편에 속하기 때문에 급하게 쓰기 보다는 무조건 맞춘다는 생각으로 확실하게 쏘는 게 좋습니다. 수면총은 많이 맞추는 것보다 필요할 때 정확하게 맞추는 게 더 중요합니다. 예를 들어서 상대방이 석양이 있다면 아나가 다른 영웅들에게 수면총을 먼저 쓰면 안 됩니다. 재사용 대기시간동안은 석양을 끊어줄 수 없기 때문입니다. 만약 상대방의 궁체크가 잘 안되어 있는 상황에서 수면총을 써야 할지 말아야할지 모르겠다면 ‘내가 적을 재웠을 때 확실히 잡을 수 있는 경우’에만 활용하고 상대가 자다가 그냥 일어날 것 같으면 아껴두시는 게 좋습니다. 물론 아나를 물러온 적에게는 무조건 맞춰주는 편이 좋습니다. 아래 1분정도 되는 짧은 동영상을 먼저 봐주세요.

 



(영상 뒤가 짤린건 포지셔닝 강의를 이어서 찍으려다가 따로 찍어서 그렇습니다 ㅎㅎ)
아래 움짤은 제 경쟁전 플레이영상으로 전방의 아나를 잡고 후방의 자리야를 끌어치기로 재우는 모습입니다. 실전에서는 저렇게 순간적으로 반응해서 재우지 못하면 아나가 죽게 되기 때문에 급할수록 확실하게 적을 재워줘야 합니다.




① 수면총의 역할 1 : 적의 궁극기 끊기

 

오버워치에는 아나가 수면총으로 꼭 카운터를 쳐줘야 하는 궁극기들이 있습니다. 대표적으로 석양, 죽음의 꽃, 포화, 돼재앙 등을 들 수 있습니다. 위 궁극기들은 이동이 느려지거나 고정되어 있으며 수면총을 적중시키기 상당히 쉽습니다. 따라서 아나는 항상 상대방의 궁극기 유무를 확인하고 예상 가능한 사용지점에 대해서 생각한 뒤 궁이 사용되는 즉시 끊어줘야 합니다.

 

예를 들어서 적이 석양이 있는 데 전방에 보이지 않았다면 분명히 뒤나 이층으로 궁각잡으러 돌아간 겁니다. 따라서 뒤에서 석양이 나올 걸 항상 염두해야 합니다. 적 리퍼가 궁이 찼는데 보이지 않는다면 낙궁자리를 항상 의식하고 있어야합니다. 이들 궁들을 보고 대처하는 것보다 미리 예측하고 있으면 훨씬 빠르게 대처할 수 있습니다.

 

추가적으로 쉽지는 않지만 용검이나 솔져 궁 같이 시간지속형 궁들도 재워주면 좋습니다. 솔져랑 겐지 둘 다 히트박스가 작아서 맞추는 게 쉽지는 않지만 솔져의 경우 궁을 키면 움직임이 평상시보다는 단순해지니 노려볼만 합니다. 그리고 상대 주요 탱커(라인, 원숭이, 디바, 자리야)등이 아나궁을 받고 들어오는 경우에도 무조건 재운다는 생각으로 수면총을 사용해야 합니다.

 

② 수면총의 역할 2 : 탱커의 견제

 

피격면적이 넓은 탱커들은 상대적으로 아나가 재우기가 쉽습니다. 따라서 궁극기를 끊기 위해 아껴야 할 상황이 아니라면 탱커위주로 재워주는 게 좋다고 생각합니다. 먼저 로드호그의 경우 수면총을 먼저 맞추게 되면 평타 + 수류탄을 통해 체력적 압박과 힐밴(숨돌리기 사용불가)을 넣어줌으로써 쉽게 견제할 수 있습니다. 라인의 경우는 궁을 쓰고 빗자루질을 시작하려고 할 때, 아군 영웅이 적 라인의 돌진에 끌려가고 있을 때, 돌진을 잘못써서 벽에다 박았을 때 수면총을 활용해서 재워주시면 됩니다.

 

원숭이의 경우는 대부분 아나가 물리는 상황에서 싸움이 일어나는 데 방벽심리전상황만 잘 극복하면 됩니다. 초 근접거리가 아니라면 먼저 수면총을 쓰지 마시고 방벽이 빠질 때까지 기다리세요. 방벽이 빠진 뒤에 ① 방벽 안으로 들어가서 재우고 빠져나오거나 ② 혹은 수류탄을 통해 자힐을 하면서 방벽 밖으로 탈출하는 방법 둘 중에 하나를 선택하시면 됩니다. 1:1로 원숭이를 잡는 건 쉽지 않으니 재워놓고 빠지는 게 최선입니다. 그리고 혹시 적 원숭이가 궁을 쓰고 아군 진영에 들어온다면 침착하게 아나한테 올 때까지 기다렸다가 재워주시면 됩니다. 자리야는 방벽 타이밍만 조심하시면 되고 마지막으로 디바의 경우는 매트릭스가 빠지고 디바가 데미지를 넣기 시작할 때 쏘시면 됩니다. 매트릭스가 사정거리가 길어서 그냥 쏘면 먹힐 가능성이 높습니다. 이처럼 아나 수면총은 상대방의 탱커를 견제하는 데 효과적입니다.

 

③ 수면총의 역할 3 : 아나의 생존 및 아나를 물러 온 적 제압

 

수면총은 상대가 아나를 물러왔을 때 아나가 사용할 수 있는 가장 효과적인 방어수단입니다. 그러나 아나가 아무리 자기를 물러온 상대를 재워놨다고 해도 대부분의 경우 아나 혼자서는 적 영웅을 잡을 수 없습니다. 따라서 상황에 따라 다른 방식으로 대처를 해야합니다.

 

먼저 상대가 맥크리처럼 체력 200에 도주기 및 생존기가 없는 영웅이라고 가정합시다. 아나가 단일 콤보로 줄 수 있는 데미지는 평타(80) + 생체수류탄(60) = 140입니다. 여기에 평타를 더해서 80의 데미지를 보태면 총 220의 데미지를 줄 수 있습니다. 그런데 유의할 점은 마지막 평타는 확정으로 들어가는 게 아니라는 점입니다. 평타 + 생체수류탄을 맞은 적은 일어나서 움직이게 되고 마지막 평타 한 방은 약간의 예측이 필요합니다. 그럼에도 대부분은 맞출 수 있으니 200짜리 영웅은 평타 + 수류탄 + 평타 콤보로 잡아주시면 됩니다. 혹시 해당 영웅이 풀피가 아니라면 평타 + 수류탄 + 근접공격도 괜찮습니다.

 

반면 체력은 적지만 생존기나 도주기가 있는 영웅은 아나 혼자 잡을 수 없습니다. 트레이서, 리퍼, 아나, 겐지 등이 해당 영우군에 속합니다. 이 경우는 아군이 약간의 화력지원만 해 주면 무난하게 잡을 수 있기 때문에 특정 아군을 지목해서 도와달라고 하거나 혹은 ‘여기 아나 위치에 누구 재워놨어요 한 분만 와서 같이 잡아주세요’라고 말 한 뒤에 같이 잡아야 합니다. 아나가 ‘하나, 둘, 셋’을 빠르게 센 뒤 동시에 치는 게 가장 좋습니다. 혹시 모르시는 분들을 위해 알려드리면 수면 중인 적군은 데미지가 들어와도 0.5초 동안은 스턴효과를 받아 생존기를 사용할 수 없습니다. 마지막으로 원숭이나 라인, 디바, 자리야 같이 탱커들이 아나를 물러온 경우 재워놨다고 해도 탱커들을 그 자리에서 잡기는 어렵습니다. 이 경우 어디에 적 누구 자요라고 팀원들에게 브리핑을 해 준뒤 안전한 위치로 이동해주세요.


 


3. 궁극기 나노강화제의 활용

 

① 궁극기 연계를 통한 시너지

저는 현 시점(이동속도너프)에서 나노강화제와 연계효율이 가장 좋은 궁극기는 전술조준경과 용검 두 개라고 생각합니다. 두 궁극기 모두 지속시간이 6초이며 나노강화제의 지속시간은 8초입니다. 따라서 나노강화제를 타이밍 맞춰서 먼저 넣어주거나 해당 영웅들이 요청할 때 주면 됩니다. 겐지의 경우 가려져서 나노강화제를 주기 힘들다면 2단 점프나 위로 질풍참을 써달라고 요청하시면 되고, 혹시 솔져 위치가 안보이면 솔져한테 어디있냐고 물어보시면 됩니다.

 

② 아군의 생존을 위한 활용

저는 아군을 살리기 위해 나노강화제를 종종 쓰곤 합니다. 대표적으로 아군 루시우가 소리방벽이 있는 상황에서 적 로드호그에게 끌려갔다면 좌클+근접공격을 맞기 전에 나노강화제를 줘서 루시우를 살립니다. 전면에 있는 아군 라인이 방벽이 곧 깨져서 죽을 것 같다면 미리 나노강화제를 주고 힐을 넣어서 살립니다. 즉, 아군을 살리는 게 나노강화제가 빠지는 것보다 이익이라고 판단되면 아끼지 말고 사용하셔도 괜찮습니다.

 

③ 상대 한명을 확실하게 끊어내기 위한 나노강화제

저 말고는 이런 플레이를 하는 사람을 많이 본적은 없는데 저는 종종 아군 로드호그가 적 자리야, 라인, 아나, 원숭이같이 원콤 내기 힘들 영웅들을 그랩하는 데 성공하면 그냥 나노강화제를 줘버립니다. 공격력이 50% 올라가기 때문에 누구를 끌어도 거의 원콤을 낼 수 있습니다. 이렇게 상대 한명을 확실하게 끊어먹으면 아군의 나노강화제가 없다고 해도 수적 우위를 바탕으로 상대 리스폰을 꼬거나 지속적인 이득을 취할 수 있기 때문에 낭비가 아니라고 생각합니다. 꼭 로드호그가 아니더라도 비슷한 상황에서는 아낌없이 궁을 줍니다. 뭔가 약을 파는 느낌이지만 리스폰을 효율적으로 꼬이게 만들 수 있는 레벨에서는 한 번 시도해 보셔도 좋을거라고 생각합니다. (댓글을 보니 실제로 대회에서도 나온 활용법이라고 합니다!!)

 

④ 연계할 궁극기가 없는 경우

이동속도가 삭제된 현 시점에서 나노강화제의 영향력은 이전에 비해 많이 떨어졌습니다. 따라서 연계할 궁극기가 없는 경우 그 상황에서 가장 좋을 것 같은 영웅에게 궁을 유동적으로 주면 됩니다. 저 같은 경우는 원숭이, 게이지가 높은 자리야, 적과 근접해 있는 라인 혹은 아군 원거리딜러(솔져, 맥크리) 등에게 상황에 따라 궁을 줍니다. 최근에는 디바의 전투시 이동속도가 빨라져서 디바에게 궁을 주는 것도 효율이 좋습니다. 이처럼 궁연계 없이 독자적으로 나노강화제를 써야하는 상황이라면 가장 기대값이 높은 아군을 찾아 주시면 됩니다.

 

⑤ 나노강화제 사용시 아나의 행동지침

나노강화제를 사용함에 있어서 가장 중요한 두 가지는 ①제 타이밍에 줄 것 ②잘못 주지 말 것입니다. 먼저 나노강화제는 시기적절하게 투입해야 합니다. 아나가 나노강화제를 제 타이밍에 주기 위해서는 줄 영웅의 위치를 파악하고 바로 줄 수 있는 위치에 가 있어야 합니다. 혹시 솔져등이 뒷 궁각을 잡는 경우에는 같이 따라가주는 것도 좋은 방법입니다. 겐지한테 궁을 줄 때는 겐지가 달라고 할 때 주는 방법이 제일 좋으며 겐지의 위치를 항시 체크해야 합니다. 두 번째로 가끔 앞에서 루시우가 뛰거나 아나가 잘못 누르거나 여러 가지 이유가 있는데 앞에서 갑자기 뛴 루시우 잘못도 있겠지만 기본적으로는 아나 실수라고 봐야합니다. 궁을 줄 때 누구 뽕 줄테니까 시야 가리지 말아달라고 부탁하세요.

 

 

생체수류탄, 수면총의 활용은 아나의 실력을 가르는 가장 큰 지표입니다. 여러분들이 기억하셔야 할 건 딱 한가지입니다. 스킬을 너무 아낄 필요는 없지만 ‘신중하게 꼭 필요한 곳에 사용해야 합니다’. 두 스킬의 쿨타임은 상당히 긴 편이며 스킬이 없는 상태의 아나는 상당히 약하다는 걸 잊지 마시기 바랍니다.

 


 

Part 5. 아나와 브리핑★★★

 

많은 분들이 아나를 할 때 오더나 브리핑을 하는지 궁금해 하시는데요. 일단 아나는 할 말이 많은 영웅이기 때문에 마이크는 필수적으로 해야합니다. 그리고 개인적으로는 아나가 넓은 시야를 가지고 있기 때문에 오더와 브리핑을 담당해 주는 게 좋다고 생각합니다. 그러나 아나를 배워가는 단계이신 분들은 오더를 하면서 동시에 영웅조작까지 하기 어려울 수도 있습니다. 그렇다면 아나가 필수적으로 해야 하는 브리핑 정도만 할 수 있도록 노력해 보는 건 어떨까요? 아나가 필수적으로 해야 하는 브리핑에 대해서 알려드리겠습니다.

 

① 수면총 적중시 (영웅명) 궁 끊었어요 / 누구 어디서 자고 있어요

아나가 적의 석양, 죽음의 꽃, 전술조준경, 용검등 적의 주요 궁극기를 수면총으로 끊었다면 팀원들에게 바로 알려주세요. 정말 중요한 정보입니다. 정보전달은 ‘뒤에 2층에 석양 재워놨어요’처럼 간결하게 해 주는게 좋습니다. 다음으로 수면총으로 누군가를 재워둔 상황에서 도움을 요청할 때는 <해당 아군 영웅, 자고 있는 적 영웅, 적 영웅이 자고있는 위치>를 정확히 알려줘야합니다. 예를 들어 ‘겐지님 저희 아나 바로 앞에 적 겐지 재워놨어요 같이 잡아주세요. 하나, 둘, 셋!’ 이라고 브리핑 해 주면 됩니다. 혹시 팀원들이 툭 건드려서 깨울까봐 걱정된다는 ‘라인님 앞에 리퍼 자는데 돌진 박아주세요’처럼 정확하게 도움을 요청해도 좋습니다.

 

② 재울게요(재워볼게요) 건들지 마세요!

만약 상대가 용검을 쓰고 들어오거나, 원숭이가 화나서 날뛰거나, 상대 탱커진들이 아나궁을 받고 달려오는 상태라면 팀적으로 아나가 재워주는 게 가장 이상적입니다. 그러나 간혹 힘들게 재워둔 적 영웅을 아군이 툭 쳐서 깨워버리는 경우가 있습니다. 사실 아나궁을 받은 상태 탱커들은 위협적이기 때문에 자연스럽게 포커싱이 집중되는 게 당연합니다. 그리고 무엇보다 팀원이 아나가 재워놓은 걸 깨우고 싶어서 깨운 게 아니라 전투에 집중하다보니 본의아니게 깨우게 된 것입니다.

 

이런 상황을 극복하기 전에 저는 게임시작전에 팀원들에게 미리 ‘적 아나궁 받은 탱커나 용검겐지 등이 들어오면 아나가 일단 재워볼테니까 수면총 빗나갔다고 하기 전에는 치지 말아주세요’ 라고 부탁을 합니다. 그리고 그 상황이 실제로 오면 상황이 오면 ‘미친원숭이 아나가 재울게요 건들지마세요!!’ 라고 한 번 더 브리핑합니다. 사실 깨우지 말아주세요 라는 표현이 더 정중하기는 하지만 이 브리핑은 팀원들에게 정확하게 전달되어야 하기 때문에 조금 더 직접적으로 말을 하곤 합니다. 혹시 수면총이 빗나갔다면 팀원들에게 바로 말해주세요.

 

③ 케어 브리핑

아나가 적 영웅을 잡기 위해, 혹은 물리고 있는 아군을 살리기 위해 한 명을 집중적으로 봐줄때가 있습니다. 이럴 때 해당 영웅한테 아나가 지속적으로 케어를 해 주고 있다는 걸 알려줘야합니다. 예를 들어 ‘자리야님 아나가 계속 힐 드릴테니까 빼지 말고 계속 앞으로 가세요. 호그 잡아주세요 호그. 왼쪽에 리퍼있어요. 저거 망령화 없어요 저것도 잡아주세요’ 같은 식으로 브리핑을 해 줄 수 있겠습니다. 즉, 아나가 당신을 백업하고 있기 때문에 당신은 죽지 않을 것이며 앞에 보이는 적을 다 때려잡을 수 있다는 걸 알려주는 겁니다. 이 경우 해당 영웅은 아나의 관리를 받으며 더 과감하고 공격적인 플레이를 통해 이득을 가져올 수 있습니다.

 

④ 아나가 힐로스가 발생한 상황에 대한 브리핑

전방의 탱커들은 아나에게 처한 상황을 알 수 없습니다. 따라서 아나가 별 말이 없으면 당연히 힐이 들어올 것을 가정하고 행동합니다. 따라서 아나가 힐을 줄 수 없는 상황이라면 팀원들에게 바로 브리핑이 되어야 합니다. 예를 몇 개 들어드리겠습니다.

 

1) 아나가 죽은 경우 : ‘아나 죽었어요!! 아나 없어요 힐 안들어갑니다. 조금 빼세요’

2) 아나가 자리를 옮기는 경우 : ‘아나 뒤에 2층 올라가고 있어요. 지금 힐 안됩니다’

3) 아나가 장전중인 경우 : ‘아나 장전중입니다. 잠시만요’

4) 아나가 물리고 있는 경우 : ‘아나 지금 물렸어요. 지금 힐 안되요(or 도와주세요)’

5) 아나가 전장에 가고 있는 경우 : ‘아나 거의 다 왔어요. 5초 후에 도착합니다’

 

이런식으로 아나가 없는 상황과 그 이유에 대해 팀원들에게 알려줘야 팀원도 그에 맞게 대처할 수 있습니다. 특히 아나의 장전시간은 1.5초로 상당히 긴 편입니다. 재장전 하는 동안에도 얼마든지 팀원들이 죽을 수 있기 때문에 장전콜도 잊지않고 해주셔야 합니다.

 

⑤ 아나의 위치에 대한 브리핑과 아군의 위치조정 브리핑

저는 화물맵이나 거점수비를 할 때 항상 팀원에게 아나 위치를 먼저 확인하라고 합니다. ‘자자 아나 위치 한 번만 봐주세요. 아나 여기서 힐 드릴꺼에요. 동상기준 왼쪽으로 가시면 힐 안들어갑니다(도라도 A수비상황에서 아나가 뒤 2층에 자리잡은 경우)’ 제가 힘들게 이 말을 하는 이유는 아나 위치를 알아야 팀원들이 힐을 잘 받을 수 있기 때문입니다. 그리고 아나가 위치를 변경하게 되면 변경된 위치를 다시 공유해야합니다. ‘아나 지금 1층으로 내려왔어요. 거점 오른쪽 뒤에서 힐 드릴게요’ 처럼 말입니다. 또한 팀원들이 힐을 받을 수 없는 위치로 가는 경우 팀원들의 위치도 조정해줘야 합니다. 수비팀일 경우 아나가 이미 좋은 자리를 잡고 있다면 힐을 주기 위해 그 자리를 포기하는 건 비효율적이기 때문입니다. 이 경우 ‘라인님 더 앞으로 가시면 힐 안들어가요. 뒤로 더 빠지세요’ ‘겐지님 거기 힐 못드려요. 왼쪽으로 빠져보세요’처럼 팀원들의 위치 또한 아나가 힐을 받을 수 있도록 조정해줘야합니다.

 

⑥ 힐밴 브리핑

아나의 생체수류탄으로 적의 주요 영웅이 힐밴이 된 경우 팀원들에게 브리핑해주셔야 합니다. 그 중에서도 포커싱이 필요한 대상이 있는 경우 ‘지금 젠야타 힐밴이에요 먼저 잡아주세요’처럼 직접적으로 팀원들에게 요청할 수 있습니다. 또한 아나가 생체수류탄을 적 다수에게 적중시킨 상태라면 ‘지금 적군 다 힐밴이에요 가서 덮쳐요 (or 이니시걸어요)’처럼 필요에 따라 이니시에이팅콜도 해주시면 좋습니다.

 

⑦ 딸피 브리핑

아나는 저격을 통해 적들에게 데미지를 주는 게 가능하기 때문에 상대의 체력상황을 알 수 있습니다. 따라서 일반 딜러들이 해 주는 것처럼 딸피 브리핑 및 데미지 관련 브리핑을 해주셔야 합니다. 특히 체력이 200인 영웅에게 저격을 두 번 성공했거나 적 주요영웅을 딸피를 만드는 데 성공했다면 적극적으로 딸피브리핑을 해서 팀원들의 포커싱을 유도해줘야합니다.

 

⑧ 나노강화제 사전알림 서비스

보통 딜러들이 아나에게 직접 나노강화제를 달라고 하는 경우가 아니면 많은 경우 나노강화제를 예고없이 받은 아군 영웅들은 당황하기 마련입니다. 따라서 궁극기가 80~90%정도 찬 상태라면 미리 아군의 궁체크를 하고 이어지는 전투에서 누구한테 궁을 줄지 사전에 알려주는 게 좋습니다. 이후 ‘겐지님 지금 뽕드릴게요’ 같이 궁을 주면서 한 마디 더 해주시면 됩니다. 만약 급박한 상황에서 궁을 써야한다면 ‘자리야님 뽕드릴게요!!!’처럼 궁을 주는 시점에서라도 팀원이 알 수 있도록 알려줘야 합니다.

 

정리해보면 아나는 브리핑 할 게 정말 많은 영웅입니다. 전체적인 오더는 못한다고 해도 적어도 자기가 하고 있는 일에 대해서는 팀원들에게 확실하게 브리핑을 해 줘야 좋은 아나 플레이어가 될 수 있습니다.

 

 

 

Part 6. 아나의 근접전투 및 영웅별 상대법

 

1. 아나의 근접전투 방식

 

아나는 근거리 혹은 애매하게 가까운 거리에서 노줌 혹은 저격중에 어떤 걸 사용해야 할까요? 여기에 대해 개인적인 생각을 말씀드리자면 아나가 적과 싸우다가 죽을 수 있는 상황이면 노줌, 아나가 비교적 안전한 상황이라면 저격을 선호합니다. 예를 들어서 겐지가 아나를 물러 왔을 때는 좌우무빙도 해줘야 하고 필요에 따라 점프도 해야 하기 때문에 무빙을 하면서 노줌으로 겐지를 맞추려고 노력해야 합니다. 반대로 겐지가 체력적 압박을 느끼고 도망가는 상황이라면 초근거리가 아닌 이상 줌을 켜서 확실하게 맞춰주는 게 좋습니다.

 

다른 예를 들어볼까요? 아나가 트레이서와 1:1을 하고 있다고 가정하겠습니다. 트레이서가 아나에게 근접한 상태라면 좌우앉기무빙을 하면서 평타를 넣으려고 노력하거나 아니면 수면총 or 생체수류탄 각을 볼 것입니다. 만약 트레이서가 뒷점멸로 빠지거나 옆점멸을 쓰면서 거리를 유지한다면 노줌대신 패줌이나 빠른 끌어치기를 통해 확실하게 딜을 넣으려고 시도할 것입니다. 트레이서가 가까이서 아나를 때리게되면 죽을 수 있으니 무빙을 해야 하지만, 일정 거리를 유지하면서 견제를 하는 상황이라면 아나가 죽을만큼 데미지가 크게 들어오지 않기 때문입니다. 이해가 잘 되셨나요?

 

2. 영웅별 간단한 1:1 상대법

 

① 원숭이 : 원숭이는 못 재우면 아나가 죽습니다. 따라서 수면총을 성급하게 쓰지 마시고 방벽을 칠 때까지 기다렸다가 방벽 안에 들어가서 재우거나 아니면 생체수류탄으로 자가회복을 하면서 방벽 밖으로 도망가세요. 궁 킨 원숭이는 기다렸다가 아나 근처로 왔을 때 재워주시면 됩니다. 그리고 원숭이가 궁이 있는 경우 상대방 입장에서는 대부분 알 수 있습니다. (진입방식 자체가 용감해질때가 많습니다). 이 경우 생체수류탄을 맞춰주면 궁을 켜도 최대체력이 500이 된다는 점 기억해주세요.

 

② 겐지 : 궁 없는 겐지는 처음 접근했을 때 무빙을 하면서 평타를 넣어주면서 시작됩니다. 아나의 피가 100정도 빠지면 겐지가 접근하는 시점을 기다렸다가 생체수류탄을 자신과 겐지 에게 둘 다 맞춰주세요. 겐지가 힐밴을 맞으면 대부분은 도망갑니다. 겐지가 도망간다면 줌을 키고 침착하게 맞춰주시면 됩니다. 겐지를 맞출 때는 급하게 쏘지 마시고 침착하게 2단 점프를 기다린 뒤에 겐지의 움직임이 제한될 때 저격해주시면 됩니다. 그런데 수류탄 겐지한테 같이 맞추고 싶다고 본인 체력이 70~80까지 떨어졌는데도 수류탄을 아끼시면 안됩니다. 겐지를 너무 얕보지 마세요. 아나가 항상 1:1을 이길 수 있는게 아닙니다.

 

반대로 용검을 든 겐지는 정말 무섭습니다. 용검겐지를 재울 수 있는 타이밍은 크게 네 번정도 있다고 생각하시면 됩니다. ① 용검을 발동하는 모션에서 재울 수 있습니다 ② 용검을 쓰고 질풍참을 쓰려는 루트가 아나와 직선이라면 예상되는 질풍참의 궤적에 미리 쏘는 방법으로 재울 수 있습니다 ③ 다음으로는 질풍참을 쓴 뒤에 잠깐 멈추는 타이밍에 재워볼 수 있습니다 ④ 마지막으로 칼질 한 번을 생체수류탄으로 버틴 뒤 겐지의 궤적을 끌어치기해서 재워볼 수 있습니다. 그런데 사실 말이 네 번의 기회지 실제로 겐지를 재울 수 있는 확률은 제 경험상 30% 정도입니다. 잘하는 겐지들은 진입할 때 수면총 각을 거의 주지 않습니다. 용검겐지를 재워야 할 때는 최대한 침착하게 수면총을 쓰셔야 합니다. 어짜피 수면총은 한 발 밖에 없고, 못 재우면 아나가 죽는 겁니다.

 

③ 트레이서 : 트레이서는 아나의 수면총과 생체수류탄을 의식하면서 거리조절 및 견제를 할 것입니다. 이 상황에서 아나가 성급하게 수류탄과 수면총을 빼 주면 높은 확률로 1:1을 지게 됩니다. 트레이서와의 싸움에서 아나가 이기는 가장 좋은 시나리오는 평타 한 대를 먼저 맞추는 겁니다. 트레이서는 아나의 평타 두 방에 죽기 때문에 평타를 한 대만 맞아도 시간역행을 빼야합니다.

 

그러나 트레이서를 맞추기 전에 내가 먼저 데미지를 입었다면 생체수류탄 타이밍을 잘 해야합니다. 생체수류탄을 본인과 트레이서 둘에게 모두 맞출 수 있다면 좋겠지만 거리조절을 잘 하는 트레는 수류탄 각을 잘 주지 않습니다. 따라서 이 경우는 상대방의 점멸 여부를 바탕으로 그냥 자가회복에 쓸지 아니면 같이 맞춰보려고 시도를 해야할지 결정해야합니다. 트레이서에게 수면총을 맞출 수 있는 타이밍은 크게 두 번이 있는데 ① 트레이서가 점멸로 아나에게 접근했을 때 ② 좌우로 점멸을 쓴 경우입니다. 만약 트레이서가 점멸에 여유가 있다면 수면총이나 생체수류탄을 빼기 위해 ①을 페이크로 쓸 수 있으니 유의하시고 ②의 경우도 확실하지 않으면 트레이서가 접근할 때 까지 아끼는 게 좋습니다. 일단 쓰지 않기만 해도 트레이서 입장에서 의식할 수 밖에 없으니까요. 그래도 전반적으로 1:1 상황에서는 아나가 이길 확률이 더 높다고 생각합니다.

 

④ 로드호그 : 아군이 갈고리에 걸리는 걸 보고 쏘면 늦는 경우가 많습니다. 호그가 아군쪽으로 갈고리를 쓰는 게 보인다면 일단 재워주세요. 호그는 히트박스가 크기 때문에서 아나입장에서는 무조건 재워야 하는 적입니다. 특히 호그 궁은 무조건 끊어줘야 하고 재운 뒤에는 평타 + 생체수류탄 콤보는 같이 넣어주세요. 아나가 풀피상황이라면 호그한테 끌려가도 잘 죽지 않습니다. 이론상으로 호그가 그랩 + 좌클 + 근접공격으로 아나를 원콤을 낼 수 있기는 한데 호그의 숙련도가 높지 않으면 대부분 살아올 수 있습니다. 이 경우 당황하지 마시고 생체수류탄을 던져서 본인의 피를 회복하고 힐밴을 넣은 뒤 수면총으로 재워놓고 팀이 있는 곳으로 도망치면 됩니다.

 

⑤ 맥크리 : 혹시 잠깐 방심하거나 저격하다가 못봐서 적 맥크리가 섬광거리까지 접근했다면 일단 좌우로 뛰세요. 제자리에서 섬광을 맞으면 꼼짝없이 죽어야하지만 점프하면서 섬광을 맞으면 살 가능성은 생깁니다. 만약 살았다면 본인과 맥크리 둘 다 맞도록 생체수류탄을 던져주시고 침착하게 좌우앉기무빙(헤드라인을 잡지 못하게 하기 위함)을 하면서 맞서 싸워주시면 됩니다. 승부는 서로의 샷발에 따라 갈립니다.

 

⑥ 리퍼 : 리퍼는 아나가 리퍼의 데미지를 과소평가 하지만 않으면 충분히 이길 수 있습니다. 아나의 히트박스가 작기 때문에 리퍼 데미지가 생각보다 약하게 들어옵니다. 또한 리퍼는 망령화가 있기 때문에 내가 리퍼를 잡을거야라는 생각보다는 망령화를 빼도록 만들어서 리퍼가 물러나게 하는 게 좋습니다. 보통 아나가 리퍼를 만나는 시점은 리퍼가 몰래 접근한 상황에서의 초근거리일 가능성이 높습니다. 놀라지 마시고 침착하게 수면총을 끌어치기하면 히트박스가 넓은 리퍼를 높을 확률로 재울 수 있습니다.

 

⑦디바 : 요새 떠오르는 1티어 탱커죠? 아나 혼자 디바를 잡기는 어렵습니다. 대신 재우는 건 크게 어렵지 않습니다. 디바가 온다고 급하게 수면총이나 생체수류탄을 쓰지 말고 침착하게 평타를 넣어주세요. 이후 디바가 매트릭스를 끄고 데미지딜링을 시도하는 순간에 침착하게 수면총을 쏘고 도망가면 됩니다.

 

⑧ 파라 : 파라를 상대할 때 줌을 키고 계속 저격하시면 안 됩니다. 일단 줌을 키면 이동속도가 느려지고 상대 파라 미사일의 궤적도 보이지 않습니다. 당연히 지상에서 아나를 노리는 적들도 볼 수 없습니다. 따라서 내가 파라를 저격하고 있고 파라가 나를 보는 상황이라면 빠른 끌어치기를 시도한 뒤 줌을 끄고 파라 미사일의 궤적을 보고 피해주세요. 이걸 계속 반복하면 됩니다. 아나가 안전한 상황이라면 3~4발 정도는 줌을 키고 저격해 주셔도 괜찮습니다.

 

 

 

Part 7. 아나는 언제 저격을 하나요?

 

시작에 앞서 아나가 저격을 하는 주 이유는 위도우처럼 상대를 제압하기 위함이 아닙니다. 상대에게 체력적 압박을 넣어서 행동반경을 줄이기 위함이며 킬은 일종의 보너스 같은 것입니다. 그렇다면 아나는 언제 저격을 해야 할까요?

 

① 공격팀 아나의 주 임무 : 수비팀 아나의 포지셔닝을 방해하라

수비측 아나는 2층 혹은 주요 저격지점을 먼저 선점하고 있습니다. 공격측 아나는 상대 수비아나를 저격함으로써 선점하고 있던 좋은 위치를 포기하게 만들어야 합니다. 할리우드를 예로 들면 A거점에서 수비팀 아나가 엘리베이터위에 자리 잡고 있다고 합시다. 이 경우 적 아나를 끌어내릴 수 있는 건 아군 아나 밖에 없습니다. 공격팀 아나가 저격을 한발 내지 두발만 성공시켜도 수비팀 아나는 그 위치를 포기해야합니다. 이는 할리우드뿐만이 아닌 대부분의 거점 & 화물맵에 적용되는 이야기입니다. 아군 브루저들이 견제하러 가기 힘들 위치에 수비팀 아나가 자리잡고 프리힐을 넣고 있으면 그걸 견제해 줄 수 있는 사람은 공격팀 아나밖에 없습니다. 적 아나를 저격으로 잡으라는 말이 아닙니다. 한~두대 정도 맞춰서 그 포지션을 포기하게만 만들어도 충분히 남는 장사입니다. 적 아나 뿐만 아니라 2층을 잡고 있는 솔져, 맥크리도 같은 맥락으로 적용할 수 있습니다.

 

② 적군 겐지와 파르시 견제의 선봉

제 경우 저격을 통해 가장 많이 견제하는 대상은 겐지와 파라입니다. 겐지 같은 경우는 2단 점프를 다 쓰고 공중에 떠 있는 듯 한 타이밍일 때 가장 맞추기 쉽고 파라는 생각보다 맞추기 쉽습니다. 겐지와 파라를 주로 저격하는 이유는 이들이 자유롭게 행동할 때 상당히 까다롭기 때문입니다. 아나가 겐지와 파르시를 집중적으로 견제해주면 상대 겐지와 파르시는 ‘적 아나가 잘 쏜다’라는 생각을 하게 될 것이며 이는 추후 상대의 행동반경을 제한하는 역할을 합니다. 따라서 겐지나 파르시가 자신감을 갖고 날뛰게 하지 않으려면 이 둘을 잘 견제해주세요.



③ 체력이 적은 딜러나 힐러 위주로 저격해라

아나의 저격 데미지는 80입니다. 체력이 적은 적은 한~두대만 맞아도 상당한 체력적 압박을 느끼고 전장에서 이탈하게 됩니다. 반면 탱커들은 몇 대 쳐 봐야 크게 아파하지 않습니다. 오히려 내가 궁을 채운 만큼 상대 힐러 들도 궁을 채우게 될 겁니다. 따라서 아나가 궁이 이미 찬 경우에는 탱커들을 배제하고 딜러나 힐러들을 노려주는 게 좋고, 궁을 빨리 채워야 할 상황이라면 탱커들을 궁셔틀 용도로 활용해주시면 됩니다. 저격을 조금 더 전략적으로 운용하려면 상대 자리야를 두발정도 맞춰서 자기방벽을 빼게 만드는 식으로 운용할 수도 있고 점프각을 보고 있는 원숭이에게 저격을 해서 진입 타이밍을 흐트러뜨리는 플레이도 가능합니다. 그러나 일단은 체력이 적은 적 딜러 & 힐러를 노리는 게 더 좋습니다.

 

④ 저격시 줌을 키고 있지 말고 빠른 끌어치기를 활용해라

아나의 저격은 꽤 강력하지만 저격을 위해 줌을 키고 있다면 아군을 케어하지 못합니다. 따라서 아나는 적을 저격할 때 빠른 끌어치기를 하거나 혹은 저격모드로 쏜다고 해도 2~3발 정도 저격을 하고 다시 아군의 상태를 점검해야합니다. 또한 아나로 상대 라인 방벽 깨지 마세요. 혹시 방벽 깨느라 탄창을 소모하는 과정에서 아군에게 급하게 힐을 줘야 하는 상황이 발생하면 정작 필요한 순간에 총알을 활용할 수 없게 됩니다. 아나의 장전은 리스크가 꽤 큽니다.

 

아나의 본분은 힐러입니다. 저격은 어디까지나 ①아군의 방어라인이 탄탄하게 유지되어 있어서 아나가 저격할 여유가 있거나, ②파르시나 겐지처럼 필수적으로 견제나 필요한 영웅이 상대방에 있거나 ③상대가 주요 고지를 포기하게 만들기 위함이거나 ④궁극기를 빠르게 채워야 할 때 하는거고 위 경우들이 아니라면 힐에 집중해주세요. 특히 상대 위도우랑 저격싸움 하시면 안됩니다. 아나는 세 번 맞춰야 하고 상대 위도우는 한 발만 잘 쏘면 됩니다. 아나의 죽음은 대가가 크다는 사실을 잊지마세요.

 

 

 

Part 8. Q & A

 

Q. 아나는 탱커 말고는 케어를 안 해주나요?

 

A. 그렇지 않습니다. 저는 탱커를 제외하고는 겐지, 파라, 리퍼 등을 주로 케어해주는 편입니다. 해당 영웅들은 존재자체만으로 어그로가 엄청 끌리기 때문에 아나가 바로 힐을 주지 않으면 순식간에 죽을 수도 있습니다. 특히 적진에서 어그로 빼고 질풍참 or 망령화로 빠지는 겐지와 리퍼는 힐을 바로바로 줘야합니다.

 

 

Q. 장전은 언제 해야 하나요?

 

A. 아나의 장전시간은 1.5초로 상당히 깁니다. 총알이 많음에도 습관적으로 장전하시는 습관을 버리시고 확실하게 아무 일이 일어나지 않을 때 장전을 하세요. 혹시 총알 수가 다소 적어서 (4~5개) 불안하더라도 일단 전투가 열리거나 힐이 필요한 상황이 올 것 같으면 장전을 하지 않고 남은 총알로 해결해야합니다. 추가적으로 장전할 때 엄폐물 뒤에 살짝 숨는 건 기본적인 센스입니다.

 

 

Q. 아나의 궁극기 활용에는 우선순위가 있나요?

 

A. 일단 용검이나 전술조준경이랑 맞추는 게 제일 좋습니다. 그게 아니라면 잘 하는 딜러나 게이지 높은 자리야, 적군와 근접해 있는 디바 or 라인 등 상황에 맞게 주시면 됩니다. 특별히 웅선순위가 있는 건 아닙니다.

 

 

Q. 수면총 미세팁이 있을까요? 겐지나 트레는 맞추기가 너무 힘듭니다.

 

A. 수면총을 맞춰야 겐트를 이기는 건 아닙니다. 오히려 수면총을 들고 있는 편이 좋을 때도 많아요. 수면총을 심리전을 위한 카드로 아껴주고 생체수류탄을 본인과 적 겐트에게 맞추는 것에 더 집중해보세요. 겐지는 아나의 수면총을 의식해서 튕겨내기를 빼게 될 거고, 트레이서 역시 쉽게 근접거리로 들어오지는 못할 겁니다.

 

그리고 심리전도 꽤나 중요합니다. 자기가 질 것 같다고 생각해서 계속 뒷무빙 치면서 피하는 것보다 1:1은 무조건 내가 이긴다는 마음으로 자신 있는 무빙을 보여주세요. 아나가 오히려 겐트를 적극적으로 잡으려고 하고 소총이라도 맞춰서 도망가게 만든다면 상대 겐트 입장에서는 이후 아나를 혼자 잡으러 오기 힘들 겁니다. 아나가 겐트를 무서워한다는 인상을 심어주면 게임 내내 괴롭힐 당할 가능성이 높습니다. 기세싸움이에요.

 

 

Q. 미리 자리를 잡기 못한 경우 대치상황에서 아군 본대에서 힐을 하시나요? 아니면 최대한 가가운 저격장소(2층 등)으로 이동을 먼저 하시나요?

 

A. 전투가 바로 벌어질 것 같으면 본대 뒤에서 힐을 줄 자리를 찾습니다. 그러나 전투가 바로 벌어질 상황이 아니라면 2층으로 올라가는 게 좋다고 생각합니다. 일단 아나가 자리를 잘 잡아야 한타 때 이길 가능성이 높기 때문입니다. 물론 팀원들에게 아나가 2층 자리 잡는 중이니 싸우지 말라고 센스 있게 브리핑해주세요.

 

 

Q. 궁극기를 멈추는 것을 제외하고 수면총의 우선순위가 어떻게 되나요?

 

A. 누구든 재워놓고 잡을 수 있으면 상관없습니다. 저는 개인적으로는 탱커를 재우는 걸 선호하는 편입니다. 히트박스가 넓어서 재우기 쉬운 것도 있고, 또 탱커를 재운 뒤에 잡아내거나 데미지 + 힐밴을 통한 체력적 압박을 넣으면 상대의 진형 전체가 흐트러질 때도 많습니다.

 

 

Q. 쟁탈전에서 아나 포지션과 궁각을 알고 싶어요.

 

A. 궁극기 이동속도 삭제 이전에는 아나가 주도적으로 궁각을 잡는 게 가능했습니다. 그런데 지금은 아나가 단독적으로 궁각을 잡기보다는 궁체크를 하고 팀원과 조율하면서 사용하는 게 제일 이상적이라고 생각합니다. 쟁탈전에서의 포지션은 본문을 참고해주세요.

 

 

Q. 소총/저격힐 선택기준이 궁금합니다. 어중간한 거리에서는 뭘 사용해야 하나요?

 

A. 저는 근거리에서 적과 싸우고 있는 상황이 아니면 힐/딜 모두 저격(빠른 끌어치기)을 선호합니다. 탱커의 경우 원거리에서 지속적으로 힐이 필요하면 줌을 킨 상태에서 계속 주고 노줌으로도 충분히 커버가 가능한 중간거리라면 노줌으로 줍니다. 탱커들은 히트박스가 넓어서 근~중거리에서는 노줌으로 힐을 줘도 괜찮습니다. 아군이 딸피일 경우는 거리에 상관없이 빠른 끌어치기 혹은 패줌 통해 히트스캔으로 힐을 넣어주는 편입니다. 원칙을 세우자면 노줌으로 명중을 확신할 수 없다면 저격을 쓰는 게 더 좋다고 말씀드리고 싶네요.

 


Q. 수면총을 잘 맞추는 노하우가 있나요?

 

A. 본문에서 언급된 내용에서 약간 추가로 설명 드리겠습니다. 먼저 겐지, 트레이서, 갑자기 나타난 리퍼, 원숭이, 디바 등등 아나를 물러오는 적은 끌어치기로 재워야합니다. 수면총 끌어치기는 본인이 감도에 익숙한 상태에서 꾸준히 연습을 하셔야만 잘 할 수 있습니다. 스타일리시하게 재우려고 하지 마시고 침착하고 정확하게 쏘는 연습을 하세요. 두 번째로 적의 주요 궁극기를 재울 때는 대충 상대 영웅이 어디쯤에서 궁을 쓸 것이라는 걸 생각을 하고 계셔야 합니다. 보고 반응하는 것보다는 예측하고 있다가 반응하는 게 훨씬 빠릅니다. 도움이 되셨나요?

 

 

Q. 키 설정 별도로 바꾸신 게 있나요?

 

A. 근접공격을 F와 V모두로 사용하는 것을 제외하고는 바꾼 세팅은 없습니다. 조준선은 초록색 점을 사용하는데 조준점과 키세팅은 자기가 편한 게 최고라고 생각해요.

 

 

Q. 앉기나 점프가 특히 유용할 때가 언제인가요?

 

A. 먼저 아나가 점프를 하면서 힐을 주는 이유는 상대의 에임을 흐트러뜨리기 위함입니다. 앉기 같은 경우 때때로 저는 무빙의 흔들림을 최소한으로 하기 위해 사용하는 경우가 있고 (가만히 서 있으면 언제 죽을지 모르니까 서서 저격하는 습관은 좋지 않습니다) 전투시에는 좌우로 앉는 무빙을 통해 상대 에임을 상하좌우로 흐트러뜨리기 위해서 앉기를 합니다. 좌우앉기 회피무빙은 특히 맥크리처럼 헤드를 노리고 에임을 잡는 영웅들을 상대로 할 때 매우 좋습니다. 아나는 가뜩이나 히트박스도 작아서 무빙만 잘 하면 생체수류탄과 함께 오래 생존할 수 있습니다. 반면 솔져같은 영웅이랑 싸울 때 좌우 앉는 무빙을 반복적으로 활용하면 히오스탄에 직격당할 수 있으니 좌우무빙의 빈도를 좀 더 높이는 게 좋습니다. 점수가 낮아질수록 앉기보다는 점프를 자주 활용하는 경향을 보이는데 점프는 동선이 예측가능하기 때문에 오히려 맞을 확률이 더 높습니다.

 

 

Q. 아나 POTG 뜨는 방식에 대해 어떻게 생각하시나요

 

A. 저는 영웅별로 개성있는 슈퍼플레이가 POTG로 나올 수 있었으면 좋겠습니다. 아나의 경우 적 다수에게 생체수류탄을 적중시키고 한타를 이기거나 혹은 적의 궁극기를 칼같이 재우고 한타를 승리로 이끄는 장면이 POTG입니다. 언젠가 블리자드가 시스템을 잘 만들어서 궁극적으로는 각 영웅의 개성을 살린 다양한 POTG가 나왔으면 좋겠습니다. 예를들어 시메트라의 순간이동기를 아군 6명이 한 번에 타고 가는 장면처럼 말이에요. 아마 시간이 조금 필요하지 않을까요?

 

 

Q. 게임 감도와 줌 감도는 어떻게 쓰시나요?

 

A. 게임 감도는 DPI 2000에 인게임 감도 1.5를 사용합니다. 꽤 느린 편이라고 알고 있습니다. 류제홍 마우스캠을 보면 손목이 휙휙 돌아가는 모습이 보이는 데 저 또한 비슷합니다. 줌 감도가 몇일 때 일반감도와 같은지 물어보신 분도 계시는 데 잘 모르겠습니다. 저는 기본인 30을 사용하는 데 괜찮은 것 같습니다. 끌어치기를 좀 멀리 해야할 때 마우스를 휙 돌리기는 해야하는데 하다보면 익숙해집니다 ㅎㅎ

 

 

Q. 팀보 없이 솔큐로 아나를 하면 앞에서 팀원들이 알짱거려서 힐도 안되고 딜도 안됩니다. 솔큐 때는 그럼 포지션을 앞으로 변경해야 하나요?

 

A. 아군 탱커 뒤에서 탱커위주로 케어하면서 여유가 생기면 저격한다는 느낌으로 플레이 하시면 좋을 것 같습니다. 위 상황에서는 아나가 후방포지션을 잡고 있어도 팀원들이 힐을 받을 수 없는 위치로 갈 확률이 높아 보이네요. 또한 본인이 직접 딜을 하기 보다는 적들에게 수류탄을 잘 던져서 아군이 호응하도록 만드는 편이 좋을 것 같습니다. 그리고 만약 팀원들이 다 따로놀아서 정말 답이 없다면 젠야타를 뽑아서 방울 주고 직접 딜을 하거나 메르시를 뽑아서 잘 하는 딜러를 따라다니는 것도 한 방법입니다. 아무래도 아나는 영웅의 특성상 팀원들끼리 활발하게 소통 할 때 더 효율이 좋습니다.

 

 

Q. 저는 생체수류탄과 궁극기를 올바른 타이밍에 정확히 넣기 위해서는 아나의 포지션이 본대와 너무 떨어진 위치가 아니어야 한다고 생각해서 그렇게 플레이해왔는데요. 멀리 떨어진 위치에 포지셔닝할 때 생체수류탄과 궁극기는 어떻게 활용하시나요?

 

A. 일단 저는 수비팀과 공격팀의 포지션을 나눠서 생각해야 한다고 생각합니다. 화물&거점맵 기준으로 수비팀인 경우에는 본대에서 떨어져서 최대한 안전한 위치에서 안정적으로 힐을 주는 걸 선호합니다. 어중간한 거리에 있으면 상대 브루저들이 쉽게 물러올 수 있기 때문입니다. 간혹 거리가 멀어서 나노강화제 사정거리를 벗어나는 경우가 있습니다. 이 경우 제가 잡고 있는 포지를 버리기 싫을때는 팀원에게 뒤로 와달라고 요청하고, 크게 좋은 자리가 아닌 경우궁을 써야할 타이밍에 맞춰 미리 1층으로 내려오거나 자리를 앞으로 옮겨 잡습니다. 생체수류탄의 경우는 수비팀에 있을 때 아나가 주로 잡는 2층 핫스팟은 몇 군데 정해져 있습니다. 사설이나 실제 게임을 통해 수류탄을 많이 던져보면 거리조절은 크게 어렵지 않다고 생각합니다. 다만 세밀한 거리조절은 어려운 만큼 적군 누구를 맞추겠다는 마음보다는 뭉쳐있는 곳에 던진다는 생각으로 수류탄을 활용합니다.

 

반대로 공격팀인 경우에는 오히려 아나가 본대에서 멀어질수록 짤릴 가능성이 높습니다. 따라서 본대와 멀리 떨어지지 않는 위치에서 지원을 하는 게 더 좋을때가 많습니다. 또한 보통 1층에 있기 때문에 생체수류탄 대박을 노리기보다는 확실하게 1~2명씩 힐밴을 시켜서 끊어내는 플레이를 할 때가 많습니다. 실제로 공격팀 입장에서는 한 번에 적을 다 잡는 것보다 한두 명씩 계속 끊어내는 편이 더 이익입니다. 쟁탈전 역시 본문에서 설명했지만 질문자분이 말씀해주신 것과 비슷하게 포지셔닝을 하면 됩니다. 특히나 쟁탈전은 생체수류탄을 잘 던져주는 게 정말 중요하기 때문에 아군 탱커를 방패막 세워서 플레이 하는 방식이 대부분의 상황에서 가장 효과적이라고 생각합니다.

 


 

부록. 각 맵별 수비시 포지셔닝 추천 위치


아래 사진에 있는 위치들은 제가 주로 사용하는 수비팀 아나 포지셔닝 위치입니다. 아래 사항들을 중점적으로 생각하시고 아나의 포지셔닝 위치를 확인해주세요. 다른 위치들도 있지만 제가 주로 사용하는 위치들만 추렸습니다.


① 아군에 대한 시야가 확보되는 위치 (약간의 이동으로 아군 전원에게 힐을 줄 수 있는 위치)

② 주변에 엄폐물이 있는 위치

③ 적군의 우회루트를 볼 수 있는 위치

④ 주변에 힐팩이 있거나 도주시 수면총을 활용하기 좋은 위치

⑤ 공격팀 아나에게 쉽게 노출이 되지 않거나 숨을 수 있는 곳

⑥ 적의 이동기 한 번으로 접근이 불가능한 위치



1. 66번 국도


원본 크기로 보시려면 그림을 클릭하세요.


원본 크기로 보시려면 그림을 클릭하세요.

 

원본 크기로 보시려면 그림을 클릭하세요.



2. 눔바니


원본 크기로 보시려면 그림을 클릭하세요.


원본 크기로 보시려면 그림을 클릭하세요.


원본 크기로 보시려면 그림을 클릭하세요.



3. 도라도


원본 크기로 보시려면 그림을 클릭하세요.


원본 크기로 보시려면 그림을 클릭하세요.



4. 아누비스 신전


원본 크기로 보시려면 그림을 클릭하세요.


원본 크기로 보시려면 그림을 클릭하세요.



5. 아이헨발데


원본 크기로 보시려면 그림을 클릭하세요.


원본 크기로 보시려면 그림을 클릭하세요.



6. 감시기지 지브롤터


원본 크기로 보시려면 그림을 클릭하세요.


원본 크기로 보시려면 그림을 클릭하세요.



7. 하나무라


원본 크기로 보시려면 그림을 클릭하세요.


원본 크기로 보시려면 그림을 클릭하세요.



8. 할리우드


원본 크기로 보시려면 그림을 클릭하세요.


원본 크기로 보시려면 그림을 클릭하세요.




글을 마치며..

 

여기까지 달려오신 모든 분들 정말 고생 많으셨습니다. 아나공략은 예전부터 쓰고 싶다는 생각만 있었지 글로 쓸 엄두가 나지 않았는데 막상 쓰다보니 양이 참 많이 나와서 저도 당황했습니다. 그러나 진짜 승자는 여러분들입니다. 이 긴 글을 여기까지 정독해 주신 분이 한 분이라도 있어서 도움이 됐다면 정말 행복한 일입니다. 


아나는 필요한 게 많습니다. 좋은 에임과 센스, 순간적인 판단력, 전반적인 게임 이해도, 브리핑과 오더능력, 효과적인 스킬 활용 모두 아나가 가져야 하는 지식들입니다. 그러나 가장 중요한건 에임도 센스도 아닌  ‘팀에 대한 책임감’입니다. 아나는 내가 죽으면 팀이 죽는다는 생각으로 플레이해야합니다. 특히 수비팀 아나가 먼저 짤리면 유지력이 떨어진 팀원들은 속수무책으로 죽어갈 확률이 높습니다. 당신이 아나라면 최대한 생존하면서 아군이 항상 풀피를 유지하는 걸 목표로 플레이 해 주세요. 나머지 능력들은 꾸준히 아나를 플레이 하면서 차차 채워나가실 수 있을 겁니다. 



“아나는 최대한 죽지 않으면서 팀 전체의 유지력을 책임져야 하는 영웅이다”

1. 이번에 아나의 생체수류탄이 전반적인 조정을 받았습니다. 지속시간이 5초에서 4초로 감소했고 치유량 증폭 또한 100%에서 50%로 감소했습니다. 체감이 크지는 않지만 이번 너프로 인해 생체수류탄을 더 정교하게 던질 필요가 있습니다. 따라서 생체수류탄을 적재적소에 활용하기 위해서 아나가 리스크를 감수하고 조금 더 적극적인 움직임을 보일 때 좋은 결과가 나올 확률이 높습니다. 또한 예전처럼 좀비를 풀피로 살리는 건 불가능해졌으니 참고 바랍니다. (너프의 체감이 많이 있는 편은 아닙니다)

 



2. 대치과정에서 라인하르트에게 힐밴을 넣으려면 화염강타 모션에 들어가자마자 바로 생체수류탄을 던져야 합니다. 반응이 늦는 경우 라인이 다시 방패를 올려서 막게 됩니다. 상대 라인에게 힐밴을 넣으면 아군 라인혼자 힐을 받으면서 적 라인을 압박할 수 있게 되고 적 라인은 일방적으로 맞던가 아니면 궁지에 몰려서 궁을 쓰던가 하나의 옵션을 선택해야합니다. 이 때 라인간의 방벽싸움과 심리전에서 아군이 유리해질 수 있습니다.

 



3. 로드호그가 끌어온 적 탱커를 수면총으로 재워주세요. 돼지를 너프하랬더니 저팔계를 만들어버려서 매우 강력합니다. 그러나 여전히 탱커를 원콤을 내지는 못하기 때문에 적 탱커에 그랩을 적중시켰을 때 타이밍 맞춰서 수면총을 넣어주게 되면 데미지를 다 받고 이후에 자게 됩니다. 이후 자고 있는 적 탱커를 마무리 해 주면 되겠습니다. 원래 메이나 리퍼, 아나 같은 영웅들에게도 똑같이 해줬었는데 이제 확정적으로 원콤이 나기 때문에 주변에 아군 로드호그를 밀거나 방해할 수 있는 영웅이 있는게 아니라면 피가 적은 영웅들은 수면총을 안 써도 좋습니다.

 



4. 트레이서 상대를 많이 어려워하시는데 노줌으로 견제하다가 점멸 직후 패줌으로 맞추는 게 저는 제일 편합니다. 트레랑 싸울때는 아군에게 적 트레가 아나를 본다는 걸 알려주는 게 제일 중요하며 아나 본인의 피관리를 잘 해야합니다. 아나가 트레한테 평타 한 대만 맞춰도 역행이 빠질 수밖에 없기 때문에 (두 대 맞으면 사망) 침착하게 한 대만 먼저 맞춰주면 죽음의 위협에서는 대부분 벗어날 수 있습니다. 수면총은 트레이서가 아나를 보고 있지 않을 때 쏴야 재울 확률이 높고 혹은 점멸이 없거나 하나 남은 경우를 노리는 게 좋습니다. 점멸이 없으면 잘 것이고 하나 남은 상황이라면 마지막 점멸을 강제할 수 있습니다. 트레이서가 아나한테 직접적인 생명의 위협을 주지 않는 경우는 팀원들 체력관리부터 신경써야 합니다. 아나 트레를 잡겠다고 정신팔고 있으면 그 사이 아군들은 다 죽어나갈겁니다.

 



5. 겐지 상대의 정석은 생체수류탄을 땅에 던지는 것과 수면총을 튕겨내기 이후에 쏘는 것입니다. 튕겨내기 여부를 모른 상태에서 수면총을 사용한 경우 반사당할 것을 생각해서 미리 피해야 합니다. 수면총은 2단 점프 이후 사용하는 게 가장 맞추기 쉽습니다. 겐지들은 보통 튕겨내기를 쓰고 앞이나 뒤로 질풍참을 쓰는 경우가 많으니 대비하는 게 좋습니다. 용검 재우는 건 침착하게 쏴야 한다고 밖에 말씀드릴 수 없겠네요. 잘하는 겐지의 경우 일단 진입한 이후에는 수면총을 맞출 수 없다고 봐야합니다. 그렇기 때문에 용검 모션 때 재우는 게 가장 좋습니다. 겐지가 아나를 먼저 노릴 것 같으면 용검이후 질풍참 궤적에 수면총을 써 주고, 솔저나 맥크리를 먼저 보러 갈 것 같다면 (솔, 맥이 궁이 있으면 겐지는 솔, 맥부터 노릴 겁니다) 위치를 예상하고 수면총 대기를 해주세요.

 



5. 아나는 상대편 파라&메르시, 아나, 젠야타의 힐러진과 맥크리같이 피가 적은 영웅을 먼저 저격해 주는 게 좋습니다. 아나는 상대 힐러를 견제하는데 특화되어 있으며 실제로 메르시를 아나가 잡는 일도 많이 있습니다. 상대 아나가 혹시 줌을 키고 힐을 주고 있다면 수면총을 쏘면 높은 확률로 재울 수 있으니 참고 바랍니다.

 



6. 수면총을 움직이면서 쏠 경우 궤적이 바뀝니다. 많은 초보자분들이 움직이면서 수면총을 쏘다가 적중시키지 못하는 경우가 많습니다. 따라서 수면총이 숙달되기 전까지는 수면총을 쏠 때 아주 잠깐 움직임을 멈추고 쏘는 걸 추천합니다.

 



7. 아나는 탄창 가는 시간이 ‘매우’ 오래 걸립니다. 탄창관리는 아나의 기본이며 불가피한 상황이 아님에도 탄창을 갈다가 팀원이 죽는 일은 있어서는 안됩니다. 만약 아군 탱커가 맞고 있는데 탄창이 얼마 없는 상황이라면 수류탄부터 던지고 힐을 줄 수 있어야 합니다. 혹시 탄창수가 부족하고 맞고 있는 아군이 있다면 아나 총알 별로 없다고 말해주면 팀원들이 알아서 숨던가 사리던가 할 겁니다.

 



8. 아나 뽕을 받은 탱커들(라인, 호그, 디바, 원숭이)는 움직임이 단순해지기 때문에 정확하게 재워줘야합니다. 또한 수면총을 쏘기 전에 ‘xx 재울게요. 깨우지마세요’ 라고 이야기해야합니다. 제 경험상 수면총을 맞추고 깨우지 말라고 이야기하면 말할 때 벌써 일어나 있습니다. 석양과 리퍼궁은 충분히 예상할 수 있는 위치에서 나오기 때문에 생각하고 바로 재워줍시다. 전술조준경을 킨 솔저의 경우도 움직임이 단순해지고 또 좌우로 점프를 하는 경우도 보이는 데 착지시점이나 움직임을 잘 보고 침착하게 수면총을 쏘면 꽤 높은 확률로 재울 수 있습니다.

 



9. (엄근진) 아나가 보이스를 안하면 본래 능력의 반도 발휘할 수 없습니다. 이전 공략을 참조하시면 알겠지만 아나가 필수적으로 해 줘야 하는 브리핑이 정말 많습니다. 대표적으로 힐을 줄 수 없는 상황, 수면총, 힐밴, 궁대기 브리핑이 있는데 하나같이 아군의 호응을 필요로 합니다. 따라서 당신이 아나인데 마이크를 안 한다면 팀의 전력을 좀먹고 있는겁니다.

 

 

10. 대치과정에서 디바에게 힐밴을 넣을 때는 매트릭스가 꺼지는 걸 보고 바로 던지면 다음 매트릭스가 돌기 전에 생체수류탄이 디바에게 적중합니다. 이후 디바가 체력이 없어서 도망갈 때 수면총으로 재워주면 됩니다. 현실은 패치이후 호그한테 끌리면 바로 터지는 방산비리라 디바를 만날 일이 많이 없어보입니다

 



11. 자리야가 1:1로 아나에게 다가올 때는 대부분 자기방벽이 있습니다. 따라서 수면총을 먼저 쓰면 방벽에 먹히고 죽을 일만 남게됩니다. 고로 방벽이 빠지기 전에는 평타로 견제하거나 생체수류탄을 통해 자기방벽을 강제한 뒤 방벽이 빠지고 나면 수면총을 쓰는 걸 권장합니다. 물론 자리야 방벽이 빠졌다고 팀적으로 콜이 있거나 자리야가 아나를 보는 경우가 아니라면 그냥 재우시면 됩니다.

 



12. 아군 디바가 적 뒤로 궁을 날리면 타이밍 맞춰서 방패를 돌린 상대 라인을 재워야 합니다. 이 플레이는 라인궁 + 디바궁 연계와 흡사합니다. 디바가 적 뒤로 궁을 날릴 때 상대 라인은 이를 막기 위해 뒤로 돌게 됩니다. 그 때 아군 라인이 돌아있는 상대 라인에게 망치를 넣으면 라인이 눕게 되고 디바궁이 확정타로 들어갑니다. 여기서 아군 라인이 궁을 써야 할 것을 아나가 수면총으로 대체하게 되는 겁니다. 그러나 패치 이후로 디바의 티어가 많이 떨어진 만큼 실전에서 연습하기는 힘들수도 있을거라고 생각합니다.

 

13. 자리야 방벽 커버가 되거나 아나의 생존이 가능한 경우 상대 라인 방벽 안으로 파고들어서 힐밴을 넣어주거나 수면총을 활용해 상대 라인을 재워주세요. 상대 라인을 재우면 아군 망치를 확정타로 넣거나 돌진을 통해 한 방에 제거할 수 있습니다. 라인에게 힐밴을 넣었을 경우의 이점은 위에서 말씀드린 바 있습니다. 특히 적 라인이 본대와 살짝 떨어져 있는 상황 (도라도 A 거점 다리 아래 라인 혼자 비비는 상황이라거나)에서 아나가 파고들어서 상대 라인에게 스킬을 넣는다면 상대 라인을 짤라낼 수 있는 확률이 높으니 과감하게 시도해 보는 게 좋습니다.

 



14. 아군 자리야가 중력자탄으로 적 다수를 묶었는데 라인이 방벽을 들고 있는 경우도 마찬가지로 안으로 파고들어서 땅에 힐밴을 넣어주면 됩니다. 아나도 피가 차기 때문에 생각보다 죽을 위험이 적고 혹시 죽더라도 일단 힐밴이 다 들어가면 아군이 이득을 볼 확률이 더 높습니다. 아군 자리야가 중력자탄을 공중(기둥)에 묶은 경우 라인방벽 아래에 있는 땅이나 방벽 위 기둥 쪽을 향해 생체수류탄을 던지면 힐밴을 넣을 수 있습니다.

 



15. 호그 갈고리에 끌려간 아군을 세이브하려면 갈고리가 팀원에게 닿기 전에 미리 수면총을 써야합니다. 다시말해 아군이 갈고리에 피격당한 상황에서는 살리는 게 불가능하고 상대 호그의 그랩모션이 있을 때 바로 수면총을 발사해야 끌려가는 아군을 살릴 수 있습니다. 호그에게 아군이 끌린 걸 보고 수면총을 쓰는 게 아니라, 호그에게 끌릴 것 같으면 바로 써야 합니다.

 

16. 패줌은 선택이 아니라 필수입니다. 또한 간지스킬이 아닌 실전에 최적화된 기술입니다. 아군에게 힐을 줄 때, 1:1 맞다이를 할 때, 딸피를 처리할 때 패줌의 활용도가 극대화됩니다. 또한 패줌은 감으로 쓰는게 아니고 ‘정확하게 조준하고’ 쏘는 겁니다. 패줌은 연습한 만큼 늡니다. 연습하는 방법은 게임 시작전 대기시간에 계속 패줌을 해서 움직이는 아군을 맞추는 연습을 하고 인게임에서 비교적 여유가 있을 때 시도해보세요.

 



17. 요새 공격형 아나가 많이 보급되고 있습니다. 적극적인 위치에서 힐밴을 넣고 고지대를 미리 선점한 뒤 저격과 아군지원을 동시에 하는 매우 스타일리쉬한 플레이입니다. 그러나 이런 플레이는 스킬 활용능력이 좋은 상황에서 포지셔닝 센스까지 있을 때 가능한 플레이입니다. 아나의 숙련도가 높지 않은 상태에서 이런식의 플레이를 추구하면 높은 확률로 팀원들한테 욕을 먹게됩니다. 명심하세요 공격형 아나라는 건 언제 힐을 주고 언제 딜을 넣을지 확실하게 알고, 생체수류탄을 적재적소에 활용할 수 있으며 나를 물러오는 적을 재우고 내 한 몸 보전할 능력이 될 때 비로소 사용할 수 있는겁니다.

 



18. 힐밴각은 적이 주는 게 아니라 아나가 만드는 겁니다. 대표적으로 고지대를 선점하고 생체수류탄을 던지거나 아군 라인이 정면에 방벽을 키고 버티사이 아나가 옆으로 돌아서 힐밴을 넣는 방법이 있습니다. 공격형 아나의 주 운용방식이 이와같은 힐밴각을 잡는 것으로부터 시작됩니다. 아나의 이런 운용방식이 극대화 되는건 쟁탈전입니다. 화물과 거점은 비교적 안정적으로 힐링에 집중하고 생체수류탄으로 변수를 만드는 정도지만 쟁탈전에서는 힐밴이 곧 전투의 성패로 연결되는 경우가 정말 많습니다. 따라서 쟁탈전의 경우는 아나가 얼마나 적극적으로 변수를 창출해 줄 수 있는지가 타 맵들에 비해 중요하다고 말할 수 있겠습니다.

 



19. 아나의 딜량은 힐량만큼이나 중요합니다. 물론 아나의 기본은 힐을 주는 게 맞지만 힐은 힐대로 주면서 딜까지 넣을 수 있다면 베스트입니다. 실제로 아군이 방벽과 매트릭스로 버티는 상황에서 아나의 케어가 필요 없는 시점이 있고, 아군이 좀 맞고 있어도 아나가 그 사이 적 영웅을 저격해 주는 게 좋을 타이밍이 분명히 나옵니다. 마찬가지로 방벽딜을 하는 건 권하지는 않지만 힐 줄 사람이 없는데 아무것도 안하는 것보다는 방벽딜을 하는 게 당연히 좋습니다. 따라서 아나는 딜을 할 수 있는 타이밍에는 최대한 딜을 넣어주는 게 좋으며 딜량은 아나의 주요 지표중 하나입니다. 저 역시 현재는 시즌2(아나 공략을 쓸 시점) 보다 아나를 훨씬 공격적으로 운용하고 있으며 (지금 생각해보면 제 아나도 시즌2에는 플레이상으로 부족한 점이 정말 많았습니다) 힐만 집중하는 것보다는 훨씬 더 좋은 성적을 낼 수 있습니다.

 



20. 탱커를 많이 플레이해보면 아나의 숙련도도 올라갑니다. 특히 라인, 디바, 원숭이 등 아나의 케어가 필요한 탱커들을 많이 해보세요. 아나만 플레이한다면 본인이 힐을 주고 있을 때 탱커들이 시야에서 사라지는 게 답답하다고 느끼겠지만 탱커를 실제로 플레이해보면 탱커들의 마음과 움직임이 이해가 됩니다. 이렇게 역지사지의 깨달음을 얻고나면 아나는 더 나은 위치를 잡을 수 있게 되고 탱커들의 위치를 더 잘 조율할 수 있습니다.

 



21. 아나는 뽕을 항상 정확하게 줄 수 있어야 합니다. 따라서 정확한 타이밍에 정확한 영웅에게 뽕을 줄 수 있도록 위치를 변경해야 하는 건 아나의 몫입니다. 간혹 흥이 넘치는 개구리들이 절묘한 점프로 뽕을 앗아가곤 하는데 이건 근본적으로 아나 잘못입니다. ‘겐지님 위로 질풍참 그으시면 뽕 드릴게요’ ‘솔져님 왼쪽으로 나오시면 뽕 드릴게요’ 등으로 아군 위치를 조율하고 뽕을 주면 뽕미스를 최소화 시킬 수 있습니다. 그런데 솔져님 왼쪽으로 나오시면 뽕 드릴게요라고 말을 했는데 눈치없는 개구리가 왼쪽으로 같이 뛴다면 그때는 루시우를 욕해도 괜찮습니다.

 



22. 생체수류탄을 던질 때 주의해야 할 점 또 한 가지는 아군의 위치입니다. 완벽한 힐밴 타이밍에 던진 생체수류탄이 아군 라인의 듬직한 어깨에 막혀버리면 아나의 입장에서 한숨이 절로 나옵니다. 그러나 날아가는 수류탄을 막은 아군은 아무 잘못이 없으며 아나가 잘못 던진겁니다. 생체수류탄을 힐밴을 위해 던지는 경우 항상 앞에 있는 아군이 막을 수 있는 가능성을 생각해야 하며 그걸 피해서 던지는 힐밴이 진정한 실력자의 것입니다.

 



23. 아나는 위도우랑 1:1을 절대 해서는 안 됩니다. 아나는 저격모드에 들어가면 움직임이 고정되기 때문에 위도우한테 높은 확률로 헤드를 따이게 됩니다. 아나가 위도우를 견제해야된다면 아군 라인에게 방벽을 들고 있어달라고 부탁하고 견제하시면 됩니다. 간혹 라인이 화염강타를 날리는 동안 아나 머리가 터지는 경우가 있는데 이건 라인 잘못이 아닙니다. 아나가 라인한테 위도우 잡을테니 방벽 내리지 말고 들고 있으라고 말해줘야 하는 부분입니다.

 



24. 아나가 후방포지를 잡는 경우 상대 라인의 망치각과 자리야의 중력자탄 각이 잘 보입니다. 대부분의 팀원들은 시야가 좁기 때문에 상대의 망치나 중력자탄각을 아나만큼 잘 볼 수 없습니다. 이 때 아나가 브리핑을 통해 ‘망치각 주지 마세요’ ‘자리야 중력자각봐요’ 라고 확실하게 전달을 해 주는 경우 팀원들이 빠르게 상황을 파악하고 대처할 수 있습니다. 보통 이 말 하자마자 다 귀신같이 누워있기는 합니다.

 



25. 아나는 기본공격(80데미지) 두 번과 생체수류탄(60데미지)으로 피 200짜리 적을 잡을 수 있습니다(80+80+60=220). 이 말은 평타 한 발을 먼저 적중시킨 상황이라면 적의 예측 도망지점에 노줌샷 + 생체수류탄을 해서 마무리를 하거나 혹은 생체수류탄을 먼저 던져서 피를60으로 만들어 놓은 뒤 패줌이나 저격으로 마무리 하는 방식으로 응용할 수 있습니다. 얼핏보면 이런 당연한 소리를 왜 팁이라고 적는가 싶지만 실전에서 정말 다양하게 응용할 수 있습니다. 아나의 원콤 콤보에 대해서 항상 인지하고 있어야 이를 효과적으로 활용할 수 있습니다.

 



26. 생체수류탄을 근거리에서 사용하는 게 아니라면 점프해서 던지는 게 좋을 때가 많습니다. 생체수류탄은 곡사가 생각보다 완만해서 원거리의 경우 정확히 원하는 지점에 던지기 어렵습니다. 따라서 점프 후 내려서 던지는 방식으로 입사각을 줄이는 경우 명중률이 더 좋습니다. 또한 이미 언급한 것처럼 아군 라인이 지나가는 생체수류탄을 어깨빵 할 수도 있는데 이를 방지하기 위한 좋은 방법이 점프해서 생체수류탄을 던지는 겁니다.

 



27. 저는 수면총을 쓴 뒤에 인사를 생활화하며 마지막 남은 적인경우 감정표현도 종종 합니다. 상대방이 미친x 아니냐고 전쳇으로 욕을 하면 성공입니다. 아나가 재워놓고 인사하고 감정표현 하고 있으면 당사자는 기분이 매우 더럽습니다. 그럼 그 상대는 더 악착같이 아나를 잡으려고 할 것이며 전반적인 멘탈 유지에 어려움을 겪게 됩니다. 이는 상대의 에임과 플레이를 흔들어뜨릴 수 있는 전략적인 행동이기 떄문에 습관화 시켜두시면 활용가능성이 높습니다. 저 아나코패스 아닙니다.





28. 이제 원점으로 돌아와볼까요? 아나의 기본은 살아남는 것입니다. 상대팀 모두가 아나를 물러올 것이며 적군에게 아나는 항상 최우선순위로 죽여야 할 대상입니다. 아나가 뒤에 떨어져 있다면 누군가 돌아서 아나를 잡으러 올 것이고 앞에 있다면 순간적인 점사가 들어올겁니다. 아나가 포지셔닝을 어디에 잡는가와 관계없이 아나는 항상 위험에 노출되어 있습니다. 어떤 일이 있어도 살아남으세요. 아나가 플레이 하는 건 FPS가 아니라 생존게임입니다. 아나가 죽는다면 아군 팀원들도 순차적으로 킬로그에 초상화를 올릴 것입니다. 필요하다면 아군을 미끼로 주는 한이 있어도 아나가 살아남아야 합니다.

 

29. 첫 번째 글인 ‘호랭이가 들려주는 아나의 모든 것’과 이번 ‘호랭이가 들려주는 아나의 모든 것(심화편)’을 읽으셨다면 아나에 대한 전반적인 이해는 다 하셨다고 보시면 됩니다. 따라서 이 글을 다 읽으셨다면 지금 바로 아프리카, 트위치, 유튜브 등 다양한 플랫폼에 있는 천상계 아나들의 플레이 영상을 보러 가시기 바랍니다. 글로 배운 내용을 실제로 플레이와 연관시켜보면 도움이 많이 됩니다. 그 중에서도 아나의 포지셔닝, 패줌 활용, 스킬 활용 및 궁 타이밍등을 중점적으로 관찰하고 특히 아나가 힐밴각을 어떻게 잡는지를 주의 깊게 보시면 도움이 많이 되실 겁니다. 예외가 있다면 류제홍 방송은 그가 류제홍이라 가능한 것 뿐입니다. 당신은 류제홍이 아닙니다.