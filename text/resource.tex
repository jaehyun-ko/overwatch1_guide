 \chapter{오버워치의 자원}{\label{sec:res}}
% 뭘 쓰는게 좋을까요?
\section{introduction : 오버워치의 자원?}\label{res:intro}
오버워치는 서로의 자원을 교환하며 상대의 진입은 막고, 아군의 진입은 돕는 게임이다. 오버워치에서의 자원의 종류와 특징에 대해서 알아보고, 어떻게 자원을 교환해야 유리한 게임이 진행되는지 생각해 본다.

오버워치의 자원
자원의 획득
자원의 소모 -교환
자원의 낭비
자원의 흡수
자원의 상대성
자원의 순환


%  레킹볼을 플레이할 때 아군의 주는 방벽을 받고 굴러다니면서 수면총과 방패 밀쳐내기 등 CC기를 빼면서 보호막을 채우고, CC기 사이클이 돌기 전에 파일드라이버를 사용하면 상대는 '주는 방벽' 이라는 낮은 코스트 기술에 CC기를 소모하였으므로 파일드라이버 후 CC기 포커싱을 사용할 수 없게 된다. 레킹볼 입장에서는 상대의 견제가 약해졌으므로 오랜 시간 동안 어그로를 끌 수 있다.
% 마지막으로 상대방 용검턴에 지원가가 빠져있는다거나, 자탄턴에 팀원이 나누어 자리를 잡으면 템포를 늦출 수 있다.

\section{자원의 종류}
\subsection{주의력}
신경론
시야, 사운드
\subsection{체력}
오버워치의 체력에는 네 가지 종류가 있다.
\begin{enumerate}
    \item 아머 : 한 대당 데미지 3씩 감소시켜주는 체력이다. 탱커들의 아머를 채워주면 최대한의 힐 효율을 볼 수 있다.
    \item 파란색 보호막 : 상대의 궁극기를 채워주지 않는다.
    \item 하늘색 보호막 : 데미지를 입지 않고 3초 이상 있으면 다시 채워지는 체력이다.
    \item 흰색 체력 : 가장 일반적인 체력이다.
    
\end{enumerate}
\subsection{기본공격, 스킬, 궁극기}
각 영웅의 기본공격과 스킬, 궁극기라는 자원은 다음 여섯 가지 목적으로 구분할 수 있다.
\begin{enumerate}
    \item 이동
    \item 이동 방해
    \item 힐
    \item 힐 차단
    \item 딜링
    \item 딜 차단
\end{enumerate}
\subsection{지형 calling}
코너 기준 번호판처럼 생각, 위 아래로 구분
\subsection{지형과 포지셔닝}
\begin{enumerate}
    \item 고지대는 왜 중요할까?
     오버워치를 하다 보면, 마스터 후반 유저들부터 고지대싸움을 치열하게 하는 모습을 볼 수 있다.
     그 이유는 고지대를 뺏으려고 진입하던 중에 변수가 생기거나, 자원이 소모되거나, 
\end{enumerate}

    % \begin{figure}
    %     \centering
    %     \includegraphics[scale = 0.2]{figures/step5.png}
    %     \caption{STEP 1 ~ STEP 5를 수행한 후 μvision 화면}
    %     \label{fig:step5}
    % \end{figure}
 
% \clearpage
