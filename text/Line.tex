\chapter{Line:오버워치의 전선개념}{\label{sec:state}}
\section{introduction : 오버워치의 상태 구분}
오버워치를 플레이하다보면 푸쉬 콜이나 '빨자' 는 말을 자주 들을 수 있다. 이 장에서는 해당 개념들을 Push, Face-off, Pull이라는 세 State로 체계화한다. 이를 바탕으로 어떻게 State 사이의 전환이 일어나는지 알아보고 각 State에서 자원의 우위를 점하는 방법을 이해한다.
\section{오버워치의 세 가지 상태}
\input{diagrams/state_diagram}
\subsection{Hold \& Filter}
상대의 위치 등 정보를 수집하고, 자원 교환을 하는 상태이다. 흔히 오버워치에서 사용하는 오브젝트의 점령 및 화물의 이동을 멈추는 개념보다는 조금 더 넓은 범위로, 유리한 위치에서 상대의 진입을 저지하는 것까지 포함한다. 
홀딩을 하는 조건은 다음과 같다.
\begin{enumerate}
    \item 이 위치에서 홀딩했을 때, 상대의 스킬 투자가 없다면 탱커가 지속적으로 힐을 받을 수 있다.
    \item 이 위치에서 홀딩했을 때, 아군 플랭커들의 스킬을 적게 소모하면서 진입할 수 있고, 원거리 딜링이 유리하다.
    \item 이 위치에서 홀딩했을 때, 상대가 플랭킹하지 않으면(리스크를 감수하지 않으면) 딜각이 좁다.
    \item 상대 대다수가 이 구간을 지나야 한다.
\end{enumerate}
홀딩하는 동안, 다음 변수에 주의한다.
\subsection{Push}
자원을 투자해서 주도권을 가지고 자리를 가져오는 상태
\subsection{Pull}
상대의 자원을 소모시키고 아군의 자원을 모으는 상태
\section{Start State}
\subsection{Hold로부터 시작하는 경우}
궁극기 파밍 및 상대 진형에 대한 정보를 수집한다.
\subsection{Push로부터 시작하는 경우}
자원을 투자해서 주도권을 가지고 자리를 가져오는 상태
\subsection{Pull로부터 시작하는 경우}
상대의 자원을 소모시키고 아군의 자원을 모으는 상태


210715 깨달은 바가 있어서 메모.
Hold도 결국 자원을 투자해야 하므로 Push의 일종이라고 볼 수 있음. 그러나 오직 간보기만.
Push는 메인탱커 뿐만 아니라 본대까지 진입할 수 있는 상황으로 구분.
가운데 단계로 Reset을 추가. 이는 터져서 재정비하는 거라고 생각할 수 있으나 체력관리도 이에 포함됨
Pull 이전에 상대를 흘리는 단계인 Filter를 추가.



\section{State Transition}
\subsection{자원의 우위에 의한 State Transition}
\begin{enumerate}
    \item 스킬 차이
    \item 궁극기 차이
\end{enumerate}
\subsection{진형의 우위에 의한 State Transition} 