\subsection{브리기테와 함께하는 다이브 조합}
\subsubsection{introduction : 브리기테의 사기성}
오버워치 전성기의 패왕이었던 윈디 조합은 222 고정과 2방벽 이후 벨런싱이 어느 정도 마무리된 현재 사용되지 못하고 있다. 이 이유는 윈스턴 또는 디바의 다이브를 처음부터 차단할 수 있는 브리기테의 도리께 투척 때문이다\footnote{부가적인 이유로는 윈스턴의 방벽 및 디바의 매트릭스와 상관없이 힐이 가능한 수리킷}\footnote{그래서 브리가 못하거나 안 나오면 디바를 쓰는 게 더 좋다고 생각함.}. 디바를 기용할 경우 윈스턴의 방벽과 상대 브리기테의 shift를 교환해야 하므로 너무 큰 낭비가 생긴다. 따라서 이를 막기 위해 자리야를 사용하게 되었다. 주는방벽과 동시에 다이브할 경우 브리기테의 밀쳐내기를 막을 수 있을 뿐만 아니라 방벽 사용 전 데미지를 흡수할 수 있기 때문이다.

\subsection{기본 조합 : 윈자애트아브}
힐러 및 자리야로 구성된 본대, 윈스턴과 트레이서로 구성된 별동대, 애쉬의 말뚝딜의 삼박자로 구성된 갉아먹기 조합이다. 상대의 진입을 차단하면서 궁극기를 모은 뒤, 자원의 우위가 확실할 때 자리를 먹게 된다.  
\subsubsection{윈스턴의 역할}
\subsubsection{자리야의 역할}
\subsubsection{애쉬의 역할}
\subsubsection{트레이서의 역할}
\subsubsection{아나의 역할}
\subsubsection{브리기테의 역할}