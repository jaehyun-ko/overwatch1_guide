
Introduction
Before I get into the outline of this Guide I would like to introduce myself and talk about my qualifications. My name is Ridouan Bouzrou, I go by “ioStux” and I’m a 21-year-old professional coach for Overwatch.

The last major teams I worked with were XL2 Academy (Co-Head Coach), the Contenders team of the New York Excelsior, and Uprising Academy (Head Coach), formerly known as Toronto Esports and the Contenders team of the Boston Uprising. I also offer private coaching to individuals and have created content on my Youtube channel in the past.  

This guide is aimed at professional teams and high tier competitive players. I use terminology that might be unfamiliar to less experienced readers and a lot of this is simply irrelevant unless you are playing in a competitive team environment. The concepts I address and the way I explain them is very similar to what I would do working with professional players as their coach, so the phrasing and tone might come off as harsh and too direct to regular players.

Now that I am no longer a part of the Uprising Academy I am allowed to share my knowledge and insights with the public, something that was previously reserved to the players I worked with. If you are actively scrimming then most of the sections within this guide can be implemented in your own gameplay and allow you to better understand how to execute Goats. If you don’t actively scrim but enjoy watching OWL or Contenders then this guide will give you a better understanding of the comp, allowing you to pick up some subtle details you might have previously missed while watching Goats.

I’d like to preface this by saying that the individual “chapters” are really more of a general guideline. Simply reading through the communication chapter, for example, won’t explain everything communication-related in Goats, because I might mention certain aspects in the Ult Management chapter. If you want a full understanding of Goats you will have to read through the entire thing; everything ties together.
Table of Contents
Introduction	1
Table of Contents	2
Communication	3
Pre Fight Planning	4
Important Fundamentals	5
Target Calling	7
Clear Comms and Mid-Fight Planning	9
Communication Example	10
Engages	11
The Three Primary Engages	13
1. Highground engages	13
2. Punishing split targets	14
3. Engaging off of Zarya Bubbles	15
Evaluating Engages	15
Playing the Line	16
Engagements against Spam Comps	18
Target Selection	18
Engage Example	20
Ability Usage	21
Harmony Orb Usage	22
Discord Orb Placement	24
Zarya Bubble Usage	24
Squeezing	25
Case 1: You use your bubble first and they don’t use theirs	26
Case 2: You use your bubble first and they use theirs shortly after	26
Case 3: They use bubble first and you don’t use yours	28
Case 4: They use bubble first and you use yours shortly after	29
Baiting with Bubbles	29
Bubble Usage against Grav Combos	30
Matrix Usage	30
Reinhardt Ability Usage	31
Brigitte Bash Usage	33
Brigitte Armor Pack Usage	34
Armor Management	35
Positioning	36
Abusing Corners	37
Playing around Chokes	38
Dynamic Positioning + Example	39
Highground in Goats	40
Taking advantage of Point Pressure	41
Off Angles and Fanning Out	42
Bodyblocking as D.va	44
Positioning Example	45
Positioning contextualized	47
Ult Usage	50
The Importance of Ult Management	50
Support Ultimates in Goats	51
Communication and Ult Stacking	52
How to get value out of Shatter	53
Bash Shatters	53
Boop Shatter	55
Communication and Individual Responsibility	56
Graviton Surge and D.va Bomb	57
Boop Follow-up	57
Brigitte and Reinhardt Follow-up	58
Brigitte Rally Usage	59
Ult Economy and swapping from/to Goats	60
Goats Variations	62
Disclaimer	62
Sombra Goats	62
Mei Goats	64
Brig McCree	66
Dive Winston Goats	66
Moira Goats	69
Ana Goats	69
Closing Remarks	70



Communication

I’d like to start this guide off by going over the most fundamental part of any composition or playstyle, communication. Individual ability becomes irrelevant if a team isn’t able to communicate effectively. Goats especially is a composition that relies on coordination and everyone being in sync with each other, so understanding individual responsibilities within the team is crucial to even begin improving at Goats.


Pre Fight Planning
First lets address Pre Fight Planning. Pre Fight Planning is important because it prevents comms from getting cluttered during a fight. Talking about which support ultimate you want to use first before the fight means that you won’t have to talk about it in the middle of a fight. Speed is everything, calls need to be quick and efficient, there is no time to think or ask questions, you want to go into each fight knowing exactly what you want to do. To be more specific, some of the things you should talk about before going into a fight as Goats are as follows:

Which Support Ultimate should be used first if both are available? Which offensive ultimate do you want to use? How do you want to set that ultimate up? Where do you want to set it up? How do you want to take this fight, do you want to go for a fast engage? Play it slow and bait out cooldowns? Where should we hold, where should we fall back to, etc. The more you talk before fights the better your understanding of each point is, which means that over time more and more things become second nature and don’t have to be talked about.

Eventually, you arrive at a point where pre-fight planning can be done in 10 seconds because everyone on the team already knows where they should hold by default and how they should rotate during the fight.
I will go into detail on how to answer these questions later on in the guide, but going into a fight everyone on the team needs to know exactly where to play, what to look for in the enemy team, how to react to whatever the enemy team is doing, and when they should use their ultimates.

As with every comp ults need to be tracked after every single teamfight so I won’t go into too much detail, but there are some specific cases that need to be pointed out in Goats especially. If the enemy Zenyatta doesn’t have his Transcendence and your Zarya has her Graviton Surge it is of utmost importance that she knows. Landing a Grav when the enemy team has no Trans against Goats is a free team fight win in the majority of cases, and not communicating that effectively will lead to you missing out on the opportunity because you didn’t use your Grav by the time he gets his Trans.

Important Fundamentals
Once you are done planning out your fight you need to make sure that your engages are communicated effectively. I will go into more detail on when to engage later, but there are two important fundamentals that you need to keep in mind when calling an engage as Goats: 

1. Avoid Indirect Communication
    Goats is all about speed. Good teams will only give you very small windows that you can punish them in, so when an opening reveals itself it is important that you capitalize on it quickly. The only way to do that is to be decisive. With Goats you shoot first, ask questions later. You do not ask questions in Goats when it comes to engages. For example, if the enemy Zarya wasted her Ally Bubble you cannot call “They used Ally Bubble, can we engage here?”. By the time you get an answer, it is too late already.

A common misconception in Goats is that the Main Tank has to call engages, but Goats is not a team comp around 6 individual players, Goats always plays as a Unit. Individual players don’t engage, the team engages.
And it is important that everyone on the team understands exactly when to engage. Your Lucio needs to understand when to engage just as well as your Reinhardt. If the enemy team gives you an opener everyone on the team needs to recognize that opener immediately. If you are playing Zarya and you are confident that this is the right time to engage, you have to call the engage. You can’t ask “They gave us an opener, can we engage here?”. You simply call “They did this, go on X” and engage.

If the engage turns out to be bad (They were too far away to engage on for example, or maybe you didn't’ notice someone on your team wasn’t in a position to follow up on) then that isn’t an issue with you not asking your team whether they are ready, the issue is that you do not fully understand when to engage as Goats, or your teammates weren’t ready when they needed to be.

If the enemy Zarya wasted both of her bubbles you should call an engage (Assuming there is no obvious reason why you shouldn’t engage, i.e. you not having your cooldowns yourself, or you not being fully regrouped). If someone objects to the engage he can always do it after the call, just shout “Don’t!” and you can abort the engage.  Everyone needs to recognize openings and have the confidence to call engages.

Keep in mind that I am not saying that the Brigitte, for example, should call engages over the Reinhardt, the point I am making is that an engage call is an engage call, it doesn’t matter if your Lucio recognizes and calls the engage or if your Reinhardt does. You don’t have to get your teammates approval on the engage before you can engage, if you are confident in the engage you call it, if you aren’t you have to find out why (Maybe the engage wasn’t that convincing, or maybe your understanding of when to engage lacks).
2. Target Call
    Every single engage call in Goats needs to include a target. Calling “No ally bubble, go in!” is not a good call. Calling “No ally bubble, kill Rein” is. If you call an engage without a clear target you will end up split. If the enemy team makes a mistake and you engage, they do not want to take that fight. They want to kite back and stay alive, and then potentially reengage once their cooldowns are back up. While disengaging the enemy team might move into different directions. If you call to just “go in” then everyone on the team will have a  different idea of who to go on based on their positioning.

Someone will focus the Brigitte, someone the Reinhardt, someone the Zarya, someone the D.va. Each of those targets might disengage into a different direction, so the players focussing them will also end up chasing in multiple directions. This splits the team and makes it much more difficult to burst down targets because no everyone can shoot the same target. If you start your engages with a clear target call it makes sure that when the enemy team disengages, everyone is following the same target.

Do not overthink the initial target call, the majority of the time simply calling whoever is closest to your team is completely fine. Do not start an engage by calling “No ally bubble, run on Zarya!” when running on Zarya means you’ll have to run past a Reinhardt and Brigitte. When engaging you want to avoid running past targets or crossing large distances, you want to start dealing damage as fast as possible to take space and create an advantage. If the Zarya is standing right next to the Reinhardt then calling her as the initial target is fine, but you shouldn’t go out of your way and overcomplicate the initial engage just to shoot the “correct target”.  Getting space and a guaranteed kill is much better than putting yourself out of position to chase the “optimal target”.

Target Calling
Target calling is something I also want to go into. Goats is very unique when it comes to target calling because it isn’t really that important which hero calls the target. With Dive, for example, having your Winston call the target is optimal, but whether your Reinhardt or Zarya call targets during team fights in Goats doesn’t make a big difference. You should rotate and engage stacked and together anyways, which means that individual field of view and positioning isn’t that relevant when it comes to calling targets.

I personally select the main target caller based on how vocal someone is, ideally the Rein, Zarya or Zenyatta (Rein because he has no mobility, Zarya because she is your main source of damage, Zen because of Discord orb). This doesn’t mean that a Lucio couldn’t call targets if he really had to (In case everyone else on the team isn’t as vocal for example). Don’t hesitate to experiment with target callers either. If you have someone on the team that is very vocal and super comfortable calling targets then have him try calling targets during fights for a block, don’t force players to call targets just because they play a certain hero, let them play to their strengths. A very vocal and skilled Brigitte player calling targets is much better than a quiet and unconfident Reinhardt calling targets.

Depth in comms is also very important in calls. Someone who is calling with very little energy isn’t great, but someone who is always calling with 100\% energy isn’t optimal either. You want your comms in Goats to have a lot of dynamic range. Info calls need to be communicated calmly, while very urgent engage calls need to be called loudly and quickly. If your “They are rotating right side” sounds just as loud and urgent as “Reinhardt pinned behind us, kill him” your comms won’t be as efficient.




A good way to look at it is like this: Your comms need to be understandable by someone who doesn’t speak your language. If you take someone who can’t speak Korean and put him into a Korean speaking team that plays Goats, he should be able to very clearly tell when an engagement is being called just by the intonation and urgency of the call, even though he doesn’t understand a word. 

Earlier I mentioned that everyone on the team needs to have an equally outstanding understanding of Goats in order to succeed. That everyone should be able to understand when to engage or not just off of experience. Communicating cooldowns before fights are an important part of that. Calling when you are running back from spawn, calling when you use your Ally Bubble and when it is up again, calling when you get high energy, calling when you land a discord on someone.

All of that is very important, and sharing information between teammates makes sure that everyone is on the same page. A Zarya, for example, might see an engage but she decides not to act upon it because she knows that she doesn’t have her bubbles ready yet. Your Reinhardt might not know that, see the opener and decide to call an engage. You then have to abort that engage because you don’t have bubbles, which wasted everyone's time. Sharing your perspective constantly before fights ensures that everyone has access to the same information and as a result ensure that everyone's decision making process looks the same.

Clear Comms and Mid-Fight Planning
The last point I’d like to address regarding communication is mid-fight setups. Pre-fight planning is very important but plans change. Sometimes fights drag on for longer than expected (Mainly when playing against unskilled Goats teams) which means that ultimates become available you didn’t take into account in your pre-fight planning. In that case, it is very important that you are flexible. 
Setting up a Grav Combo, for example, is something that you shouldn’t just do before a fight if you end up getting it during a fight you need to be able to quickly set it up during a fight. This is only possible if your pre-fight planning and target calling is solid. If you don’t pre-fight plan properly or your target calling is messy, then your comms will be so cluttered during fights that you simply cannot set up ult combos.

A team I visited recently, for example, had extremely cluttered comms because they would say “nice” whenever something even remotely interesting happened. After every kill, each landed ultimate, every successful rotation they would all say “nice”, which left no room for constructive calls. I hope they are fine with me leaking their comms, but here is a short clip of them calling in a Goats vs Goats mirror matchup: https://www.youtube.com/watch?v=UAdqolU8ZIA . If you notice that your fight win rate decreases as the fight length increases, you have to ensure that you are making continuous calls during a fight.

A lot of teams plan their ultimates beforehand and call a target on the engage, but then don’t call anything afterward. They end up splitting because there is no clear target, they end up holding onto potentially fight winning ultimates because they don’t set them up during the fight, they don’t communicate their cooldowns during fights which gets everyone out of sync. Communication in Goats needs to be extremely consistent, you cannot slow down at any point in the round. There is no “downtime”.







Communication Example
To better understand the past paragraphs here is an example of what good communication in Goats should look like.

1. Ult tracking (Usually Lucio)
    “They have Trans Rally Grav Combo,”
2. Which ultimates do you want to use if any? (Individual responsibility, Zarya calls when she wants to Grav, Rein calls when he wants to Shatter, etc.)
    “I want to look for a Shatter this fight”
3. Where do you want to set up? (Individual responsibility. Zarya has to set up where she wants to play for the Grav for example. If you don’t have any ultimates the Reinhardt calls which corner he wants to play, more on that later.)
    “Playing around this choke, call for speed and we can look for a bash shatter” or “Play this corner, pressure their Rein and look for an opening”
4. Calling Cooldowns and waiting for an opener
    “I have bubble, pressuring the Rein Shield, etc.”
5. Calling an engage
    “They used Ally bubble, kill Rein”
6. Mid Fight Calls
    “Rein Rein Rein *Reinhardt dies* kill D.va behind us D.va D.va D.va” or “Rein Rein Rein, ok don’t push past the choke, stack up, heal up and we are looking for a Grav Combo here”.
7. Back to Ult tracking as soon as the fight is won (You can start Ult tracking even if you are still cleaning up kills assuming that the fight is won. If you got 2 kills you want to keep calling targets to make sure you clean up the fight properly.
But if everyone is dead and you are only demeching and stalling the D.va you can start ult tracking so you have more time to set up for the next fight.

You can make tweaks to this structure depending on your team's needs, but this is a solid foundation to build on.
Engages
Knowing when to engage is key in Goats, and as mentioned previously is a shared responsibility within the entire team. No matter what role you play, you are responsible for fully understanding what kind of mistakes to look for and when an engage is appropriate. If someone on your team calls an engage and you do not understand why then you are letting your team down. If you don’t understand the engage, you can’t be confident about it. And to engage successfully everyone on the team has to be confident that committing and going in at that moment in time is 100\% the correct decision.

Going over every possible engage would exceed the scope of this guide, but I’d like to address the 3 main things you can engage off in Goats. To prevent confusion I’ll address ult engages first as they are not included here. In Goats, it is not necessary to force ultimates every single fight. If you play too proactively and engage a fight with an ultimate you aren’t giving the enemy team a chance to mess up, and the enemy team messing up is the best thing that can happen.

If you are looking for a Grav Combo, for example, you don’t have to just run in, Grav straight away and put a D.va bomb in it. Especially on defense abusing your positioning and giving the enemy team room to make mistakes is important. There are some exceptions of course (If you have 6 ultimates and the enemy Zen doesn’t have Trans you can rush a Grav, or if you know this is the last fight anyways, you don’t have to hesitate as much with quickly engaging with your ultimates.

If you are playing Goats against Sombra Goats you also have to play a lot more proactive with your engages. I will go into more detail about those things later.). Here’s an example. If you find yourself in a Goats vs Goats mirror on 2nd point Kings Row Defense, you might not want to engage with ultimates. The choke at the start is incredibly strong, and engaging with ultimates will set you up for a difficult fight afterwards.
Depending on the situation it is usually more beneficial to play the corner, abuse the choke, waste the enemy team's time and force them to make a move. The ideal outcome is that they lose a lot of time, they lose 1 or 2 ultimates, and they lose the fight because they made a mistake while fighting an uphill battle inside that choke.

If you were to just rush in with a Grav D.va bomb you could win the fight, but the enemy team just quickly regroups and then pushes through the choke because you have no ultimates up your sleeve to pressure them back. If you are in a good position and there is no time pressure you do not have to engage. Play slow, don’t make a mistake, pressure them and look to win the fight off of an enemy mistake, instead of just brute forcing it by throwing a bunch of ultimates. 

Winning a teamfight and being even or ahead in ultimates is vastly superior to winning a teamfight and being at a massive ult disadvantage afterwards. Don’t just analyze the fights you lose in Goats mirrors, look at the fights you win as well and ask yourself how you could have won them harder.












The Three Primary Engages
The 3 most common engages are as follows:

1. Highground engages
    Pretty self-explanatory, if you are defending and the enemy team is playing on high ground you kill whoever drops first. For example, Lijiang Night Market. You just captured the point and you are holding in the main courtyard. To avoid getting held at a choke the enemy team decides to retake via high ground.

A coordinated team can drop simultaneously and stabilize which is very difficult to punish, but most T2 teams mess up and their Reinhardt drops very early while the rest of his team is still at the high ground. If that happens you rush the Reinhardt and kill him, it’ll force them to use cooldowns to save him, at which point you have a cooldown advantage and can snowball and close out the fight from there.

Another example is Oasis Gardens. If you play regular Goats you play on the point and the enemy team might set up on the high ground. Your D.va and Lucio can look for boops on those targets and call when they get someone, in which case you speed and blow up that target, forcing them to use their abilities early and defensively. You get a bunch of space and can continue chasing afterward, using your cooldowns to save yourself. If played correctly you should be able to win off that.

2. Punishing split targets
    If the enemy D.va is off angling or trying to dive your Zenyatta you can call that and collapse on her. If the enemy Brigitte doesn’t realize that her team is falling back and plays in front of everyone you can go and kill her. If the Zarya is focussing your D.va and doesn’t recognize she is out of position you can rush and kill her. This is something you learn with experience, but as long as you play proactive and don’t hesitate to engage on targets you think are out of position you will quickly develop an understanding for what constitutes an “out of position” enemy you can engage on.

This is something that everyone on the team should understand ideally, the more eyes are looking out for out of position enemies to engage on the better. A Reinhardt/Zarya/Brigitte might not see the D.va behind them since they are playing the front line, in that case, the Zenyatta will have to recognize that she is out of position and call it. People don’t have eyes on their back. This engage especially needs to be very quick, if someone is out of position you need to punish them quickly otherwise you might miss your chance.



One special mention is that engages on people out of position don’t always need a speed boost. If the D.va stands behind you without boosters you don’t have to use your amp, just walk to her and she’ll die. Every cooldown you use in Goats can get punished, so don’t use them if you don’t have to.

3. Engaging off of Zarya Bubbles
This is the big one. Everyone on the team needs to understand the dynamics of Zarya bubbles, it is the foundation that the majority of engages rely on. I will go into more detail about bubble usage later on, but to address the basics, if the enemy team used their Zarya Ally Bubble (the one she puts on teammates) and you still have yours, you engage.

There are exceptions of course, depending on ults and positioning, but those make up a small portion of situations. 90\% of the time, if the enemy team uses their bubble first (ideally defensively as a result of the pressure you apply by positioning better with chokes and corners, or by better managing your shield and armor as Rein, more on that later) you need to punish that.

Evaluating Engages
I need to clarify something about engages as well. Engages don’t have to result in a kill. If the engage results in the enemy Zen having to use Trans to keep his Rein alive, that’s a win. If the engage results in you guys capping the point on a KotH map, that’s a win. If the enemy team is forced to fall back through a choke that you guys can hold them in afterward, that’s a win.

Getting a kill is optimal, but if you guys engage and you use your Zarya Bubble, Brig Armor Pack and Speed Amp you need to stabilize. As mentioned earlier, every cooldown you use makes you vulnerable, so if you use all your cooldowns in an engage you need to start stacking back up and playing safe assuming you don’t get a kill out of it. Good teams will be able to kite some of your engages and then reengage with cooldowns while you just used yours.

Playing off of Discords is also a way to engage, but it is less reliable and impactful so I don’t consider it something you should engage off on its own. If you get a Discord on their Zarya for example and she doesn’t have self bubble and is playing pretty far up you can engage on her, but if you get a Discord on the enemy Reinhardt you might not want to commit to an engage (aka use bubble, armor pack, speed, etc) but rather pressure a bit harder.

To clarify, pressuring means going for a few swings on the enemy Rein to force him back and then holding shield, the main goal behind pressuring is to get them to use cooldowns. An engage is a commitment, once you engage you can’t back out without getting punished for it, you need to get an advantage out of it.






Playing the Line
Another way to engage is something I like to call “playing the line”. It’s something that I have learned from Fusions (to be fair a lot of my Goats knowledge came from watching Fusions play). As mentioned earlier, engages need to be synchronized and fast. That can be very difficult because if not everyone on the team sees the engage opportunity at the same time it can be very difficult to engage fast while being synchronized.

Either you rush the engage in which case some might fall behind if they didn’t see the engage opportunity by themselves, or you don’t rush it to make sure everyone knows that you want to engage, in which case the enemy team can react to them making a mistake and give up space to prevent you from killing them.

Playing the line is a simple way of having synchronized and hard hitting, fast engages. You mark a line of the map (a choke usually) and as soon as someone on the enemy team passes that line, you engage on that person. The best example for this is on Hanamura first point defense in a Goats mirror. Teams want to push through that choke on offense, they don't want to get stuck on it. But when they do they are very vulnerable, they don’t have any cover and are all stacked so each Reinhardt swing can potentially hit 6 people, whereas the defending team is a lot more split so it is physically impossible for the attacking Rein to get as much value as the defending Rein.

The problem is that if you don’t punish them fast enough for pushing through they’ll be able to push you back and start taking space and spreading out which you don’t want to happen. So you can call to play the line at the choke, and everyone on the team should know that as soon as they push through you engage on whoever is closest.

Let’s say you are on attack Hanamura and you manage to push through the choke and kill the Brigitte doing so. This allows you to take space behind the choke but it isn’t enough to win a fight. You make your way onto the point and get a tick, but the enemy Brig is back and you see the enemy team preparing for a reengage. In that case, the attacking team call to play the line at one of the entrances to the point, so that everyone knows to play aggressively to punish the enemy team for rushing through the choke.

Playing the line becomes less effective the better the opposing team is because those teams are more likely to bait bubbles or pressure safely, but against less skilled Goats teams or on certain points playing the line can be a solid defensive foundation that is extremely easy to execute and get results out of. You can’t do it all the time of course and it doesn’t replace a solid understanding of when to engage.

Engagements against Spam Comps
I’d like to address the matchup against spam comps here as well. On Uprising Academy we decided for reasons I will go into later that running Goats on first point defense (Can’t disclose the map) is optimal even though the enemy team could go with a Hammond Spam Comp and take the point. In that case, we didn’t play to first hold, we played to stall for time and build up ultimates for a strong second point because we knew that pushing through the choke with Spam comp is very difficult.

In that case, you don’t engage per se, you focus on staying alive, contesting point and only punishing them when they do something really stupid. If the enemy Pharah, for example, is playing too far up you can land a Discord on her and let your D.va kill her, but other than that you don’t really engage on anyone, they have the mobility to kite you so engaging just puts you out of position. When playing against a spam comp as Goats you focus on staying alive, turtling on the point and punishing people that try to touch the point.
Target Selection 
Let’s go over target selection after the initial kill as well. A lot of teams do an excellent engage, end up getting space or a kill, but then fail to close out the fight. This usually comes down to poor target selection.

First, Zarya is an amazing target if she is within range. If there is a Zarya and a Reinhardt both the same distance away from you, go for the Zarya. D.va and Zarya are major damage sources in Goats, and both of them deal less damage against Armor. Zarya doesn’t have any armor. She gets melted if she is rushed on. On top of that Zarya has no mobility. She can’t charge out, she doesn’t have rocket boosters to escape, she can’t wall ride or use a Trans to become invulnerable.

If you decide to engage on her she is screwed without a support ultimate (which the enemy team doesn’t really want to use after already losing someone, and even if they do, using your Trans for your Zarya means you won’t have it for Grav, which is extremely punishable as outlined earlier). Killing the Zarya early also takes out the enemy team's main damage source (Assuming she has high energy which she should and that you guys don’t let the enemy Rein swing into 6 people), so killing her reduces the chance the enemy team might come back from that initial engage drastically.

Killing the Zarya early also prevents her from farming Grav, which can delay the enemy reengage. Reinhardt is a great target as well of course, mainly because it gets rid of the shield, allowing your Reinhardt to land a free shatter in case the situation goes sour, and it also means that everyone on your team can shoot whomever they want without wasting damage into a shield.




D.va is a target you generally want to avoid killing early unless you absolutely have to. She has armor which makes her harder to kill, and if she has her boosters she can fly around a corner or behind the Reinhardt shield forcing you to select a new target, and selecting a new target means downtime where you guys are changing targets instead of actually killing something.

If she has D.va bomb she can also use that, D.va bomb is pretty disposable in Goats and farmed pretty quickly against a bunch of tanks anyways, and a bomb on the point will prevent you from playing aggressive, so demeching her when she has bomb doesn’t really get you anywhere either.

Going for a Zen is risky unless he is super close. If he is playing in the backline you don’t want to run past everyone just to try and kill him and have him use Trans. Lucio is a bad target in most cases, too much mobility, hard to kill, small hitbox.

Killing a Lucio is extremely high impact, but it’s too difficult to warrant the cost. Make sure to not even bother shooting him unless you guys are actively trying to kill him, it’s wasted damage that could go into someone more useful. Burning down the enemy Rein Shield is better than poking the enemy Lucio for 10 seconds without ever getting him below half HP. 









Engage Example
First fight Lijiang Control Center. Both teams play Goats and decide to go straight to the point. Harmony Orb is on the Reinhardt and he abuses the pillar in the center of the point for cover. He starts pressuring the enemy Reinhardt but hides behind the pillar if he takes too much damage. The enemy Zarya uses her bubble on the Reinhardt. Someone on your team immediately begins shouting “No ally bubble, kill Rein” and you use Speed Amp to run on the Reinhardt. Your Reinhardt holds M1, Zarya and D.va run past the shield with the speed and focus on dealing damage. Your Zarya does not use her bubble until your Reinhardt is low HP. Your quick engage burned through the enemy Reinhardt's armor, your D.va and Zarya are melting him and he dies. Zarya was right next to him so you call to go for her next. She has no mobility and melts. Everyone else on the enemy team is too far away to kill so you call to stack and hold up shield to push up to the choke to prepare your defense. 

Having ultimates adds a lot of depth to this. If that fight happened the enemy Zen might end up using Trans to save the Reinhardt. In that case, you simply start kiting back and look for the next mistake you can engage off of. If the enemy team ends up getting your Rein low after using Trans you can use your own Trans and push into the enemy team with a Trans advantage. Note that going back and stabilizing after their trans is preferred to just using trans either way because it gives you a chance of winning the fight despite the enemy team using an ultimate.
 
If they mess up their bubble, you engage, they have to use trans, you kite back, they mess up bubble again, you engage, you kill the enemy Rein and win the fight, then they used an ultimate to lose a fight. It hardly gets any better than that. If you use 6 ultimates and they use 0 then you didn’t really win the fight, you just moved the loss to the next fight. Not losing =/= winning. This is all very baseline and there are a lot of situational things that go into how these fights go, but that’s something that needs to be learned through experience. 
Ability Usage
The last 2 chapters were mainly focused on decision making and communication, I’d like to address the mechanical aspects of Goats next by going over the most optimal way to use each key ability within the team comp. It is important to note that as with engages, ability usage has a lot of depth and is very situational, it is impossible to find a simple way to explain ability usage that covers every single situation, and everyone that tells you “Just do this every single time” does not know what he is talking about.

What I will instead do is give you a baseline you can work off of with each ability, something that isn’t set in stone but will hopefully help you with understanding the thought process behind Goats ability usage more as you practice and progress.

Harmony Orb Usage
Let’s start with Harmony Orb Placement. There is a key mechanic within the game I need to explain first, and that is that 30 HPS is not just 30 HPS. The amount of healing done is independent of the type of HP being healed up. If you would have to rank the 3 forms of HP, it would go Armor>HP>Shields. Putting your orb on a Zarya will heal up 30 HP in a second, which can only block a maximum of 30 damage.

Putting it on a Reinhardt instead will heal up 30 Armor, which can ideally block up to 60 damage. So in an ideal situation, having your Harmony heal up someone's armor is comparable to having 2 Harmony orbs heal up someones HP. The actual value you get out of healing armor is rarely twice that of healing HP, but it can get pretty close at times, for example against D.va.


The problem with healing up armor with characters like Ana, for example, is that you can easily miss your window. If a D.va pummels a Reinhardt who still has armor, and Ana keeps shooting at him she will have a much easier time keeping him alive, but if she has to reload or doesn’t react quickly enough she might not be able to heal him up until he already lost his armor, at which point she has to heal against twice the damage.  

Not only that but, Reinhardt is also your frontline, he determines the space you control, he is the main damage source if the enemy Rein holds up his shield, he’s the one stopping the enemy Zarya and Zen from doing what they want. Making sure that he is always in a position to capitalize on enemy mistakes (as outlined in the previous chapter about engages) is vital, putting your orb on a Zarya because she is taking a bit of damage will give the enemy team a Window in which they might be able to break through your Reinhardt's armor, at which point healing him up might become so difficult that he has to fall back and give up space, or even worse, it will force your Zarya to use your bubble early, which gives the enemy team a mistake they can engage off of. 

Brigitte’s AOE healing and Lucio’s heal aura if necessary should be enough to keep the rest of the team alive. By default before every single teamfight you want to have your Harmony Orb on your Reinhardt, I cannot think of a situation in a Goats Mirror where you wouldn’t want to open up with Orb on Reinhardt.

You need to be extremely careful about swapping orb off of the Reinhardt, a good Reinhardt will always play under the assumption that he has the orb on him, so if he doesn’t have the orb, especially if you don’t communicate it, and starts applying pressure, he might get overwhelmed. Zarya falling back and letting her shields regenerate, or Lucio swapping to heal aura to keep D.va afloat are much less punishable than taking your orb off of the Reinhardt.

You should also avoid thinking of Harmony Orb as an emergency heal. It is not. It is at its strongest when it is placed preemptively on someone. If your Zarya is getting rushed on, putting orb on her might not save her. If your Brigitte overextends, she can get blown up especially if she is discorded before your Harmony Orb even arrives.

Zarya Bubble and Brigitte Armor Pack are your burst emergency heals, Harmony Orb is a key part of your Reinhardt's gameplay in Goats, and keeping it on him as much as possible is optimal. If you feel that your Reinhardt isn’t taking any damage and he won’t in the immediate future you can swap your orb to someone else like Zarya or D.va, but you need to keep an eye on your Reinhardt and avoid letting his armor go down at all costs, because the damage he will take afterward will snowball immensely.

Discord Orb Placement
Discord Orbs are pretty simple. By default, you want to get your Discord on the enemy Reinhardt and keep it on him if possible. Even if you decide to shoot someone else during a teamfight, keeping the Discord on the Reinhardt will prevent him from playing aggressively and dealing damage, while also potentially forcing the Zarya to bubble him to cleanse it, so she won’t have it for whatever target you guys are focussing.

Or vice versa, Zarya might bubble your current target so she won’t cleanse the Discord off the Reinhardt. If the Reinhardt is out of the fight then put your Discord on whatever the current target is, but having your Discord on the enemy Reinhardt even if you aren’t trying to directly kill him is incredibly valuable, the amount of pressure on him will prevent him from being truly effective throughout the entire fight.

Zarya Bubble Usage
Zarya Bubbles are so important that I considered putting them into their own chapter, misused Zarya bubbles are the #1 cause of lost team fights in Goats vs Goats mirrors, and understanding Zarya Bubble Usage is not just important for the Zarya player; everyone on the team needs to understand how to use Zarya bubbles.

First off, Zarya decides when to bubble. Some teams’ Zaryas tell their Reinhardt “Call when you want me to bubble you”. Don’t do this, I can see that some coaches might disagree with me here but my stance is firm, it is the Zarya players responsibility to determine the best time to bubble, and adding an extra layer of complexity will only complicate things and make them less efficient. From my experience having the Zarya bubble on someone else's demand is incredibly awkward and delayed and super inefficient all around. Don’t do it. 

Squeezing
I also need to explain squeezing. Squeezing is holding onto certain abilities until the last possible moment to gain an advantage. Zarya’s Ally Bubbles or Transcendence are the best examples of squeezing.  The later you use them the better, assuming that your team trusts you. If you tell your Reinhardt that you have bubble, he goes in, he gets low HP and then backs off just as you bubble him then he doesn’t trust you. This is unacceptable and something you have to address immediately.

Your Reinhardt needs to understand that when you say “I have bubble” that you are not taking your eyes off him and you will bubble him before he dies no matter what. The more trust your Reinhardt has the more you can squeeze your bubble, and the more you squeeze your bubble the more likely that the enemy Zarya uses her bubble first. It’s a battle of guts, whoever blinks first loses. If the enemy Zarya loses her composure and bubbles at 300 HP, you need to keep it and bubble at 200 HP; that’s how you can consistently outplay the enemy Zarya. 
Not only that, but squeezing also allows your Brigitte to get value out of her armor pack, and it prevents the enemy team from punishing you. Let’s say you go for an engage and you one-shot the enemy Brigitte. Your Rein keeps pushing and you bubble him while he is full HP. You didn’t squeeze your bubble, so you end up using it even though you didn’t need to use it at all. This gives the enemy Rein an opening to push into your Rein, pressure him and potentially kill him. 

You having your Ally Bubble prevents the enemy team from engaging, so you never want to throw it away. The only time you would want to bubble a Full HP Reinhardt is when a:) He doesn’t have a shield and stands in front of a D.va bomb or b:) he is getting pinned by a Reinhardt. In all other situations, you have to hold onto your bubble until he gets low HP to around 200 HP if not lower. The more you squeeze it, the more often you can get away with not having to use your bubble at all, and not having to use your bubble in Goats is fantastic.

So TL;DR, if you bubble your Reinhardt as soon as he runs in while he is full HP, you messed up and threw the fight. Period. Your bubble is the emergency button, don’t be squeamish. 

There are 4 cases of bubbles. Either you use your bubble first and they don’t use theirs, they use their bubble first and you don’t use yours, they use their bubble first and you use it a couple of seconds after, or you use your bubble first and they use theirs a couple of seconds after.







Case 1: You use your bubble first and they don’t use theirs

Worst case. Don’t let this happen. This usually happens if your Rein doesn’t have a harmony orb on him, if he doesn’t play the corner properly or if he puts his shield up too late and gets discorded and out pressured. If this happens you have to kite back, your focus at this point is to live at all costs. Best case you give up all the space you have, worst case the enemy team rolls over you and wipes you out. Do not bubble first unless you are forced to keep someone alive or you are 100\% confident that the enemy Zarya will have to use her Ally Bubble after yours to keep someone on her team alive, or if you are bubble baiting (more on that later)


Case 2: You use your bubble first and they use theirs shortly after

This is a good outcome but not great. This usually happens when someone on the enemy team is out of position, you engage but you don’t squeeze your bubble at all and use it on your Reinhardt while he is still full HP, and then a second or two later the enemy Zarya has to use her bubble on whoever was caught out of position. If you use your bubble all you need to do is pray that the enemy Zarya uses her bubble.

It’s like playing Blackjack and landing on a 16. You know you fucked up, all you can do is hope that the house screws up their draw as well. If she ends up using her bubble after yours you have an amazing opportunity that can put the fight in your favor if you capitalize on it. Since you used your bubble first you will also get it back first. This creates a small window in which you will have your bubble available while the enemy bubble is still on cooldown. If you use your bubble, and the enemy Zarya uses hers 3 seconds later, as soon as you get your bubble there will be a 3-second window in which you have a bubble advantage.

If this happens the call usually goes (assuming there is no other reason not to reengage like someone dying or you guys simply being in a position where you don’t want to push, or the enemy team kiting super far back and you don't want to do a long distance engage) “They used bubble, I can bubble you in 3...2....1...”. 

Your entire team needs to understand exactly WHAT this call means and WHEN you make it. They have to understand that it means the enemy Zarya used her bubble after yours, and they have to understand that they will push in.

It is important that the actual push happens before your bubble is available. Earlier I mentioned how important squeezing your bubbles are, the later the better. But the problem is that when the enemy Zarya uses her bubble after you, you want to use your bubble as soon as possible! It’s a bit of a dilemma, if you use your bubble immediately you will get the biggest bubble advantage over the enemy Zarya, but the bubble itself might not get any value because its placed on a full HP Reinhardt, or you squeeze your bubble and don’t end up using it by the time the enemy Zarya has her bubble back up.

This is very difficult to execute properly but if drilled and practiced enough it makes a huge difference. The Zarya counts down her self-bubble cooldown, and the Reinhardt has to go in in such a way that the Zarya can use her ally bubble on cooldown and get high value out of it.

So you have to go in slightly before the bubble comes off cooldown, trade damage until you are down to around 200-300HP, get bubbled by your Zarya as soon as she gets it and then use the few seconds where the enemy Zarya doesn't have bubble to break through the enemy Reinhardt's armor, take all the space and win the fight. 


Case 3: They use bubble first and you don’t use yours

This is the best case. If you manage to get this case to happen the enemy team just messed up, or you are applying so much pressure that the enemy team is forced to do it. This is pretty much the same as the last case except that you don’t need to do any countdowns, you just go in as soon as their bubble wears off, and that instead of a 2-3 second bubble advantage depending on how much later than you the enemy Zarya uses her bubble, you have a full 8 second Bubble advantage, which is more than enough time to win the fight.

If the enemy team uses their bubble first without you using yours and you don’t win the fights afterward some alarms need to go off. If you can’t punish the enemy team for making one of the biggest mistakes you can make in Goats you will struggle to execute more difficult win conditions. Make sure that your engages are synchronized and confident, that your orbs are placed correctly, that your target calls are clean, consistent and fast.

If you lose fights where the enemy team uses their bubble super early as Goats then addressing that has to be your number 1 concern. Don’t focus on anything else until you are able to punish this fundamental mistake.

Case 4: They use bubble first and you use yours shortly after

Not the worst situation to be in, but also not great. You need to try and punish the enemy team for using their bubble first of course, but you have to be very careful about them re-engaging on you. If you keep pushing in without getting a kill, and they know what they are doing, they will reengage and get a kill just as your bubble is about to come back up. Don’t let that happen. If you don’t think you can kill anything just start kiting back and create a neutral situation again to prevent them from taking advantage of the small window they have when they get their bubble back. 
Understanding these 4 cases is crucial, everyone on the team has to understand what to do in all 4 of these cases. If someone struggles identifying them, having your Zarya be more vocal about when bubbles are being used can help. 

Baiting with Bubbles
Now, most of the good cases rely on the enemy team making a mistake, but in some situations, you need to bait the enemy team into making a mistake. For example, if you are attacking first point Hanamura, you can try to rush through the choke, but if the enemy team is good at Goats they should be able to punish you as you squeeze through the choke.

In that situation, the only way you can consistently take the point is by punishing the enemy overusing ultimates and then overwhelming them with ultimates in the next fight. But you don’t want to rely on ultimates to win fights, if you can win a fight off of ability usage and save ultimates you should do that instead. This is where bubble baiting comes in handy. It’s a way to bait the enemy team into entering a Case 2.

It’s pretty simple, you do a fake engage, you go in, use your bubble on your full HP Reinhardt and int on their Reinhardt. But you don’t have any intention of actually taking space or killing anyone, you just throw yourself at them and try to get them to use their bubble after your “bad early bubble”.

If you go in and use your bubble early you might be able to overwhelm the enemy Rein, scaring the Zarya into bubbling him. At that point, you do what I just explained, count down your bubble and then engage in a way that allows your Zarya to both get a good squeezed bubble off while also using it off cooldown. 

Bubble Usage against Grav Combos
Lastly, I need to address Grav Combos and how that affects bubble usage. To keep it short, if the enemy team has a Grav Combo, Trans can only get you so far. If you get Brig bashed or charged as Rein your team will still most likely die. If you know that the enemy team is looking for a Grav Combo you need to be extra careful with your bubbles, especially your self bubble.

A lot of Zarya's like to use self bubble for energy before a fight, but if the enemy team has a Grav Combo they can punish you for self bubbling, and you won’t be able to shield yourself. Not only that, but self bubble as Zarya is an excellent tool to save your team. On Uprising Academy Iced did a fantastic job after a bit of practice of saving self bubble for Grav Combos, and then body blocking for his teammates by standing right next to the bomb and self bubbling at the last second. 

Matrix Usage
Let’s talk about Matrix usage next. It’s pretty simple, you need to balance using your matrix early so it starts recharging, giving your more matrix to use throughout long fights and so that you can keep your Rein alive when the inevitable initial wave of burst damage comes in with your engage (Zen 5 orbs can melt a Reinhardt if he drops his shield and the enemy Zen times it correctly, eating that 5 orb makes a big difference), while also keeping it to eat Grav.

If the enemy Zarya doesn’t have Grav you can use matrix to just keep your Rein alive during the engage. If she has Grav keeping your matrix off cooldown will make it much more difficult for the enemy Zarya to get a good Grav off. You can’t eat all Gravs, but if you force the Zarya to grav in a spot she doesn’t really live then that’s still good value even if you didn’t eat it.


Against comps like Ana Goats you need to use Matrix on the Ana at the start of an engage to a:) prevent her from hitting a nade and more importantly b:) give your team a window of opportunity where they can burst through the enemy Rein Armor and then snowball from there. Once you run out of matrix, body blocking is your main way of preventing the Ana from healing whoever your team is focussing. 

Reinhardt Ability Usage
Reinhardt Pin usage is situational as well of course, but there is only 1 situation in which you would consistently pin, and that’s as a follow up for Grav Bombs. Using pin outside of Grav follow up is not great in a Goats mirror, extremely easy to punish and usually forces your Zarya to use bubble first. If you are chasing after stragglers at the end of a fight it’s fine, but outside of that, it’s best to just swing and hold shield, deals more damage against stacked enemies anyways. 

When following up on a Grav Combo how you use your pin is extremely important. If you pin too early then the enemy Reinhardt will shatter you and have enough time left to shield the D.va bomb. So make sure to pin at the last second. You also want to make sure that you go behind the enemy team for your pin so you pin into your team. There’s nothing worse than doing a Grav Combo, them surviving because of Trans and saved Zarya bubbles, and you charging off into the middle of the enemy team, dying, and allowing them to win the fight without using any ultimates (besides Trans) after you just used Grav and D.va bomb. Take the time to walk around them. 

Firestrike against good D.vas will get eaten most of the time anyways, but understanding its purpose and value is still important. Firestrike is a way to soften up enemies, apply pressure, take space and bait bubbles. If you land a 5 man firestrike it’ll take the enemy team quite a while to heal back up, allowing you to push up and take space, ideally forcing them into a choke. Take Nepal Shrine for example, you walk up the stairs, throw a firestrike, hit all 6 of them. You apply pressure and all of a sudden you turned an even fight on the point into an uphill battle for the enemy team as they are stuck on the stairs in front of a choke while you guys have all the space you need on the point, allowing your Rein to abuse the corner next to the stairs and make the enemy team uncomfortable. Again, it is unlikely the enemy D.va won’t eat that initial firestrike, but it is important to understand the impact if she doesn’t, and how to react to it. 

Firestrike is also great for re engaging. It beats swinging the air, so if you are speeding onto someone you can use firestrike while closing the gap. Firestrike is also good for bursting down armor, if you land it on the enemy Reinhardt it can soften him up, so if you start pressuring him afterward he can’t trade with you because he will lose his armor before you, so if you land a firestrike you might not be able to engage off of it, but you can up the pressure significantly, which can lead to an enemy ally bubble being used, which in turn allows you to engage off of that.

Firestrike is also a good way of baiting the enemy D.va into using matrix so your Zarya gets a free Grav. Be careful about getting shattered or discorded of course, but throw a firestrike around a corner at the enemy team and either it will hit 6 people and you can up the pressure from there, or it’ll get eaten and your Zarya gets a free Grav off.

Using firestrike optimally is a bit more advanced and not really something you should focus on until your communication, ability to recognize engages and armor management are taken care of, but if you feel like getting your Rein play in Goats to the next level putting some thought into firestrike usage is very valuable.

Brigitte Bash Usage
Brigitte's Bash has 3 main uses, mobility, kill confirm and 5 orb setups. Mobility is self-explanatory, using it to create some distance between you and the enemy Reinhardt when you get low is completely fine and you’ll end up using it just to move around pretty frequently.

Using it to kill confirm is when your team starts pressuring the Reinhardt, and he doesn’t stop swinging/starts holding up his shield until he is very low HP. If you pressure him, he is at 250 HP, he is about to hold up his shield but you stun him, you can give your team that extra bit of time they need to burst him from 250 HP to 0 HP. That is an optimal case of course, most Brigitte's just use Bash as soon as they know it’ll land, which still generates value but is not as optimal as saving bash for when it increases your chances of confirming a kill.

5 orb setups are super min-max and the only times I have seen it executed properly was seemingly unintentional. It is ridiculously effective and can single-handedly win a teamfight, but the execution behind it has to be so ridiculously flawless that it really shouldn’t be something you think about until almost everything else Goats related with your team has been perfected.

It’s pretty simple, your Zen charges up a 5 orb, your brig does a countdown on the stun, stuns just as the Zen releases the 5 orb and the enemy Rein takes over 400 damage and dies shortly after. There is nothing funnier to watch in Goats than a Rein having no idea where his HP just went. I’ve seen it happen once and match chat after that was just a bunch of question marks by everyone in the lobby.

If your team can do this consistently send me a DM. I’d pay to spectate your scrims.

Brigitte Armor Pack Usage
Brig’s Armor Pack is very similar to Zarya’s bubble with 1 key difference. The first paragraph of this chapter. The highest value you can get out of your pack is 150 armor, translating to up to 300 damage healed. That is insane value. Use it too late or on someone without armor and you’ll only heal 150 HP. Use it too early and you’ll only give them 75 armor.

But if you use it just right, you can heal your Reinhardt and give him 150 armor, which turns the tide in a fight. Zarya and Brigitte both want to squeeze their abilities, it’s just that Zarya’s final squeeze is much later than Brig. Zarya wants to use her bubble at around 200-250 HP (if not even lower) on the Reinhardt, the Brigitte, on the other hand, wants to use it at around 300-350 HP.

In an ideal case, the Brig Armor Pack should be used before the Zarya bubble, and assuming Zarya, Brig, and Reinhardt play correctly and trust each other that is exactly what happens in the majority of cases. That’s why it’s so important that your Reinhardt trusts his team, even if he doesn’t get a bubble he will get an Armor Pack. He can’t engage, go down to 300 HP and then start turtling, Zarya won’t even get a chance to use her cooldown.

Ideally you engage, Reinhardt goes down to 300-350 HP, Brig armor packs him, he is back at 500 HP, he goes down to 200 HP, Zarya bubbles him, Harmony, Brig Passive and potentially Lucio healing heals him back to 300 HP at least giving him his armor, which will sustain him throughout the rest of the fight. That’s what should happen ideally. If it doesn’t something needs to be fixed. 

Armor Management
Lastly, I want to address armor management for Reinhardt. To clarify, the following mainly applies to the pressuring phase of Goats (Where you try to get the enemy team to use their bubble but you don’t actually engage yet and you don’t want to use your cooldowns).
If you are pressuring you want to go and swing at the enemy Rein but start holding up your shield when you are about to lose your armor. Hold up your shield early and you won't apply enough pressure for the enemy Zarya to break and use bubble, hold it up too late and you might lose your armor at which point the incoming damage will be impossible to counterheal by your team.

You want to get as many swings as you can without losing your armor, that’s how you maximize the amount of pressure you apply. Your teams should want to avoid having to use cooldowns on you because they will need them available for the engage, so while you are pressuring the enemy Reinhardt you rely solely on Brig Passive, Harmony orb and potentially Lucio Heal Aura.

If you notice that you don’t get the enemy Zarya to use her bubbles then you aren’t applying enough pressure and should look if you are managing your hp and armor correctly, and if you notice that your shield is breaking before you get an engage, you are most likely holding up your shield too soon and not taking advantage of your armor and your teams healing as much as you should. 

Assuming both teams play well a Goats Mirror shouldn’t be decided by which Reinhardt loses his shield first. If you do, you don’t apply enough pressure. 










Positioning
Disclaimer: After rereading this chapter and thinking about it I felt putting a small disclaimer at the start is important. Explaining positioning in Goats is very difficult without a specific VoD to go over or a specific map to cover. This chapter will cover a lot of examples rather than strict rules to follow.

I hope that the examples I give will give you a good general idea of what is important when positioning at Goats, but when working with a professional team I would address this completely differently, using specific examples from their VoDs and tackling positioning one map at a time.

If you are playing on a team and reading this, you have to realize that unless you sit down with your team and review your position specifically on your scrim Vods you won’t be able to make any progress. Communication, Bubble Usage, when to engage, those are all things that you just have to understand once and you can immediately start implementing it.

You as an individual player can improve those aspects of your play, but you cannot improve your team's positioning on your own. Positioning requires a lot of experience to work well and intuitively, and that takes time and effort through reviews. If you want to improve your personal gameplay then focus on the other chapters first, positioning is something that the entire team has to get better at together, it’s no good to have 1 player on a team understand positioning and the rest of the team has no clue. 





Positioning in Goats is dynamic, so first I’ll address the most important factors when it comes to choosing your positions in Goats, and after that I’ll go more into how to abuse these aspects. I won’t go into individual positioning much in this guide, mainly because it depends on a player’s playstyle and there are no true “correct” answers. I hope that what we talked about so far and what we will talk about in a bit will provide the necessary fundamental knowledge needed to make decisions about your individual positioning.

Abusing Corners
First, corners are vital in Goats. Standing in the middle of the road out in the open is not good whether you are attacking or defending. Corners provide multiple key advantages. They make it more difficult for the enemy team to engage on you, because in order for some of the characters, especially Zenyatta, to get Line of sight on you they have to push up all the way around the corner.

If you are out in the open you have nothing to run behind when the enemy team engages. Zarya Graviton Surges are especially vulnerable to this. If you are playing Zarya on the 2nd point defense of Rialto you usually hold on top of the stairs. If the enemy team decides to go highground to push you off the stairs you might want to abuse that enclosed area for a quick 6 man grav. But if you place the Grav too far inside the building no one on your team will be able to follow up.

You want to use your Grav to pull enemies out into the open. If someone gets too close to the door you can use your Grav to pull them out which makes it easy for the rest of your team to kill that target. Graving corners and pulling people out is also very valuable when the enemy team has a Trans, because the corner might split multiple enemies, preventing them from getting any Trans healing. 


Corners are also important to limit the damage you take, especially as Reinhardt. If played correctly you aren’t exposing yourself to the enemy currently, and when you do it’s only half your body, the other half is hidden behind the wall. You can land swings and apply pressure while taking a lot less damage yourself, and the little damage you take can easily be healed by the Harmony Orb. Talking about orbs, Discord Orb can also be dropped by running behind the corner. If you are discorded out in the open your Zarya’s bubble is the only way to get rid of it, and you don’t want to find yourself in that position. 	

Playing around Chokes
Some corners are better than others, and corners forming chokes are the best positions to play at. When defending you want to hold the corner around a choke, when you are attacking you want to prevent the enemy team from holding you around corners at chokes. Either by pushing up and setting up around a corner yourself, or by using bubble baits (or just plain pressure against worse teams) to give yourself an opener to push through.

Holding corners at chokes doesn’t automatically win you the fight, but it puts you in a position where it is very easy to apply a lot of pressure without getting pressured yourself. If the enemy team uses their cooldowns better than you, then playing the choke won’t make a difference, but assuming that 2 teams are at an equal skill level, the one holding the corner at the choke will out pressure the team stuck in the choke.

The corner next to the library on Kings Row second (the room with the mini at the start of the phase) is an example of a corner without a choke. Holding there is better than holding in the middle of the road (the long stretch between that corner and the choke at the start of the 2nd phase for example would be a horrible place to stand in), but it isn’t as good as holding right at the start of second phase at the large choke, because it is easy to stuff the enemy team in that choke.
As long as you don’t waste your cooldowns or get baited into using them, and you don’t mess up your ultimates (Getting greedy with them, whiffing them, panic ulting etc.) you will make it very difficult for the enemy team to push through.

Chokes are also phenomenal for Reinhardts because they allow you to swing more effectively. If you are all spread out and the enemy team is stuck in a choke you will land 4, 5 or 6 man swings while the enemy reinhardt is only hitting you. So you are farming shatter incredibly quickly while pummeling the enemy team down.

Dynamic Positioning + Example
It is important that you are dynamic with your positioning as well. If we take that Kings Row defense example again.

Let’s say you are losing a fight at the initial choke and you have to reset. In that case you don’t just want to rush on cart, you want to find a corner where you will wait for the enemy team. It is better to give them 5 seconds of cart push and then holding at a corner, then beginning to contest immediately but taking a fight out in the open, especially because good attacking teams in that situation would only leave one person on cart and then set up on the closest corner themselves.

Another example of this is Lijiang Control Center. Let’s say you have the point, you decide to hold close at the entrance to the interior next to the rover/satellite. You use your bubble first, so you have to disengage. In that case you want to disengage all the way back to the point and then hold at the corner leading into the point. You always want to hold at some form of choke or corner if possible, even if it means giving up large amounts of space.

Taking 2 good fights is better than taking 2 good fights and 1 bad fight, especially when you want to consider using ultimates. Certain ultimates require a bit of setup, bash shatter for example (will go more into that in the next chapter). 

Here’s how that would work. You would hold close on Control Center at the rover. You look for a Grav Combo but it ends up killing nothing. You decide to kite back because the enemy team has used trans. You set up at the corner on the point. As the enemy team approaches your Brig announces that she is looking for a bash shatter. She calls for a speed, does a countdown, runs past the Rein Shield and bashes him and you win off the shatter.

That’s a good example of solid repositioning and falling back, and midfight ult usage that wasn’t set up before the team fight. Those kinds of plays can win you crucial games, but a lot of teams would either zerg at the first choke after they use the Grav instead of just falling back to another solid defense position, or they would fall back, but not actually make a plan besides just standing there and pressing M1. You need to be able to adapt, when a plan fails, come up with another one. If there is a plan, there’s confidence. And confidence wins games.

Highground in Goats
Highground is a bit of a special case in Goats, it’s useful but can be punished situationally. The main use of highground is to put yourself in a position where it is impossible for the enemy team to engage on you.

On Lijiang Night Market, if you decide to just retake through main you will get held there by the defending team (It’s a corner and a choke, can’t get any better, the defense has a huge advantage and you should not attempt that route unless you have a solid plan with your ultimates). If you go via highground you decide when the fight starts. If you drop you can expect the enemy team to engage you. If they don’t they are bad and you shouldn’t have lost the point in the first place.

It also allows your Zenyatta to be in a relatively safe position. Placing discords is much easier from above and as long as you call for help when the D.va dives you you don't’ have to worry about anyone running you down.

Highground is a pretty in-depth topic and incredibly map specific, so I can’t cover all cases, but think of it as a great way to get into a somewhat neutral fight when all the alternative engages or retakes are bad, or if your Zenyatta feels like he is more focussed on staying alive then actually being useful. 

Taking advantage of Point Pressure
Point pressure factors into all these past points. Point pressure is how the offense forces the defense to play proactively. From what we talked about so far it seems that a solid defense almost always has a huge positional advantage, and they usually do, but point pressure is a great way to get a nice choke/point to defend. 

Let’s take Nepal Village for example. If the enemy team holds highground you can ignore them and go straight to point. At this point they are forced to contest, which you can punish them for. If you are attacking on Volskaya first point it can be difficult to beat very mobile comps because there is seemingly nothing to engage on. Look to push them off highground and then set up on the point, the defending team has to contest eventually, putting themselves in a position where you can punish them.

If you feel like you are stuck ask yourself what you can do to force the enemy team into awkward situations, and having to contest when a 6 man deathball is capping the point is always a bit awkward.



Off Angles and Fanning Out
Off-angling is something a bit more advanced and shouldn’t really be implemented until the fundamentals are rock solid. If you are very confident in your Goats, off angling can be a great way to turn wins into stomps. The difference being that a stomped team fight can be snowballed into another won teamfight afterwards. So you are potentially getting 2 won team fights just by playing well in the first one.

Off angling punishes the enemy team for playing aggressively, but if executed poorly will give them a free engage. Zarya and D.va are the main candidates for off-angling, although the Zenyatta can take slight off angles in very specific situations to be more effective.  If your D.va and Zarya are both taking opposing angles 90 degrees from the rest of the team (For smart people, the enemy team would be positioned at the center of the hypotenuse of a right-angled triangle, between the Zarya and the D.va and in front of the rest of the team, each represented by one corner) then any forward movement by the enemy Reinhardt will only make him more vulnerable to the D.va and Zarya.

If he pushes up too far he will get melted from 3 directions at which point his shield won’t help him. If his teammates focus down the off-anglers then there won’t be enough pressure on the frontline and the Rein will be forced to push back. So how do you punish D.vas and Zarya's that off angle? It’s about speed, communication and discipline. A D.va can be forced out quite easily, you just have to be careful about not getting yourself out of position in the process or having to use cooldowns for it which would give the other team an engage. If a Zarya plays too far up you can look to rotate towards her, positioning yourself between the Zarya and her team. She has no mobility so the enemy team is forced to go in to save the Zarya. You can focus out the Zarya and then abuse the fact that the enemy team had to engage when they really didn’t want to.


Off angling only works when everyone is applying pressure at the exact same time. If only the D.va is applying pressure and the rest is staying back then she will get forced out. But if you push up in sync the enemy team will be forced to rotate or give up space. Off angling won’t win you a fight, but it can allow you to turn a bad fight into a good one.

Take Lijiang Control Center for example again, you are pressuring each other on the point. No one has any real advantage here. D.va takes the window side, Zarya off angles from white, you slowly push up. The enemy Reinhardt is forced to start falling back, giving up more and more space. Eventually they are in the choke, at which point your Zarya and D.va can’t push up any further. You regroup and are now holding at a choke. You haven’t won the fight but you have point control and have a corner to abuse.

Ilios Ruins is also a great example of off angling and “fanning out”. If you win the fight you can set up in front of the stairs leading into the mega in front of their spawn. You can fan out around it, so if the enemy team pushes through the choke, not only will they get pummeled by a Reinhardt abusing the corner, they will also be surrounded right from the get go, putting them into a tough spot. Most teams would rotate to the side there (the one with the mega) instead of going through the main choke, but you can hold them there just as well. 

Positioning like that won’t guarantee you a fight win, but it will make the retake much more difficult for the attacking team, and if the retake is difficult they are forced to use ultimates, and most likely multipleto push through the choke and get back into a neutral position.  As mentioned earlier, Off-angling is very advanced, has a lot of specifics to it and is incredibly point specific. This just touched the surface and if you think this is something you’d like to implement in your games, take it one map at a time and make sure your entire team is on the same page as to when you want to do it, how, where, etc.

Off-angling can go right and you can reap enormous benefits, but it is very easy to make a mistake that the enemy team can punish, and all of a sudden you are in a 5v6 without your Zarya, or your D.va lost her mech or is low HP and she can’t really do anything for the rest of the fight unless you orb her, at which point your Reinhardt gets pressured to death.

Bodyblocking as D.va
Bodyblocking or causing a distraction is especially important for D.va players. Just shooting the shield or poking people from a distance doesn’t do much, but you can’t play dive D.va and jump their Zen over and over again, a good Goats team will kill you for using up your boosters to fly behind them. If you didn’t take a point of damage throughout an entire lost fight you aren’t doing everything you can to keep your Rein alive.

Try to position yourself in front of your Reinhardt to prevent the enemy Zarya from melting him. Fly behind the enemy Rein on an engage to prevent the Zen from landing a Discord on your Reinhardt. Losing your mech as D.va is almost always preferable to losing your Reinhardt (If you have a Grav Combo then keeping your mech is important, although even then you wouldn't want to do it at the expense of letting your Rein die unless you are confident the Grav Combo would win the fight despite his death) so if your Reinhardt is taking too much damage or he is scared of playing aggressive because he gets blown up, think about how you can minimize the damage he takes.






Positioning Example
Positioning is pretty difficult to fully understand so I wrote up a little example situation that hopefully explains how it all goes together.

Kings Row: You are playing the line on defense between the statue and hotel. You are using the statue as a corner and trade swings with the enemy Reinhardt. The attacking team rotates to the left of the statue, you play around the mini in the statue and try to hold your ground. The enemy team is stuck in a narrow area and your swings outvalue the enemy Rein’s, forcing them to use an early bubble. 

You call an engage and push into them, blowing up the Reinhardt. The rest of the enemy team falls back throughout the choke. You chase instead of calling a stack, push through the choke and your Brig overextends and dies. You fall back to the statue and call to wait for the Brig to come back. The enemy team regroups and meets you at the statue. The 6v5 applies too much pressure so you make the call to fall back, give them point and hold around hotel in the back until Brig comes back. The attacking team gets a tick and you know you aren’t in a position to hold for much longer, you don’t really have a corner or choke.

You decide to walk on the point. At this point you guys trade ultimates between each other and you end up losing the fight (if you won the fight you would just set up at the statue again). The attacking team caps the point, puts their D.va on cart and the rest pushes up to the library corner on second to avoid getting held at the choke. You go for a quick regroup and push up to the library corner where they are holding. The attacking team can’t really fall back until they push the cart through the choke otherwise they’ll be stuck, so they decide to take the fight at that corner even through it’s not great to play that far from their D.va. You abuse the fact that the D.va isn’t there (If they have the Zen on cart it’s almost the same thing really, if you play the corner the Zen won’t be able to do a whole lot, if he even is in range that is).
The enemy team tries it’s best to hold that ground but realizes the defense isn’t making a mistake they can engage off of so the attacking team falls back to the point. The cart has already pushed through the choke so you decide to push into them during the rotation and punish them for it, the Rein gets low, gets bubbled, you guys engage off of that and you win the fight. Even though the cart has already pushed through the choke you just walk past it and hold at the choke. 

You have both Grav and Shatter, since you won the last fight by playing disciplined and punishing them instead of just throwing ults at them. You call to hold the choke, look for a bash shatter first and if that doesn’t work go for a Grav(More on ult economy in a bit). Brig 3rd person peeks the corner and calls “Speed boost me”, she runs up, does a quick “3-2-1” (Because someone has asked me, that doesn’t have to be seconds. You can call whatever you want and the speed is dynamic. Sometimes the 3-2-1 is over in half a second, sometimes it takes 5 seconds because the enemy Zarya used bubble so the brig stretches the countdown out until it is over) and then bashes the Reinhardt.

If the bash slam lands you push through the choke and kill them. If it doesn’t land you don't want to immediately throw in a Grav. You can call “Look for an engage, use grav combo if we really have to”. You trade with the enemy Rein a bit and continue playing the corner. You see that they are speeding in looking for a bash shatter, so your Zarya goes for a Grav to prevent that play. The enemy team has pushed out into the open, so they are exposed in the Grav and die despite using Trans.

The enemy team regroups, you just used Grav Shatter in the last fight and your ult economy isn’t great but the Rein got enough swings into the Grav that allowed him to get another shatter. You don’t feel confident that you can win the fight if they get to engage with their ultimates, so you call another proactive bash shatter. You go for it, same as last time, it misses, and you call to give up the choke and go all the way back.
The enemy Zarya tries to grav you while you fall back but your D.va eats it. You let them push cart for free and set up at the corner around the library. You don’t have any ultimates but they just whiffed Grav, so you just play the corner, let cart be stuck there, and pressure their Rein. You are on defense so they are wasting their time. You continue this throughout the rest of the map.

Positioning contextualized
So in short, always look for good positions to hold even during fights when you are getting pushed back. Take the map geometry and where you are holding into account when planning ultimates. If you are on offense and you are stuck in a choke, use bubble baits and proactive ultimates to push your way through those chokes.

If you win a fight on offense don’t just stand on cart, think about where the next fight will happen and make sure you set up around that corner preemptively. Fanning out (Off Angling) and stacking are more advanced strategies you can practice later on, but the basics of positioning are incredibly important.

If you take fights in poor positions and blame the lost fight on your mechanics you won’t get anywhere. If fights are feeling too difficult don’t just assume that it’s you guys doing something wrong during the fight, if a fight isn’t properly set up and planned then the fight will be incredibly difficult to win no matter what. 

Here’s the order of win conditions so to speak. Positioning sets up ability usage, the better your positioning, the better your ability usage can be. If both teams have equal ability usage, then the team with better positioning will be able to force the enemy team to use their cooldowns first. But Positioning won’t win you a fight on it’s own.


You can have really good positioning but even with that advantage, your ability usage could be so much worse than the enemy teams that despite your positioning you still end up having to use cooldowns first (Maybe you don’t apply enough pressure because your Rein just holds up shield, or maybe your Zarya panic bubbles too early, or maybe the Zen decides to put Harmony on his Zarya or D.va for no reason).

If both teams pressure each other and never give each other an opener to engage off of then ultimates will decide the fight. If 2 teams are always fighting with equally good positions, and they are both perfectly using their abilities, then the team that used their ultimates more efficiently last fight will have an ult advantage this fight that they can capitalize on.

But you always want to punish bad positioning first, then you want to push bad ability usage, and only if neither of them are punishable do you want to rely on an ultimate to win you a fight (For example if you can’t get into a good position on a retake, or if you won’t have time to pressure out abilities because you know the enemy team will proactively engage and not even have time to make any mistakes). 

This chapter consisted mostly of examples, but that’s what positioning in Goats is all about, very simple rules and facts that are applied dynamically in different situations. It’s not just about understanding that holding chokes and corners is good, it’s about understanding which ones to hold, when to hold them, when to fall back, when to push up, and all of that depends on multiple factors, your cooldowns, your ultimates, the enemy teams ultimates, whether you lost someone or not, whether you got a kill or not, etc.



Improving at Goats is not about perfecting one single aspect of it. It’s about addressing each issue one step at a time. You can focus on your positioning up to the point where you understand the basics and you have a good idea of where your default positions to play around are on each map, but then you’ll start facing diminishing returns in optimizing your positioning, so you want to address bigger issues like incorrect ability usage, poor ult economy or inefficient communication first.

It’s better to be decent at all aspects of Goats, than to be phenomenal at one aspect of Goats, but bad at all the other aspects. Make sure that you always work on the biggest issue you have at that time. If your positioning is generally fine, but you still struggle with positioning on say Oasis University, but your Zarya is repeatedly wasting her bubbles and using them too early, then that’s an issue that needs to be addressed first. Poor bubble usage will lose you a lot more fights on all sorts of maps then not being confident in your positioning on 1 particular point of a KotH map.

Don’t be a perfectionist at the start, get everything up to a fundamental level, and then you can delve deeper into certain aspects of Goats. Don’t start min-maxing until you have a solid foundation.	





Ult Usage
The Importance of Ult Management 
Ult Usage and management is vital. It’s not something you should focus on first, ability usage and engages are more important if you want to start winning maps as Goats, but managing your ultimates will make your life a lot easier and add consistency to your play.

If you are a team that seemingly does really well but then has 1 fight that they get stomped in and then get snowballed from there? That could happen if you have solid fundamentals but your ult economy gets messed up. A good Goats Team won’t have to rely on beating you in terms of ability usage if they can bait you into using 3 ultimates and then win with 2, every single fight. Correct ult usage sets you up for ridiculously easy fights and big rewards. Good ult usage can make the difference between being able to win a defense fight in 20 seconds, or winning it in 60 seconds. 60 seconds being much better of course because it gives the enemy team less time, and as such less opportunities to win.

It’s the difference between winning a fight and having a 20\% chance of winning the next fight, or winning a fight and having a 80\% chance of winning the next fight. Good Ult Management isn’t always immediately apparent, but it’s one of the most underappreciated factors of Goats and if mastered will make you so much more consistent. Sure, you could win a team fight where both teams are equal in terms of ultimates available just because you have good positioning and you are good at using your abilities, but it’s much better to have 4 ultimates when the enemy team has 1, because in case something does go wrong, in case someone on your team messed up or an accident happens, you have something to fall back on.



Good Ult Economy is like going to the arcade with a trash bag full a pennies. You won’t be better at Pacman, but if you have enough pennies to continue playing after making a mistake then you’ll get much farther than much better players that don’t have anything to fall back on when they make a mistake. It also helps with confidence.

If you play Pacman and your dog dies when you lose the round it will be much rougher to perform because the stakes are higher, but if you can always fall back on putting another dime in you’ll be able to think more clearly and act under less pressure. Don’t worry if I lost you at that analogy, the point is, even if you don’t end up using all the ultimates you have because of good ult management, just having them available gives you some peace of mind that you otherwise wouldn’t have.

Support Ultimates in Goats
Let’s go over Beat and Trans first. Beat is delayed, limited in the amount of value in can generate, takes very long to charge up. But it prevents people from getting oneshot. If your Brig is pinned a Trans won’t save her. A Beat will. Other than that Trans is generally the ultimate you want to use first. It charges faster, it is instant so it can react to people getting bursted much more quickly, and it’s healing is figuratively speaking neverending. If you are in a Graviton Surge your beat can be damaged away, a Trans on the other hand will keep people alive throughout the entire Grav, besides some discord situations. Unless there is a Reinhardt Pin or a D.va bomb, or the enemy team can somehow LoS the Trans from your teammates, they can’t really get a kill.

Beat is more of a backup ultimate, you use it when you don’t have a Trans available in a fight, or when you need to save your Trans. For example, if you see that your Rein is getting pressured hard, using beat to keep him alive is better than using Trans because using Trans would give their Zarya a free Grav afterwards.
That said it can be difficult to use Beat to save someone from dying through regular damage because it’s delayed. In general you want to save trans for Grav whenever possible, and use beat in other cases when you can. If someone is low HP and both support ultimates available you want to use beat to keep them alive ideally. If you get shattered you want to try and react to it with a Beat. Having to use Trans to keep yourself alive as Zen is something you want to avoid under all circumstances, because it sets the enemy team up for a high value ultimate afterwards.

Communication and Ult Stacking
You also have to make sure that you communicate about your support ultimates. If you have both beat and trans just talking before the fight and saying something along the lines of “You Trans Grav, I’ll beat everything else” is very helpful. Just reminding each other that both of you have their ultimates can avoid a lot of accidents. Stacking support ultimates is unacceptable at even the lowest level of pro Overwatch.

Offensive ultimates are similar. If you used Grav in a fight you need to be incredibly cautious about using Shatter, because it will leave you with no ultimate whatsoever for the next fight. If you used shatter first then using Grav afterwards is fine as long as the Rein can follow up on it and you are confident it will make a difference in the fight. So many teams end up using both Shatter and Grav at the same time, or Shatter after Grav, and then they end up wondering why they got overwhelmed in the next fight.

Don’t postpone losses, if a fight is lost because you messed up your Grav then don’t shatter unless you are very confident it will win the fight. Losing a fight after using both Grav and Shatter is a death sentence against good teams, they will snowball and even with good ability usage it will take some time to recover from it, especially if they are disciplined with their ultimates and make it difficult for you to get back to a neutral or advantageous ult game.
How to get value out of Shatter
Setting up offensive ultimates in Goats is incredibly important. Let’s start with shatter. You cannot rely on “outbraining” your opponent to land shatters, you can’t just wait for a firestrike and then shatter, or predict when he will start swinging and go for a shatter. Setting up shatters properly is very important in Goats, both before a fight and during a fight.

The only time you would want to shatter without any setup is when multiple opponents are split and not behind the Reinhardt shield, if the enemy Reinhardt shield breaks or if the enemy Reinhardt is dead. The 2 main ways to set up shatter is via Lucio Boop and Brigitte Bash. The Brigitte Bash setup is pretty simple, you choose a corner to hold (if it’s called mid fight it’s fine if the bash shatter starts behind the Rein shield, although a corner is still preferable so it’s less telegraphed), you call that you are looking for a Bash, you call a speed boost, you run past the Reinhardt shield and go for a bash.

That is the basic way to set up bash shatters, but there is a lot more complexity to it. 



Bash Shatters
Let’s go over the 3 possible outcomes of a speed bash shatter. This is under the assumption that the bash was executed properly, aka the Brigitte ran all the way behind the Reinhardt, the speedboost wasn’t late, and the bash wasn’t telegraphed and set up behind a corner and executed quickly. Either the Reinhardt keeps facing your Reinhardt, in which case the bash from behind goes through and your Rein can shatter, or the Reinhardt turns around to block the Brig Bash, in which case your Reinhardt can land a free shatter into Reinhardts back. Or the Reinhardt faces your Reinhardt, but gets bubbled so your bash doesn’t do anything.
If you see that the enemy Rein gets bubbled you don’t want to bash, if you are doing a countdown you want to simply stretch out the countdown. It’s fine to go “Three....Two.....Bash after bubble” and then bubble as soon as the bubble fades. Don’t feel forced to bash on 1 just because you started a countdown. Be dynamic. 

In some situations you can increase the consistency of your bash shatters by abusing the enemy team being split. Let’s say you are playing Kings Row Second Point (If you haven’t noticed, I prefer using the same maps for explanations so it doesn’t get as confusing) and the defending team has set up around the choke to hold you there. They are pretty spread out behind the choke, and the Reinhardt is abusing the corner. In that case, when calling for the speedboost with the Brig Bash, the Reinhardt runs in as well. Both Brig and Reinhardt are charging past the enemy Reinhardt, and your Reinhardt simply shatters the rest of the enemy team. Your shatter can’t get blocked if you just run past the shield.

This also gives your Brigitte some extra safety because she isn’t going in alone. Keep in mind that this only works if the enemy team is playing far away from their Reinhardt shield. If they are all playing next to a corner, or at the wall, or all stacked up, you can’t really force this play. But if the enemy team is very spread out then running past the shield and shattering their backline can be a good idea. Keep in mind that shattering the backline does NOT mean you kill the backline. 

If you run past the Rein shield and land a 4 man shatter, you don’t want to engage on someone you shattered if they are far away. Let’s say the enemy Zarya and Reinhardt are holding close to the corner, and the rest of the team is playing behind them. You go for the speed, and you shatter the 4 people behind. Only Zarya and Reinhardt are still standing. In that case just kill them while the rest of their team are sitting on the ground not being able to do anything. No speed amp, no armor pack, no defense matrix. Be careful about not getting counter shattered when you go for shatter plays past the enemy Rein shield and you can punish a lot of less experienced teams for not setting up their defenses properly.
Boop Shatter
Boop shatter is another way to set up your shatters, although it is usually only applicable on defensive set ups. Doing boop shatters offensively is difficult to set up, inconsistent and outright risky. It can be done in some clutch situations but it shouldn’t be a plan you make before a fight. The main advantage of Boop Shatters is that they are much more difficult to Zarya bubble. If executed quickly the Reinhardt is flying by the time he gets bubbled, allowing you to land the shatter.

Keep in mind that the booped up Reinhardt may not get shattered, so don’t just run in blindly, as soon as he lands he could go for a countershatter. You should also be careful about booping in the wrong targets as Lucio. If you boop up 3 people then those 3 people can’t get shattered. Getting the Rein up in to the air doesn’t do anything if the rest of his team is up there with him. Boop shatters are usually done as lurks, you are defending a corner further back, while your Lucio is hiding in a corner further up. The enemy team walks past that first corner, gets booped, then your team comes around the second corner and goes for the shatter.

You can do this with a Lucio because of his mobility, you might have to use an Ally bubble but you can usually get the Lucio out of there alive if it doesn’t work, so it’s pretty low risk if set up properly. Brigitte on the other hand needs to be more proactive, she can’t just wait for the enemy team to walk past her or she might die, even with Ally bubble.

As a Lucio player you can go for backup boop shatters as well. If you see your Brigitte set up a stun shatter you can piggyback off her countdown and boop at the same time. Sometimes Brigitte players don’t quite make it behind the Rein Shield and they don’t get a stun off even without the enemy Zarya using her bubble, so you booping at the same time gives an extra chance to set up the shatter.
A Reinhardt has to trust his team and shatter when the countdown ends unless its aborted, and messing up a stun shatter combo without the enemy team using a bubble is unacceptable, but having the boop as a backup can give you some consistency while your Brigitte gets better at setting up bash shatters.

Communication and Individual Responsibility
To add onto that, when landing a shatter communication is vital. Don’t get greedy, kill whatever is in front of you and call a target. Every engage needs to start with a target call, and landing a shatter is no different. If you call “Huuuuuuuuuuuuge shatter” you’re embarrassing yourself. Call a target, no one cares about how huge the shatter was, they want to kill someone and win the fight. You can talk about how big your shatter was after the round is over. (I know quite a few Reinhardt players that will feel attacked by that paragraph).

Before we get into Grav and D.va bomb usage, something that affects both offensive ultimates. Everyone involved is responsible. If I go into a team that doesn’t get any value out of their shatters, then that is not only the Reinhardts fault. It is the Reinhardts fault, the Brigittes fault, the Lucios fault. If a team doesn’t get any value out of their Gravs then that is not only the Zarya's fault. That is the D.vas fault as well.

As Brigitte it is your responsibility to speak up and say “I can’t go for a bash shatter there, we need to set them up over there instead”. It is your responsibility to say “Reinhardt, your are throwing out your shatters randomly all the time during the fight so I can never set up a bash shatter. Call that you have shatter and I will do a countdown for you”. If your Zarya gravs while your D.va bomb is at 80\%, it is your responsibility to tell her “I am close to D.va bomb, let me farm it and we look for a combo”. On the other hand, if the D.va just throws out hero bombs all the time and you never have any follow up on your Gravs, you have to tell your D.va to get her shit together and work with you. You’re a team, teamwork makes the dream work, if something isn’t working, talk about it.
If you are telling yourself “It’s not my fault D.va always throws out random bombs” or “I can’t land a shatter if my Brig always messes up her bashes” you are fooling yourself. You are just as responsible for the low ult value as your teammate, don’t shift the blame, fix it. 

Graviton Surge and D.va Bomb
Grav bombs are pretty simple. Call that you are looking for a Grav Combo, and if there are no objections, look for the grav as soon as you find an opportunity. Don’t force the combo of course. If you are defending a choke you don’t have to Grav early, hold onto it, let them waste their time, use it when they try to push through and make a play.
 
If the enemy wants to do nothing, let them do nothing. They are the ones that have a clock ticking over their heads, not you. When you got for the Grav your D.va places it in the Grav. Not 10 feet in front of it. Not 5 meters behind it. Not 3 light years above it. Right next to them. This is important for multiple reasons. For one it might mess up the enemy Reinhardt, it can be difficult to block the D.va bomb for everyone if it's right on top of everyone. Zarya bubbles also play into it. If it’s too far away then one Zarya self bubble might end up bodyblocking her 3 teammates and keeping them alive.

Boop Follow-up
But most importantly, boop follow-ups are incredibly difficult if not impossible with bombs that are placed too far away. Timing and positioning are crucial when throwing D.va bomb into a Grav. If you are just pressing shift and launching your bomb into the Grav without any thought, you are doing it wrong. Take your time, fly into them if you have to, make sure your bomb is right in there. Boop follow-up makes the difference between 2 wasted ultimates and a won teamfight.

Do not D.va bomb immediately after the Grav comes out. D.va bomb takes 3 seconds to explode. Zarya Graviton Surge lasts 4 seconds. If both are launched at the same time then the bomb will explode before the Grav expires, making it impossible for the Lucio player to boop them into the bomb. You want to launch your bomb around 2 seconds after the Grav landed, that way it will detonate 1 second after the Grav ends. You don’t want the bomb to end just as the Grav expires, you need a bit of extra time so the boop can actually launch people somewhere.

Aiming the Lucio boop is crucial, which is why the bomb placement is so important. You want to boop the enemy team THROUGH the D.va bomb. If a Reinhardt is shielding the D.va bomb, and he gets booped through it, he has to turn his shield 180 degrees in order to block it. A lot of Reinhardt's don’t react quickly enough to that. Or if they do, they mess it up somehow and someone else dies. If the bomb was placed too far away then it is impossible for the Lucio to boop the enemy team through it at the end of the Grav. If you are playing Lucio, and you can’t boop the bomb you have to talk about it. “The bomb is too early”, “The bomb is too far away”. Talk, fix it. 

Brigitte and Reinhardt Follow-up
Brig Bash and Reinhardt Pin are 2 more ways to follow up on a Grav Combo. Bash is self explanatory, bash the shield away and they can’t block it. Not very consistent however since the Zarya might bubble the Reinhardt. 
Reinhardt Pin needs to be aimed correctly. Always pin into your team if possible, even if the Grav Combo doesn’t kill anything you’ll have an opponent in the middle of your team completely separated.

That Zarya bubble is why Lucio Boop is so key. If the enemy team is stuck in a Grav and there is a D.va bomb next to them, they might Zarya bubble the Reinhardt. If the Lucio goes for a boop he will boop everyone except the Reinhardt. His teammates are no longer behind his shield, but in front of it.
It’s a dilemma. If executed properly a Grav Combo should almost always result in the bomb getting kills. 3 ways to follow up on it, try to counter one and you set up the other one. The only consistent way of dealing with a Grav Combo is to save bubbles beforehand, have Trans and try to play a bit split so they don’t get 6 people in it. If you don’t do one of those things you can get punished. And if the enemy team doesn’t do one of those things then you have to punish them for it. 

Getting value out of ult combos is very important, otherwise you’ll have to use more ultimates to salvage the fight, and you want to win fights with as few ultimates as possible while getting as many ultimates as possible out of the enemy team.

Brigitte Rally Usage
Rally Usage is a quick one. Don’t be greedy with Rally. Unless the fight is lost, you want to rally. If you rally, tell your team to stack. It is your responsibility to get everyone armored up, if your Zenyatta is playing in Kazakhstan it is your responsibility to book him an overnight flight right into Rallytown.

If you can move slightly and get him in your aura without someone else losing his armor then you should do that of course, but if you armoring Zenyatta would prevent you from armoring other people, you have to tell your Zenyatta to get moving. Rally makes fights longer, longer fights tend to have more ultimates used in them. Mainly because advantages can’t be pressed as hard.

If you have a Rally going, it is much harder for the enemy team to get the Reinhardt to an HP level where he would need a bubble. And if he never needs a bubble you will end up brawling and brawling and brawling, getting more and more ultimates, until one team snaps and ends up throwing something out to close out the fight. If you rally and the enemy team doesn’t you just walk into them and start dealing damage, they won’t be able to trade with you.
Keep in mind what I talked about at the start of this guide about how armor affects healing. That armor that Brigitte is giving isn’t just a flat aoe heal like Lucio’s Aura. It’s armor, which is very effective against Goats. Your Zenyatta can play much more aggressive, D.va can’t solo kill him if he stands in a Rally. If you win a fight without using Rally, you could have won it without using as many ultimates as you did. If you lost a fight without using Rally you could have stalled it out longer, getting more capture \% on a KotH map or waste more of the attacking teams time in timebank, and you might have forced them to use more ultimates than they wanted to because you were harder to kill.

Rally makes a big difference, you want to use it as much as you can. Especially since it is charged so incredibly quickly.

Ult Economy and swapping from/to Goats
Ult economy is also important when choosing your composition. Sometimes you want to pick your composition not based on whatever wins you that particular point, because it might help you later down the line.

Let’s say you are attacking first point on Volskaya against a Dive Goats defense. You decide to go Hammond DPS, which is a favorable matchup against Goats. You slowly spam them out and if executed properly it’s not a question of whether you take the point, it’s when you take the point. But then you get on second and you can’t run Hammond DPS anymore, you don’t have enough point presence and the enemy team can’t get spammed out with all that cover and being that close to spawn. So you switch to Goats. But the enemy team has all their ultimates, they only had to switch from Winston to Reinhardt.

So you end up losing a lot of time just catching up in ultimates. And a good team will make it very difficult for you to catch up in ultimates. So you capture the first point in 2 minutes, but then get held on second. In that case you might want to reconsider and run Reinhardt Goats on first point.
It is still a slightly favorable matchup (I will go into why Reinhardt Goats is better than Winston Goats in a bit) and while it is not as easy to take the point with Rein Goats as it is with a DPS comp, it means that you have a good shot at snowballing the second point with Ultimates.

Another example, let’s say you are playing Nepal Village. You are opening up with a spam comp, and the enemy team is going Goats. You lose the team fight but you have a bunch of ultimates so you decide to stay on that DPS comp instead of switching to Goats. But if you don’t swap now then you won't be able to get ultimates by the time the round is over. So you cap the point as DPS, and as soon as you lose it you swap instead of trying to retake. Sure, maybe you could retake with DPS characters, but you can’t change your mind and switch to Goats later. It’s better to win the round even if it means throwing away a few ultimates by swapping.
 
That is a very situational example that depends a lot on which teams are facing which, but the general idea is that you always have to take into account that just swapping to Goats isn’t going to solve your problems. The best time to switch to Goats is at the same time as the enemy team switches to Goats. Retaking as Goats into a Goats that has all their ultimates while you have none is a death sentence. 

That’s why Goats variations are so popular. Because you can swap off them without much repercussion. Winston Goats? Swap 1 Hero and you are back on Goats. Sombra Goats? Same thing. Ana Goats? Same thing. Brig McCree? Same thing. You get the idea. If you stay within the ecosystem you get rewarded. Ult economy is in my opinion one of the main reasons Goats is still meta and will be viable for quite a while, the full family of Goats Comps is super flexible and very easy to switch between.



Goats Variations
Disclaimer
This guide hasn’t covered everything Goats related so far, the sheer depth and volume of information surrounding Goats is very difficult to summarize and present. This also applies to the chapter on Goats Variations. The following paragraphs won’t cover every single Goats variation there is, and the few that I do mention are only explained at its most fundamental level. I want to quickly address the strengths and weaknesses of each variation, so if you are looking to run any of these variations yourself you might have to look at different resources to get more details on them.

One basic concept that is shared by all variations of Goats is that you are always giving something up. If you run a Moira you won’t have a Zen. If you have a McCree you won’t have a Zarya. If you have a Sombra you don’t have a D.va (or Brigitte, the variations have variations!). Whenever you substitute a character you can’t just look at what that character brings to the table, you have to take into account what you are losing out on. On the same hand, if you notice that the enemy team is running a Goats variation, the secret to beating them is to figure out what they had to sacrifice in order to make the comp work, and then play around that weakness.

Sombra Goats
Let’s start with Sombra Goats. D.va is generally substituted, although some teams have tried swapping Brigitte or even Zarya as well, but I personally don’t think that’s optimal. The support those 2 characters provide to their Reinhardt is simply too important to warrant replacing them. Removing Brigitte takes away your ability to go for bash shatters, and I hope that after reading through the previous chapters I don’t have to explain why not having a Zarya in Goats is simply unacceptable. 

As with each variation there are pros and cons. Sombra’s hack is incredibly effective against tanks because they are very cooldown dependent. EMP is incredibly strong against Goats, because it is very easy to hit multiple targets, and every target is severely handicapped by being hacked. Compared to running Sombra against say a dive comp where it is much more difficult to get high value EMPs. Goats also allows Sombra to farm her EMP incredibly quickly, especially if she harasses the D.va and abuses her gigantic headshot hitbox. Both Zenyatta and Zarya get into critical health territory when they get EMP’d. Sombra also adds a lot of consistency to Goats. You usually rely on pressuring the enemy opponent to get them to use their cooldowns or give up space. With Sombra you don’t have to rely on the enemy team making a mistake, setting up a hack against Goats as Sombra is fairly trivial and provides you with a great opportunity to engage off of. 

Looking at it so far Sombra Goats seems to be a pretty strong alternative to regular Goats. But when you look into it deeper and start taking into account it’s weaknesses and what you have to give up, it quickly becomes noticeable that a well played regular Goats will beat a well played Sombra Goats consistently. For one, your gameplay in Sombra Goats is incredibly telegraphed. A good D.va can make it difficult to get a hack off and get in position. On top of that, you are losing out on D.va. One less body to eat damage, no matrix to eat Gravs or damage, no D.va bomb follow up on Graviton Surges. On top of that playing Sombra against Lucio Zen, the default support duo of Goats, can make it difficult landing EMPs because if the supports position themselves properly, one of them should be able to use their support ultimate to counter the EMP. 

With Sombra Goats you also rely on the hack. Without the hack, a regular Goats comp will simply be more effective than Sombra Goats, D.va is much more consistent, has a higher damage output if played correctly and the potential damage she could deny with her matrix is much higher than the damage a Sombra can deal.
Take into account that Goats is very armor heavy, especially when a Rally is being used, which makes Sombra much less effective damage wise. If you land a hack as Sombra Goats you have a window of strength where you can overpower the enemy team, but if you don’t get that hack, you will lose a brawl against a regular Goats team 100\% of the time assuming both teams are of an equal skill level and no ultimates are involved.

That should answer how to play against Sombra Goats. Fast. You cannot prevent the Sombra from EMPing, you can’t always prevent her from hacking someone. But you can make it awkward for her. Sombra wants to set up hacks on her own terms, she wants to use EMP where she wants to, when she wants to. You need to be quick, against Sombra Goats you don’t have to wait for the enemy team to make a mistake. They picked Sombra Goats, that’s a mistake that will last throughout the entire map. You are at a constant advantage up to the point where they land a hack. So if you play fast, abuse your more consistent composition and just engage and apply pressure, the enemy Sombra will be forced to go for a quick hack, or EMP early. This prevents her from scouting out the supports, making it easier for you to counter the EMP with your support ultimates. If the Sombra is forced to rush her hack she will have difficulty getting a good target. Catch them off guard and go in before they can execute their game plan. You won’t be able to win every fight, but winning every fight is not what counts, winning the game is what’s important.

Mei Goats
Next, let’s address Mei Goats. You don’t have a D.va, which makes you very vulnerable to Ana Goats, and similar to Sombra Goats you rely on one trump card. Mei Goats is easier to play than regular Goats assuming that you have a decent Mei player (the same with Sombra Goats) but the maximum effectiveness is not comparable to regular Goats.

If you land a wall and communicate it properly you can get an advantage and beat regular Goats. But if you miss even a single wall, you will get rushed on and won’t be able to win the brawl against regular Goats assuming the opposing team is at an equal skill level, has no ult disadvantage and doesn’t make any unusual mistakes. 

Mei Goats is also very situational. Not a lot of maps really enable her, the only map that we found her to be viable on last season was Lijiang Control Center, and even there running regular Goats was simply the better choice. Mei Goats is one dimensional, easy to abuse, and simply doesn’t have a lot of depth in it. And a composition without depth doesn’t enable it’s players to really shine and make plays that set them apart from their opponents. You never really stomp with Mei Goats. You don’t win, the enemies just lose. I personally don’t recommend running it outside of some very specific cases and if you are very confident that you have a solid Mei player.

There is one situation where running Mei Goats can be the objectively correct choice however, and that is in retakes against Mei Goats. I can only comment on Lijiang Control Center here, but if you roll out regular Goats against a Mei Goats, and you lose because you make a mistake, running regular Goats on the retake can prove to be very difficult and not worth the effort. A well played Mei Goats can make it almost impossible to push through the choke, the only way to consistently get through that choke against Mei Goats is to have a mei yourself, turn your wall so it is being placed in a straight line in front of you, and then place it right in the middle of the choke. If the enemy team is holding on the right side, you walk through the choke along the left side of the wall, and if they are holding on the left side you are walking through the right side.



Keep in mind that this doesn’t prevent the enemy team from getting a good wall afterwards, but it does allow you to consistently get through the choke. This interaction with Mei Goats is a symptom of poor map design in my opinion, Control Center is very difficult to retake because there are so many chokes to go through, which suits Mei well. 

Brig McCree
Brig McCree is one of the most viable Goats Variations. Some might not call it a Goats Variation, but I personally believe that it’s playstyle is similar enough to Goats. You still want to play somewhat stacked and push on targets together. Brig McCree is generally being played with a Winston. The point behind the composition is to enable the McCree in any way possible, take space for him and apply point pressure as much as possible. It is very strong against Dive compositions especially. 

Brig McCree is great for teams that are confident in their McCree and who have a Reinhardt that doesn’t feel comfortable playing Reinhardt against dive heroes. It’s very important to note that Brig McCree should not be played like a dive comp. The Winston and D.va are taking space, it’s not their responsibility to be a death squad walking all over the place picking people.

If your McCree isn’t doing anything, you have to talk with each other and figure out why. If the Zenyatta on highground keeps discording him, you have to force him out. If the enemy Tracer keeps diving him, you need to keep an eye on him when he calls for help. Do not get split, staying together is what prevents flankers from being effective against Brig McCree.

Dive Winston Goats
Winston Goats is played differently from Rein Goats but it is not a dive comp. some players call Winston Goats “Doats” or “Dive Goats” which I think is an incorrect way of describing the comp.
You don’t really dive with Winston Goats. You don’t have the follow up for it, you don’t have a Genji or a Tracer that can push in with you and deal damage. Winston Goats is all about abusing chokes and being flexible.

Winston Goats is inferior to Rein Goats in a straight on brawl. Winston Goats simply cannot compete with the damage of a regular Goats comp. For one, Winston’s Tesla Cannon is not very effective against armor compared to Reinhardts swings, and not having a shield means that your Zarya, D.va, Zenyatta and Lucio will have to shoot down a 2000 HP shield while the enemy team can spam for free. Regular Goats will get ultimates faster, deal more poke damage and be more consistent at confirming kills.

So why would you run Winston Goats? For flexibility and for it’s defensive ability against Goats. Let’s take Volskaya defense for example. As mentioned in the chapter on Ultimate Usage, playing Goats Variations makes swapping incredibly easy. There are generally 2 things that could happen on Volskaya. Either the attacking team plays Rein Goats, or they play some form of DPS comp.

Winston Goats can stay alive much longer against a DPS spam comp, and if played correctly you can punish enemies out of position a lot easier. Even if you end up losing the fight, if played correctly you’ll be able to stall for a very long time, being able to simply switch to Rein Goats on second point without losing out on a lot of ultimates, giving yourself an advantage when the enemy team has to swap every hero to go Goats on the second point. 

Winston is also incredibly strong at abusing chokes. Goats is a very stacked composition, so if the Winston plays his chokes right he can get primal incredibly quickly and apply too much pressure for the attacking team to handle. Keep in mind that on Volskaya first a Rein Goats on offense is still expected to win against a Winston Goats on defense, but Winston Goats is a safer pick that doesn’t simply allow the attacking team to go DPS and expect a free win.
And because the attacking Goats team is forced to walk through so many chokes (through main, through garage, through the door on highground etc.) the Winston can deal tremendous amounts of damage and get his ultimate very quickly.

Winston Goats is a great option when you are worried the attacking team might switch to a spam comp, and if you are on defense being concerned about your second point ult economy. If you are just playing Lijiang Night Market and you expect the enemy team to run Rein Goats you should go Rein Goats yourself, unless you have chokes you can abuse it is going to be very difficult to win fights against regular Goats. Not impossible, but difficult. And if your Main Tank finds it easier to play Winston Goats into Rein Goats than just playing Rein Goats himself, then you need to look into why your Main Tank player doesn’t know how to play Reinhardt. Having a shield, burst damage and an ultimate that synergizes with the rest of the team (Being able to farm it in grav, setting it up with brig bash) is better than having a bit of mobility. 

Winston Goats is not inherently bad, and I am not saying that Rein Goats is objectively better than it in all situations. I could have written a Guide of the same length about Winston Goats and go in depth about all the things you can do with it, and these few paragraphs don’t even begin to scratch the surface of how Winston Goats works. Make sure to do your own research on Winston Goats before you decide on whether you want to invest time into the comp, fully understanding the advantages and disadvantages of Winston Goats is important if you want to make smart choices about what you run. Don’t run or not run a comp simply because someone told you to, be inquisitive and curious and learn more about the game at every opportunity thrown your way.


Moira Goats
Running a Moira instead of a Zen in Goats is generally run as an answer to spam compositions. You don’t have to worry about the flex support dying, because killing a Moira is 10x more difficult than killing a Zenyatta, and the extra healing allows you to sustain the spam damage for longer. That said whenever I looked at Moira Goats I concluded that the damage output of the comp is simply pathetic. 
Killing anything with Moira Goats proves to be incredibly difficult and playing Moira Goats into regular Goats is not viable, you end up having no damage and swapping out Trans for Coalescence is not even remotely comparable. 

If you are playing something like Nepal Village and you are certain the enemy team is running a spam comp, you can open up Moira and turtle on point. But at that point you might as well run a DPS comp of your own or go Brig McCree to counter the composition directly. 

Ana Goats
Ana Goats is a decent answer when you know the enemy team will run Mei Goats, but at that point you might as well run Mei Goats yourself or simply play regular Goats and have a more solid comp in the long run. If the enemy team has a D.va it will simply be too difficult to get value out of the Ana, your Rein will get punished every fight while the enemy D.va holds her matrix at you. Nano Boost is surprisingly solid but so is transcendence. Landing anti nades is incredibly high value but again, landing those consistently against a D.va is challenging. And without the nade the damage output of your comp is just too low. You don’t have any burst from 5-orbs or consistent damage with your Discord orb. Run Ana Goats into Mei Goats if you are very confident in your Ana player, in all other situations there is generally a stronger answer available than simply swapping your Zen with an Ana. 



Closing Remarks
This guide is the product of months of studying and experiencing Goats at a professional level first hand. Fully understanding all of the concepts and implementing them takes a lot of time, but I am very happy with how it turned out and I hope you enjoyed it. If you have any questions or inquiries, simply contact me via the contact information at the bottom of each page. My twitter is @ioStux, my Discord is @ioStux#0304, my email is coaching@iostux.com and my username on reddit is /u/ioStux. 

Feel free to join my Discord, it is the easiest way to get in touch with me: https://discord.gg/YBzfqMT

While I understand that I haven’t been able to cover all cases and go through all possible situations and examples, I had to draw the line somewhere. I could go on and on, day in and day out about how to play Goats, but I feel like I am already pushing it with the length being as long as it is, and committing your time to a 500 page guide is not something I can expect.

Feel free to share this document with players you know that are interested in playing Goats at a competitive level, I hope that some of you will be able to apply this in your scrims and achieve more consistent performances as a result of it. 

My name is ioStux, and I’d like to thank you for learning.	
